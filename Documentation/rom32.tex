% 12/28/01
% $Id: rom32.tex,v 1.3 2002/04/16 15:19:09 ellard Exp $

\section{Introduction}

This section describes the default Ant-32 implementation and the
default ROM.  The default ROM supplied with the implementation of the
Ant-32 architecture contains a boot routine for initializing the
machine and several utility functions to simplify writing small
programs.

\section{Hardware Overview}

The default machine has 4 megs of physical RAM, contiguous from
physical address 0 to physical address {\tt 0x3fffff}.  Because of the
manner in which physical memory is addressed in (unmapped) system
mode, this means that this RAM appears to begin at virtual address
{\tt 0x80000000} and ends at {\tt 0x803fffff}.  Other RAM sizes are
possible, however, so it is a mistake to assume that this is always
the amount of RAM available.

In addition to the RAM, there are 4 pages (16K) of ROM located at the
top of the physical address space.  This small area of memory is where
the default ROM is located.

The details of the ROM are best described in the source code for the
ROM itself, and readers interested in more detail should refer to it. 
The source for the default ROM is provided as part of the normal
Ant-32 distribution.

\section{Initialization}

When the machine is booted, if the memory image was constructed in the
usual fashion (as described in {\tt aa32\_notes.html}), a short
initialization routine located in the ROM is called before execution
continues with the main program.

The boot ROM assumes that the memory image for the main program has
already been loaded into memory, starting at memory location {\tt
0x80000000}.

Note that there is nothing sacred here-- all of the initializations
done here can be overridden by the main program.  The purpose of the
routine supplied in the ROM is simply to supply reasonable defaults so
that it is, for many purposes, unnecessary to override anything.  The
only really important thing that the main code needs to take into
account is that exceptions are enabled by the ROM.

The steps taken by the initialization routine are as follows:

\begin{enumerate}

\item	{\em Determine the size of physical memory.}

	This is done by iteratively probing each page of RAM, starting
	at physical location 0 and continuing until either the address
	space of physical RAM is exhausted (at 1 Gbyte) or an invalid
	page is encountered. 

\item	{\em Initialize the {\tt sp} and {\tt fp} registers.}

	The frame pointer and stack pointer are initialized to point
	to the ``top'' of physical memory (via addresses in the
	unmapped segment).  Note that because of the way that the
	stack operations are implemented, the initial location pointed
	to by the frame pointer and stack pointer is actually one word
	{\em past} the end of physical memory.

	The initialization code assumes that it knows how much RAM is
	actually present.  It is possible to write this routine in
	such a way that it first detects how much memory there is in
	the machine, but this has not been implemented yet.
	
\item	{\em Prepare for Exceptions, and Zero the Cycle Counters}

	First, the exception handler is set to the address of a
	routine located in the ROM (named {\tt antSysRomEH}) that
	prints an error message and halts if a run-time exception
	occurs.  This is a minimal exception handler (since it
	doesn't really ``handle'' exceptions, it just makes the
	results a little less messy).

	Next, the {\em exception disable} flag is cleared, permitting
	exceptions to occur.

	Finally, the cycle counters and registers used by the
	probing routines are set to zero, 

\item	{\em Call the Main Code}

	The call to the main code of the program is implemented in the
	same manner as a zero-argument function call, so that if the
	``main'' of the program returns, this code will be able to
	properly halt the machine.

\end{enumerate}

\section{ROM Routines}

The functions in the ROM use the calling conventions described in
Chapter \ref{advanced-programming-chap}.  For routines that require
more than one parameter, the parameters are listed in the order that
they should be pushed onto the stack.

\subsection{Memory Management}

\begin{description}

\item[{\tt antSysSbrkInit}]

	Set the initial address of the boundary (aka {\em break})
	between preallocated memory and memory available for dynamic
	memory allocation.

	Note that all memory {\em after} this boundary (at higher
	addresses) is implicitly assumed to be available for dynamic
	allocation, which is not a completely accurate assumption,
	because the stack is also located in this region, and grows
	down towards the break.  If the break and the frontier of the
	stack cross, disaster is very likely.  Detecting this
	situation without adding costly overhead to every {\tt push}
	requires advanced techniques not described here.

\item[{\tt antSysSbrk}]

	Takes a single argument {\em size}, which is the amount to
	move the break.  The previous value of the break is returned.

	If the {\em size} is positive, the break is advanced,
	effectively allocating memory.  If the {\em size} is negative,
	memory is deallocated.

	Note that the {\em size} is always rounded up (towards
	positive infinity) to the nearest multiple of 4, in order to
	ensure that the break is always properly aligned for any
	memory access operation.  This can cause confusing behavior
	when trying to deallocate a small amount of memory.  For
	example, using {\tt antSysSbrk} with a size of 1 advances the
	break by 4 bytes (allocating 3 extra bytes), but using {\tt
	antSysSbrk} with a value of -1 does not move the break at all,
	so no memory is actually deallocated. 

\end{description}

\subsection{Simple I/O Routines}

\begin{description}

\item[{\tt antSysPrintString}]

	Print the zero-terminated {\sc ASCII} string pointed to by the
	argument.

\item[{\tt antSysPrintSDecimal}]

	Print the argument as a 32-bit signed decimal integer.

\item[{\tt antSysPrintUDecimal}]

	Print the argument as a 32-bit unsigned decimal integer.

\item[{\tt antSysPrintHex}]

	Print the argument as a 32-bit hexadecimal integer.

\item[{\tt antSysReadLine}]

	Read characters until end-of-line or end-of-input is reached. 
	(The behavior mimics the {\tt fgets} function from the
	standard C library.)

	This routine takes two parameters, which are pushed onto the
	stack in the following order:

	\begin{description}

	\item[buffer length] The maximum number of characters to read
		from the console.

	\item[buffer address] The address of the buffer to place the
		characters read from the console.

	\end{description}

\item[{\tt antSysReadDecimal}]

	Read characters from the console and interpret them as a
	32-bit signed decimal number, which is returned.

	Invalid input characters (such as non-digit characters, or a
	number too large to represent in 32 bits) will result in an
	arbitrary value being returned.  No error checking is
	performed.

\item[{\tt antSysReadHex}]
\index{antSysReadHex}

	Read characters from the console and interpret them as a
	32-bit hexadecimal number, which is returned.

	Invalid input characters (such as non-hex characters, or a
	number too large to represent in 32 bits) will result in an
	arbitrary value being returned.  No error checking is
	performed.

\end{description}

