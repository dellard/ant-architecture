% $Id: ant32-init.tex,v 1.3 2002/04/16 15:19:09 ellard Exp $

\chapter{Initialization}

When the Ant-32 processor is initialized or reset, the following steps
are taken:

\begin{enumerate}

	\item The processor mode is set to supervisor mode.

	\item The exception-disable flag is set.

	\item The interrupt-disable flag is set.

	\item All of the TLB entries are set to be invalid, by setting
		each entry's valid bit to zero.  The value of the rest
		of the bits in each TLB entry is undefined; no
		assumptions should be made about their initial values.

	\item All of the general registers are set to zero.

	\item The cycle counter registers ({\tt r240} - {\tt r248})
		are set to zero.

	\item The supervisor-only registers ({\tt r247} - {\tt r251})
		are set to zero.

	\item The timer is set to zero and disabled.  Any pending
		timer interrupts are discarded.

	\item Any pending console I/O is discarded.

	\item Execution begins at the address stored in virtual
		address {\tt 0xfffffffc} (the address of the last word
		in the unmapped segment).  This corresponds to
		physical address {\tt 0x3ffffffc}.

		If memory location {\tt 0xfffffffc} cannot be
		accessed, then the behavior of the processor is
		undefined.

\end{enumerate}

Therefore, when the processor boots, it is in supervisor mode, and it
begins executing from the address that has been loaded into physical
address {\tt 0x3ffffffc}, using unmapped virtual addresses.

The value of the exception handler address is undefined after a reset. 
The {\em leh} instruction should be used to load an exception handler
address before the exception-disable flag is cleared.

The contents of the exception register are undefined after a reset.

It is assumed that the firmware located at in the top page of physical
memory will install its own exception handler during bootstrap in
order to be able to handle device interrupts while initializing and
loading an operating system.

