%%% 02/26/94
%%% An attempt at writing my own macros...

\newcommand{\codesize}{\small}

%%% algorithm environment:
%%%     Arg 1 is some description of the algorithm.
%%%     Arg 2 is the reference name of the algorithm.
\newenvironment{algorithm}[2]{
        \begin{figure}[hbtp]
        \hrule
	\vspace{2mm}
        \begin{alghead}
        \label{#2}
        {\em #1}
        \end{alghead}
        % \hrule
}{
        \vspace{3mm}
        \hrule
        \end{figure}
}

%%% dan-code environment:
%%%     Arg 1 is the name of the program or code fragment.
%%%     Arg 2 is the reference name of the code.
%%%     Arg 3 is any additional text to add to the preface of the code.
\newenvironment{dan-code}[3]{
        \newpage
        \section{{\tt #1}}
        \label{#2}
        \index{#1 (complete listing)}
        #3
        \vspace{3mm}
        \hrule
        \vspace{3mm}
        \footnotesize
}{
        \normalsize
        \vspace{3mm}
        \hrule
}

\newcommand{\marginnote}[1]{\marginpar{{\bf {\footnotesize #1}}}}

%% The variables must be set before any of the following macros
%% can be used.  They should be assigned in the root of the document,
%% not here, however.
%%
%% UseDefaultAntOpcodes - True if the Ant uses the default opcodes
%%	(whatever they happen to be at any given instant...) instead
%%	of some variation.  Currently, the only variation is for QRR,
%%	which omits sys and adds halt and in/out.
%%
%% AllowUnwriteableDes - The current ANT faults if the program tries
%%	to write r0 or r1.  If this bool is true, then the processor
%%	does not fault.  I just silently ignores the write, leaving
%%	r0 with zero (always) and r1 with whatever it would have been
%%	had the destination register been something else...
%%
%%	As if 1/19, the default is to fault.  However, this is very
%%	likely to change in the future.

\newboolean{UseDefaultAntOpcodes}
\newboolean{AllowUnwriteableDes}

%%% end of dan.tx
