% $Id: ad8.tex,v 1.5 2002/04/16 01:02:37 ellard Exp $
%

\section{Getting Started With {\tt ad8}}

The ordinary way of running {\tt ad8} is to invoke
{\tt ad8} with a single command-line argument, which is the name of
the {\tt .ant} file you wish to debug: 

\begin{verbatim}
        % ad8 myfile.ant
\end{verbatim}

In response, {\tt ad8} will print out the first instruction in the
program, and then prompt the user for more instructions.  For example,
if the first instruction in {\tt myfile.ant} was {\tt add r2, r3, r4},
then the screen would look like:

\begin{verbatim}
	0x00:   add  r2=(0x00), r3=(0x00), r4=(0x00)
        >> 
\end{verbatim}

The first number on the line (in this case, {\tt 0x00}) indicates the
address of the next instruction to execute.  In this case, since we
have loaded the program but have not executed anything yet, the next
instruction the {\sc Ant-8} will execute is the first instruction in the
program, which is always the instruction at address 0.

The number in parentheses following each of the register names is the
current value stored in each register, expressed in hexadecimal-- in
this example, all of registers {\tt r2}, {\tt r3}, and {\tt r4}
contain the value zero.

Whenever {\tt ad8} is waiting for you to type something, it will
display the next {\sc Ant-8} instruction that it will execute (unless
you tell it to jump to some other part of your program), followed by
the \verb$>>$ prompt.

All {\tt ad8} commands end in a newline; {\tt ad8} will not
carry out any commands until you press the return or ``enter'' key.
Furthermore, only one {\tt ad8} command can appear on each line.

The most important command to remember is {\tt h}, which prints the
{\em help} screen.  All of the {\tt ad8} commands are listed on the
help screen, so if you forget the names or uses of any of them, you
can always find them again here. 

Most of the commands consist of a single character 
followed by the enter key.  Several of the commands, however, also
allow an {\em address} to be supplied.  This address specifies
the location in data memory or instruction memory that the command
will act upon.  The address can be specified as an integer (in
octal, decimal, or hexadecimal notation) or as a label.  If the
address is omitted, an address of 0 is used.

\subsection{Other Uses of {\tt ad8}}

Ordinarily, {\tt ad8} is used for interactive debugging.  It does
have two other modes, however: it can simply run the Ant-8 program and exit
when the program is finished, or it can disassemble the program and then
exit.  These modes can be specified on the command-line:

\begin{center}
\begin{tabular}{|l|l|}
\hline
{\bf Flag} & {\bf Meaning} \\
\hline
\hline
{\tt -r} & Run the program, and then exit. \\
{\tt -d} & Disassemble the program, and then exit. \\
\hline
\end{tabular}
\end{center}

If both the {\tt -r} and {\tt -d} flags appear on the
command-line, whichever appears {\em last} determines which mode is
selected.

Finally, if a {\tt -h} flag appears on the command-line, {\tt ad8}
only prints a short help message and then exits. 

\section{ {\tt ad8} Commands}

\subsection{Debugger Commands}

\begin{description}
\item[\Large \fbox{\tt q}]

	Quit {\tt ad8}. 

\item[\Large \fbox{\tt R}]

	Reload the program and reinitialize memory.

	If you are editing and reassembling the program in another
	window, there is no need to exit {\tt ad8} and start again
	with the new version:  you can use the {\bf R} command to
	quickly load the new program.

	The {\bf R} command is also useful for resetting the
	contents of the data memory and the registers to their
	initial state.  Note that the {\tt r} command
	(described in the next section) {\em does not}
	reset the contents of memory or registers.

\end{description}

\subsection{Running a Program}

\begin{description}

\item[\Large \fbox{\tt r}]

	Run the program, starting from PC = 0.

	The program will run until it executes a {\tt hlt}, detects a
	fault of any kind, or a breakpoint is reached.

\item[\Large \fbox{\tt g}]

	Start (or resume) executing the program, starting from
	wherever the PC currently points.  Very useful when
	used with the {\bf j} command (described next) or
	breakpoints (described later).

\item[\Large \fbox{\tt j {\em addr}}]

	Jump to the specified address.  Sets the PC to
	{\em addr}.

	This command is particularly useful, in combination
	with {\bf g}, as a way to test a small section of a
	program, instead of executing the entire program.

\item[\Large \fbox{\tt n}]

	Execute the next instruction and then stop.

\item[\Large \fbox{\tt t}]

	Toggle trace mode.  If trace mode is off, then turn it
	on; if on, then turn it off.

	When trace mode is on, {\tt ad8} prints each
	instruction and its address before executing it.
	This provides a {\em trace} of the execution.
	It is particularly useful for finding bugs like
	infinite loops.

\end{description}

\subsection{Examining the Ant-8}

\begin{description}

\item[\Large \fbox{\tt p}]

	Print the contents of the registers.

	The contents are displayed in the following format:

{\small
\begin{verbatim}
r00  r01  r02  r03  r04  r05  r06  r07  r08  r09  r10  r11  r12  r13  r14  r15
 00   00   00   00   00   00   00   00   00   00   00   00   00   00   00   00
  0    0    0    0    0    0    0    0    0    0    0    0    0    0    0    0
\end{verbatim}
}

	The first line of the output lists the register
	numbers (in decimal).  The second and third lines show
	the values in each register:  the second line in
	hexadecimal notation, and the third line in signed
	decimal.

	Note that this display is designed to fill an
	80-column screen.  If your display is less that 80
	columns wide, this output will be considerably more
	difficult to read. 

\item[\Large \fbox{\tt d}]

	Displays the values stored in memory as data, in hexadecimal
	notation.

	The {\tt d} command has three forms:

	\begin{description}

	\item[\fbox{\tt d}]

		Print the byte values of every byte of memory.

	\item[\fbox{\tt d {\em addr}}]

		Print the value of the byte stored in the given {\em
		addr} in memory.

	\item[\fbox{\tt d {\em addr1}, {\em addr2}}]

		Print all of the byte values starting at {\em addr1}
		and continuing to {\em addr2} in memory.

	\end{description}

\item[\Large \fbox{\tt i {\em addr}}]

	Disassemble and print instruction stored in memory.

	Like the {\tt d} command, the {\tt i} command has three forms:

	\begin{description}

	\item [\fbox{\tt i}]

		Disassemble and print all of the instructions in
		memory.

	\item [\fbox{\tt i {\em addr}}]

		Disassemble and print the instruction at the given
		{\em addr} in memory.

	\item [\fbox{\tt i {\em addr1}, {\em addr2}}]

		Disassemble and print all of the instructions starting
		at {\em addr1} and continuing to {\em addr2} in
		memory.

	\end{description}

\end{description}

\subsection{Setting Breakpoints}

{\em Breakpoints} provide a way to stop execution at any point in the
program.  The typical use is to set a breakpoint at the start of an
interesting part of the program, and then run the program (using the
{\tt r} or {\tt g} command).  The program will execute until the
instruction at the address of the breakpoint is about to be executed,
and then stop.

\begin{description}

\item[\Large \fbox{\tt b {\em addr}}]

	Set a breakpoint at the specified address.

\item[\Large \fbox{\tt c {\em addr}}]

	Remove the breakpoint (if any) at the specified address.

\item[\Large \fbox{\tt C}]

	Remove all breakpoints (if any).

\end{description}

