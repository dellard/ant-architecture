% $Id: ant-spec.tex,v 1.7 2002/01/02 02:53:15 ellard Exp $
%
% Macros for printing the info for each instruction in a regular manner.
%
% This is a useful hack.  Instead of just scribbling out the
% spec/description/etc for each opcode as it appears in each place in
% the document, we put a little effort into trying to make this more
% modular.  For each instruction, we define a bunch of attribute variables--
% such as its mnemonics, numeric opcode, brief description, etc.  Then these
% can be invoked at will.
%
% The shortcoming is that these attributes are all bound to global variables,
% so there can only be one value visible at any instant.
%
% So, here's how it typically works:
%
% 1. Unset all the variables, using the \INSTreset command, defined here.
%
% 2. Set all the variables (or the subset of variables that are relevant)
%	to whatever instruction you are documenting).  This is typically
%	accomplished by \input'ing one of the files in Arch32/ or Arch/.
%
% 3. Display the values in whatever canonical style you wish.  Typically
%	only a subset of the variables is shown.  One canonical style is
%	defined here: 


\newcommand{\INSTmnemonic}	{}	% asm mnemonic
\newcommand{\INSToptype}	{}	% type of op.  Must be one of:
					% ThreeReg, TwoReg, OneReg
\newcommand{\INSTopcode}	{}	% numeric value of opcode
\newcommand{\INSTfieldA}	{}	% First field of instruction
\newcommand{\INSTfieldB}	{}	% Second field of instruction
\newcommand{\INSTfieldC}	{}	% Third field of instruction
\newcommand{\INSTdesc}		{}	% brief mnemonic description
\newcommand{\INSTsemantic}	{}	% Concise semantics.
\newcommand{\INSTprose}		{}	% Prose explanation.
\newcommand{\INSTexceptions}	{}	% Exceptions the inst can generate


% Should always do an \INSTreset before setting up the values for
% a new op.  Otherwise, the previous values will linger, and if they
% are not reset can pop up in unexpected places.

\newcommand{\INSTreset}		{
	\renewcommand{\INSTmnemonic}	{}
	\renewcommand{\INSTdesc}	{}
	\renewcommand{\INSTopcode}	{}
	\renewcommand{\INSToptype}	{}
	\renewcommand{\INSTfieldA}	{}
	\renewcommand{\INSTfieldB}	{}
	\renewcommand{\INSTfieldC}	{}
	\renewcommand{\INSTsemantic}	{}
	\renewcommand{\INSTprose}	{}
	\renewcommand{\INSTexceptions}	{}
}

\newcommand{\INSTunusedField}	{\bf 0}

\newcommand{\INSTshowExceptions}	{
	\ifthenelse{\equal{}{\INSTexceptions}}{}{
		\\ {\bf Exceptions: } \INSTexceptions
	}
}

% INSTdisplay: How to display the summary line for an instruction.
%
% This uses the ThreeRegisterOp, TwoRegisterOp, and OneRegisterOp macros
% defined in ant-macros.tex.

\newcommand{\INSTdisplay}{
	\ifthenelse{\equal{ThreeReg}{\INSToptype}}{
		\ThreeRegisterOp
				{\INSTmnemonic}{\INSTdesc}{\INSTopcode}
				{\INSTfieldA}{\INSTfieldB}{\INSTfieldC}
		\INSTshowExceptions
	}
	{
	\ifthenelse{\equal{TwoReg}{\INSToptype}}{
		\TwoRegisterOp
				{\INSTmnemonic}{\INSTdesc}{\INSTopcode}
				{\INSTfieldA}{\INSTfieldB}{\INSTfieldC}
		\INSTshowExceptions
	}
	{
	\ifthenelse{\equal{OneReg}{\INSToptype}}{
		\OneRegisterOp
				{\INSTmnemonic}{\INSTdesc}{\INSTopcode}
				{\INSTfieldA}{\INSTfieldB}
		\INSTshowExceptions
	}
	{
		{\INSTmnemonic}:
		Oops!  Someone made a mistake in the spec!
	}}}
}

% The macro for printing an op spec in the architecture definition for
% Ant32:  reads in the corresponding file, and then displays the
% interesting parts.
% 
% Doesn't work properly with synonyms (yet).  Makes it easy (easier,
% anyway) to reformat the specs, which is likely in the future.

\newcommand{\INSTspec}[1]	{
	\INSTreset		% Reset all the variables to their defaults

	\input{Arch32/#1}	% Read in the instruction template.

	\INSTdisplay		% Emit the summary display line

	\INSTsemantic		% Emit the semantics, if any.

	\INSTprose		% Emit the prose, if any.
}

% end of ant-spec.tex
