\documentstyle[11pt,makeidx,psfig]{book}

\begin{document}

\chapter{Simple TLB Tutorials}

This chapter presents the details of how ANT virtual memory works and
gives some examples of how to use the TLB to manage virtual memory.

\section{TLB Background Information}

Ant-32 has support for virtual memory and contains a software-managed translation 
look-aside buffer (TLB) that is used to map virtual addresses to physical 
addresses. The TLB contains at least 16 entries and the number of entries 
must be a power of two. A TLB miss generates an exception.

The top 2 bits of each virtual address determine the segment number.
Segment 0 is the only segment accessible in user mode, while
segments 1-3 are accessible only in supervisor mode.

Ant-32 supports up to one GB of physical address space.  Physical
memory begins at address 0, but need not be contiguous.  Memory-mapped
devices are typically located at the highest addresses, but the
implementor is free to place them wherever necessary.

Each segment is one GB in size, corresponding to the size of physical
memory, but corresponds to a different way of interpreting virtual
addresses or accessing the memory.  Segments 0 and 1 are mapped
through the TLB, and may be cached.  Segments 2 and 3 are not mapped
through the TLB-- the physical address for each virtual address in
these segments is formed by removing the top two bits from the virtual
address.  

The next 18 bits of each virtual address are the virtual page number.
The bottom 12 bits are the offset into the page.

\begin{figure}[ht]
\caption{TLB Entry Format}

\begin{center}
\begin{tabular}{l|p{0.7in}|p{1.5in}|p{1.2in}|}
\cline{2-4}
                & bits 31-30 & bits 29-12       & bits 11-0 \\
\cline{2-4}
Upper Word      & 0     & Physical page number  & Page Attributes \\
\cline{2-4}
Lower Word      & Segment & Virtual page number & {\em (Available for OS)} \\
\cline{2-4}
\end{tabular}
\end{center}
\end{figure}

Each TLB entry consists of two 32-bit words.  The top 20 bits of the
upper word are the top 20 bits of the physical address of the page (if
the VALID bit is not set in the lower word).  Note that since Ant-32
physical addresses are only 30 bits long, the upper two bits of the
address, when written as a 32-bit quantity, must always be zero.  The
lower 12 bits of the lower word contain the page attributes bits.  The
page attributes include {\sc VALID}, {\sc READ}, {\sc WRITE}, {\sc
EXEC}, {\sc DIRTY}, and {\sc UNCACHE} bits, as defined in figure
\ref{TLB-attr-bits}.  The remaining bits are reserved.

The top 20 bits of the lower word are the top 20 bits of the virtual
address (the segment number and the virtual page number for that
address).  The lower 12 bits of the upper word are ignored by the
address translation logic, but are available to be used by the
operating system to hold relevant information about the page.

\begin{figure}[ht]
\caption{\label{TLB-attr-bits} TLB Page Attribute Bits}

\begin{center}
\begin{tabular}{|p{1.0in}|p{0.2in}|p{0.2in}|p{0.2in}|p{0.2in}|p{0.2in}|p{0.2in}|}
\hline
        {\em Reserved} & U & D & V & R & W & X \\
\hline
\end{tabular}
\end{center}

\begin{center}
\begin{tabular}{|l|l|p{4in}|}
\hline
{\bf Name}      & {\bf Bit}     & {\bf Description} \\
\hline
\hline
{\sc EXEC}      & 0     & Instruction fetch memory access to addresses mapped
                                by this TLB entry is allowed. \\
\hline
{\sc WRITE}     & 1     & Write memory access to addresses mapped
                                by this TLB entry is allowed. \\
\hline
{\sc READ}      & 2     & Read memory access to addresses mapped
                                by this TLB entry is allowed.  \\
\hline
{\sc VALID}     & 3     & Indicates a valid TLB entry.  When this is
                        set to 0, the contents of the rest of the TLB
                        entry are irrelevant.  \\

\hline
{\sc DIRTY}     & 4     & Indicates a dirty page.  When this is set to 1,
                        it indicates that the page referenced by this TLB
                        entry has been written.  This bit is set to 1
                        automatically whenever a write occurs to the page,
                        but can be reset to 0 using the instructions that
                        modify TLB entries. \\

\hline
{\sc UNCACHE}   & 5     & An uncacheable page.  When this is set to 1,
                        the page referenced by the entry will not be
                        cached in any processor cache.  \\

\hline
\end{tabular}
\end{center}
\end{figure}

Note that the top bit of the virtual address in the TLB will always be
zero, because only segments 0 and 1 are mapped through the TLB, and
the top two bits of the physical address will always be zero because
physical addresses have only 30 bits.  If values other than zero are
assigned to these bits the result is undefined.

Translation from virtual addresses to physical addresses occurs as follows 
for any fetch, load or store:

\begin{enumerate}

        \item The virtual address is split into the segment, virtual
                page number, and page offset.

        \item If the page offset is not divisible by the size of the
                data being fetched, loaded, or stored, an alignment
                exception occurs.

                All memory accesses must be aligned according to their
                size.  In Ant-32, there are only two sizes-- bytes, and
                4-byte words.  Word addresses must be divisible by 4,
                while byte addresses are not restricted.

        \item If the segment is not 0 and the CPU is in user mode, a
                segment privilege exception occurs.

        \item If the segment is 2 or 3, then the virtual address (with
                the segment bits set to zero) is treated as the
                physical address, and the algorithm terminates.

        \item The TLB is searched for an entry corresponding to the
                segment and virtual page number, and with its {\sc
                VALID} bit set to 1.  If no such entry exists, a TLB
                miss exception occurs.

                Note that if there are two or more valid TLB entries
                corresponding to the same virtual page, exception 13
                (TLB multiple match) will occur when the entry table
                is searched.  (This exception will also occur if the
                {\tt tlbpi} instruction is used to search the TLB
                table.)

        \item If the operation is not permitted by the page, a
                TLB protection exception occurs.

                Note that a memory location can be fetched for
                execution if its TLB entry is marked as executable
                even if it is not marked as readable.

        \item Otherwise, the physical address is constructed from the
                top 20 bits of the upper word of the TLB and the lower
                12 bits of the virtual address.

        \item If the physical address does not exist (which can only
                be detected when a memory operation is performed) a
                bus error exception occurs.

\end{enumerate}

The following sections show how to use ANT-32 instructions to create
and access TLB entries.

\section{Storing TLB Entries}

The first step in creating a TLB entry is storing all the 
necessary information in registers so that it can be transformed 
into a TLB entry using the {\tt tlbse} instruction. This instruction 
has the following syntax (des and src are both registers):

\begin{verbatim}
        tlbse 	des, src 
\end{verbatim}

The result of executing this instruction is to store the values 
from R(src) and R(src+1) into the TLB at index R(des1). 
Specifically, the lower word of the TLB entry is stored using 
R(src) and the upper word from R(src+1). Des must be an even register.

The following sample code shows how to store a TLB entry that
maps virtual page 1 to physical page 1 with the attributes:
readable, writable, valid and dirty.

{\small
\begin{verbatim}
        lc      r48, 1           # Load the index for the TLB entry
        lc      r42, 0x00001000  # Denotes virtual address 1 
        lc      r43, 0x0000101e  # Denotes physical address 1, 
                                 # attributes 0x01e 
        tlbse   r48, r42         # TLB (r48) <- r42r43
\end{verbatim}}

The result of this is to map virtual page 1 to physical page 1 with 
the given attributes (Binary equivalent of 0x101e is 000000011110).   


\section{Loading and Probing TLB Entries}

Once a TLB entry exists, we are able to load it into registers using the
{\tt tlble} instruction. We may also probe the TLB to see if an entry exists 
without actually loading it into registers. This is done using {\tt tlbpi}. Here 
we give examples of how to load the entry for the virtual address mapped
in the last example and how to probe the TLB to see if an entry
exists for a certain virtual address. 

The syntax of the tlble instruction is as follows:

\begin{verbatim}
        tlble	des, src
\end{verbatim}

This loads the lower word of the TLB entry at index R(src) into 
register R(des) and the upper word into R(des+1). Thus to load the 
entry created in the previous example into r30 and r31, we would
use the following piece of code:

{\small
\begin{verbatim}
        lc       r44, 1          # Specifies index 1
        tlble    r30, r44        # Load TLB(r44) into r30r31	
\end{verbatim}}

It is possible that we might not know where the TLB entry 
for a virtual page is located or if it even exists. In this
case we can probe the table using a specific virtual address.

The syntax of {\tt tlbpi} is as follows:

\begin{verbatim}
        tlbpi	des, src
\end{verbatim}

This probes the TLB for entries for the virtual address
R(src) and places the index in R(des), if it exists and
there are no duplicate entries for this page. The following
is code fragment that illustrates how to probe the TLB for an entry 
for virtual address 0x00001.

{\small
\begin{verbatim}
        lc       r47, 1          # Denotes virtual page 0x00001
        tlbpi    r30, r47        # 
\end{verbatim}}

\end{document}
