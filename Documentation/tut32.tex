% DJE, PAE
%
% 02/20/94 - Created for MIPS/SPIM, then later ported to Ant,
% then Ant8.
%
% 11/02/01 - Retrofitted to Ant-32.
%
% $Id: tut32.tex,v 1.14 2002/04/22 16:32:22 ellard Exp $

\chapter{Ant-32 Assembly Language Programming }
\label{ant-asm-tut-intro}

This chapter is a tutorial for the basics of Ant-32 assembly
language programming and the Ant-32 environment.  This section
covers the basics of the Ant-32 assembly language, including
arithmetic operations, simple I/O, conditionals, loops, and accessing
memory.

\section{What is Assembly Language?}
\index{assembly}

Computer instructions are represented, in a computer, as sequences of
bits.  Generally, this is the lowest possible level of representation
for a program -- each instruction is equivalent to a single,
indivisible action of the CPU.  This representation is called {\em
machine language}, and it is the only form that can be ``understood''
directly by the computer.

A slightly higher-level representation (and one that
is much easier for humans to use) is called {\em assembly language}.
Assembly language is closely related to machine language,   
and there is usually a straightforward way to translate
programs written in assembly language into machine language.
(This translation is usually implemented by a program called
an {\em assembler}.)
Assembly language is usually a direct translation of the
machine language; one instruction in assembly language
corresponds to one instruction in the machine language.

Because of the close relationship between machine and assembly
languages, each different machine architecture usually has its own
unique assembly language (in fact, a particular architecture may have
several).

\section{Getting Started with Ant-32 Assembly: {\tt add.asm}}
\index{add.asm}

To get our feet wet, we'll write an assembly language program named
{\tt add.asm} that computes the sum of 1 and 2.  Although this task is
very simple, in order to accomplish it we will need to explore several
key concepts in Ant-32 assembly language programming.

\subsection{Registers}

Like many modern CPU architectures, the Ant-32 CPU can only
operate directly on data that is stored in special locations called
{\em registers}.  The Ant-32 hardware architecture has 64
general-purpose registers.  However, some of these registers are
reserved for use by the assembler, and some are reserved for other
special purposes.

In the Ant-32 software architecture, there are 56
general-purpose registers available.  These are named {\tt g0} through
{\tt g55}.  Each of these registers can hold a single 32-bit value.

One of the registers that is defined to have a special meaning is
the {\em zero register} ({\tt ze}), which always contains the constant
zero.  Any values can be assigned to {\tt ze}, but the assignment has
no effect.

While most modern computers have many megabytes of memory, it is
unusual for a computer to have more than a few dozen registers.  Since
most computer programs use much more data than can fit into these
registers, it is usually necessary to juggle the data back and forth
between memory and the registers, where it can be operated upon by the
CPU.  (The first few programs that we write will only use registers,
but in section \ref{load-store-sec} the use of memory is introduced.)

\subsection{Commenting}
\index{commenting}

Before we start to write the executable statements of our program,
it is important to write a comment that describes what the
program is supposed to do, and what algorithm will be used to
accomplish this task.  In the Ant-32 assembly language, any text
between a pound sign ({\tt \#}) and the subsequent newline is
considered to be a comment, and is ignored by the assembler.  Good
comments are absolutely essential!  Assembly language programs are
notoriously difficult to read unless they are well organized and
properly documented.

Therefore, we start by writing the following:

{\codesize
\begin{verbatim}
# Dan Ellard
# add.asm-- A program that computes the sum of 1 and 2,
#       leaving the result in register g0.
# Registers used:
# g0 - used to hold the result.

# end of add.asm
\end{verbatim}}

Even though this program doesn't actually do anything yet, at least
anyone reading our program will know what this program is {\em
supposed} to do, and perhaps who to blame if it doesn't work.

Unlike programs written in higher level languages, it is usually
appropriate to comment every line of an assembly language program,
often with seemingly redundant comments.  Uncommented code that seems
obvious when you write it will be baffling a few hours later.  While a
well-written but uncommented program in a high level language might be
relatively easy to read and understand, even the most well-written
assembly code is unreadable without appropriate comments.  Some
programmers prefer to add comments that paraphrase the steps performed
by the assembly instructions in a higher-level language.

We are not finished commenting this program, but we've done all that
we can do until we know a little more about how the program will
actually work.

\subsection{Finding the Right Instructions}

Next, we need to figure out what instructions the computer will need
to execute in order to add two numbers.  (Since the Ant-32
architecture has relatively few instructions, it won't be long before
you have memorized the names of all of the frequently-occurring
instructions, but when you are getting started you'll need to spend
some time browsing through the list of instructions, looking for ones
that you can use to do what you want.) A summary of the user-level
instructions is given in Chapter \ref{ant32-inst}, on page
\pageref{ant32-inst}.

Scanning through the list of instructions, we find the the {\tt add}
instruction, which adds two numbers together.
The {\tt add} instruction takes three operands, which must appear
in the following order:

\begin{enumerate}

\item A register that will be used to hold the result of the addition. 
	For our program, this will be {\tt g0}.

\item A register that contains the first number to be added. 
	Therefore, we're going to have to place the value 1 into a
	register before we can use it as an operand of {\tt add}. 
	Checking the list of registers used by this program (which is
	an essential part of the commenting) we select {\tt g1}, and
	make note of this in the comments.

\item A register that holds the second number to be added.  We're also
	going to have to place the value 2 into a register before we
	can use it as an operand of {\tt add}.  Checking the list of
	registers used by this program we select {\tt g2}, and make
	note of this in the comments.

\end{enumerate}

We now know how we can add the numbers, but we have to figure out how
to place 1 and 2 into the appropriate registers.  To do this, we can
use the {\tt lc} ({\em load constant}) instruction, which places a
constant into a register.  Therefore, we arrive at the following
sequence of instructions:

{\codesize
\begin{verbatim}
# Dan Ellard
# add.asm-- A program that computes the sum of 1 and 2,
#       leaving the result in register g0.
# Registers used:
# g0 - used to hold the result.
# g1 - used to hold the constant 1.
# g2 - used to hold the constant 2.

        lc      g1, 1           # g1 = 1
        lc      g2, 2           # g2 = 2
        add     g0, g1, g2      # g0 = g1 + g2.

# end of add.asm
\end{verbatim}}

It is important to note that the {\tt lc} instruction is not
always implemented by a single Ant-32 instruction.  The {\tt lc}
instruction can handle any 32-bit constant, but the Ant-32
hardware architecture only contains instructions for dealing directly
with 16-bit constants.  In the case where the constant has a magnitude
too large to fit into 16 bits, the assembler expands the {\tt lc}
instruction into two real instructions.

For the small constants in this program, we could use {\tt lcl} (a
native instruction) instead of {\tt lc}, but it's easier to simply
always use {\tt lc} and let the assembler decide how to handle it.

%%% &&& Not done here.
%%% should there be a way to disable pseudo-ops?

\subsection{Completing the Program}

These three instructions perform the calculation that we want, but
they do not really form a complete program.  We have told the
processor what we want it to do, but we have not told it to stop
after it has done it!

Ant-32 programs always begin executing at the first instruction
in the program.  There is no rule for where the program ends, however,
and if not told otherwise the Ant-32 processor will read past
the end of the program, interpreting whatever it finds as instructions
and trying to execute them.  It might seem sensible (or obvious) that
the processor should stop executing when it reaches the ``end'' of the
program (in this case, the {\tt add} instruction on the last line),
but there are some situations where we might want the program to
continue past the ``end'' of the program, or stop before it reaches
the end.  Therefore, the Ant-32 architecture contains an
instruction named {\tt halt} that {\em halts} the processor.

The {\tt halt} instruction does not take any operands.  (For more
information about {\tt halt}, consult Section \ref{halt-inst-sec} on
page \pageref{halt-inst-sec}.)

\index{add.asm (complete listing)}
% $Id

\ThreeRegisterOp{add}{Addition}{\OPCODE}{des}{src1}{src2}

\index{add@{\tt add}!for Ant-8}

8-bit integer addition (with overflow).

\begin{enumerate}

\item Let {\em temp} $~ \leftarrow$  \Reg{src1} + \Reg{src2}.

	\Reg{src1} and \Reg{src2} are treated as
	8-bit 2's complement integers.

\item Detect whether overflow/underflow has taken place:

\begin{tabular}{lrl}

\Reg{1} $~ \leftarrow$ & 1  & if \MaxIntWord $< ~$ {\em temp} \\

\Reg{1} $~ \leftarrow$ & 0  &
	if \MinIntWord $\leq$ {\em temp} $\leq ~$ \MaxIntWord \\

\Reg{1} $~ \leftarrow$ & -1 & if {\em temp} $< ~$ \MinIntWord \\

\end{tabular}

\item \Reg{des} $\leftarrow$ the lower eight bits of {\em temp}.

\end{enumerate}



\subsection{The Format of Ant-32 Assembly Programs}

As you read {\tt add.asm}, you may notice several formatting
conventions -- every instruction is indented, and each line contains at
most one instruction.  These conventions are {\em not} simply a matter
of style, but are actually part of the definition of the Ant-32
assembly language.

The first rule of Ant-32 assembly formatting is that
instructions {\em must} be indented.  Comments do not need to be
indented, but all of the code itself must be.  The second rule of
Ant-32 assembly formatting is that only one instruction can appear on
a each line.  (There are a few additional rules, but these will not be
important until section \ref{Labels-subsec}.)

Unlike many programming languages, where the use of whitespace and
formatting is largely a matter of style, in Ant-32 assembly
language some use of whitespace is required.

\subsection{Assembling and Running Ant-32 Assembly Language Programs}
\index{ad32}
\index{aa32}
\index{ant32}

At this point, we should have a complete program.  Now, it's time to
run it and see what happens.

The principal way of running an Ant-32 program is to use the
command-line tools:  the assembler {\tt aa32}, the debugger {\tt ad32}
and VM {\tt ant32}.

\subsubsection{Using the Command-line Tools}
\index{aa32}

Before the command-line tools can run on a program, the program must be
written in a file.  This file must be plain text, and by convention
Ant-32 assembly language files have a suffix of {\tt .asm}.  In
this example, we will assume that the file {\tt add.asm} contains a
copy of the {\tt add} program listed earlier.

Before we can run the program, we must {\em assemble} it.  The
assembler translates the program from the assembly language
representation to the machine language representation.  The assembler
for Ant-32 is called {\tt aa32}, so the appropriate command
would be:

\index{aa32}
\begin{verbatim}
                aa32 add.asm
\end{verbatim}

This will create a file named {\tt add.a32} that contains the
Ant-32 machine-language representation of the program in {\tt
add.asm} (and some additional information that is used by the
debugger).

Now that we have the assembled version of the program, we can test it
by loading it into the Ant-32 debugger in order to execute it. 
The name of the Ant-32 debugger is {\tt ad32}, so to run the
debugger, use the {\tt ad32} command followed by the name of the machine
language file to load.  For example, to run the program that we just
wrote and assembled:

\index{ad32}
\begin{verbatim}
                ad32 add.a32
\end{verbatim}

After starting, the debugger will display the following
prompt: {\tt >>}.
Whenever you see the {\tt >>} prompt, you know that the debugger
is waiting for you to specify a command for it to execute.

Once the program is loaded, you can use the {\tt r} (for {\em run})
command to run it:
\begin{verbatim}
                >> r
\end{verbatim}

The program runs, and then the debugger indicates that it is ready to
execute another command.  Since our program is supposed to
leave its result in register {\tt g0}, we can verify that the
program is working by asking the debugger to print out the contents of
the registers using the {\tt p} (for {\em print}) command,
to see if it contains the result we expect:

{\codesize
\begin{verbatim}

>> p 
g0 :  00000003 00000001 00000002 00000000 00000000 00000000 00000000 00000000
g8 :  00000000 00000000 00000000 00000000 00000000 00000000 00000000 00000000
g16:  00000000 00000000 00000000 00000000 00000000 00000000 00000000 00000000
g24:  00000000 00000000 00000000 00000000 00000000 00000000 00000000 00000000
g32:  00000000 00000000 00000000 00000000 00000000 00000000 00000000 00000000
g40:  00000000 00000000 00000000 00000000 00000000 00000000 00000000 00000000
g48:  00000000 00000000 00000000 00000000 00000000 00000000 00000000 00000000
ra:   00000000
sp:   00000000
fp:   00000000
\end{verbatim}}

The {\tt p} command displays the contents of all of the registers. 
The first column shows what registers are displayed on that line.  For
example, the first line lists the values in registers {\tt g0} through
{\tt g7}.  The register values are printed in hexadecimal.

To print the value of particular registers, specify the names of those
registers as part of the {\tt p} command.  For example, to print the
values of only {\tt g0}, {\tt g1}, and {\tt g2}:

{\codesize
\begin{verbatim}
>> p g0, g1, g2
g0 :  hex: 0x00000003  dec:           3  ascii: '\003'
g1 :  hex: 0x00000001  dec:           1  ascii: '\001'  
g2 :  hex: 0x00000002  dec:           2  ascii: '\002'
\end{verbatim}}

Note that the format of the display is different when the {\tt p}
command includes specific registers.  First the hexadecimal
representation of the value in the register is printed, then the
decimal representation, and finally the {\sc ASCII} representation (if
the value is in the ASCII range).  If the {\sc ASCII} value is
printable, the corresponding character is displayed.  Otherwise, the
value is shown as a 3-digit octal number (as shown in this example).

Using the {\tt p} command, we can examine the registers to make sure
that the calculation was carried out properly.  Then we can use the
{\tt q} command to exit the debugger.

{\tt ad32} includes a number of features that will make debugging your
Ant-32 assembly language programs much easier.  Type {\tt h}
(for {\em help}) at the {\tt >>} prompt for a full list of the {\tt
ad32} commands, or consult {\tt ad32\_notes.html} for more information.

\index{ant32}

Once your program is debugged, you can use the {\tt ant32} program to
execute your {\tt .a32} files.  {\tt ant32} simply runs an Ant-32
program and then exits.

\section{Branches, Jumps, and Conditional Execution: {\tt larger.asm}}
\index{jumping}
\index{larger.asm}

The next piece of code that we will write will compare two numbers
(stored in registers {\tt g1} and {\tt g2}) and put the larger of the
two in register {\tt g0}.

The basic structure of this program is similar to the one used by
{\tt add.asm}, except that we're computing the maximum rather than the
sum of two numbers.  The difference is that the behavior of this
program depends upon the values in {\tt g1} and {\tt g2}, which are
unknown when the program is written.  The program must be able to
decide whether to execute instructions to copy the number from {\tt
g1} into {\tt g0}, or copy the number from {\tt g2} into {\tt g0}. 
This is known as {\em conditional execution} -- whether or not certain
parts of program are executed depends on a condition that is not known
when the program is written.

\subsection{Comparison Instructions}

Our program requires a way to compare two integers to determine
whether the first is larger than the second.  Fortunately, the
Ant-32 instruction set contains several instructions that make
comparing integers easy:

\vspace{3mm}
\begin{center}
\begin{tabular}{|l|l|}
\hline
{\tt eq}  & Equal \\
{\tt gts} & Greater Than (signed) \\
{\tt ges} & Greater Than or Equal (signed) \\
{\tt gtu} & Greater Than (unsigned) \\
{\tt geu} & Greater Than or Equal (unsigned) \\
\hline
\end{tabular}
\end{center}
\vspace{3mm}

The result of a comparison operation is that 1 is placed in the
destination register if the condition is true, 0 otherwise.
For example,

        \begin{verbatim}
                gts     g0, g1, g2
        \end{verbatim}

will cause {\tt g0} to get the value 1 if the value in register {\tt
g1} is greater than the value in register {\tt g2} (when the values
are interpreted as signed numbers).

\subsection{Branching and Jumping}

Ant-32 contains instructions that allow the programmer to
specify that execution should {\em branch} (or {\em jump}) to a
location other than the next instruction, or continue with the next
instruction, based on the value stored in a register.  These
instructions allow conditional execution to be implemented in assembly
language (although not nearly as succinctly as the methods provided in
higher-level languages).


In Ant-32 assembler, there are several jump instructions.  The
one we will focus on for this program is {\tt jez}, which stands for
{\em jump if equal zero}.  The format of {\tt jez} is:

\begin{alltt}
                jez     {\em des}, {\em cond}, {\em addr} \end{alltt}

where {\em des}, {\em cond}, and {\em addr} are the names of
registers.  If the value in the {\em cond} register is zero, then
execution will jump to the address specified by the {\em addr}
register; otherwise, execution will continue with the next
instruction.  In either case, the address of the currently executing
instruction is stored in the {\em des} register.  (Capturing the
address of the {\tt jez} instruction in the {\em des} register makes
it possible to use {\tt jez} to implement function calls, as discussed
in Chapter \ref{advanced-programming-chap}.)

In addition to {\tt jez}, Ant-32 includes several other jump
constructs, such as {\tt jnz} ({\em jump if not equal zero}), {\tt
jezi}, {\tt jnzi}, and {\tt j} (an unconditional jump).

In addition to the jump instructions, Ant-32 provides several
branching instructions, such as {\tt bez} and {\tt bnz} ({\em branch
if equal/not equal zero}), and {\tt bezi}, {\tt bnzi}, and {\tt b}.

There is a potential for confusion between the terms ``branching'' and
``jumping''.  In their common usage as verbs to describe what happens
in a program, they are nearly synonymous.  In the actual hardware,
however, there are two distinct kinds of instructions, which implement
this notion in very different ways, and the distinction between them
in very important.  The jump instructions cause execution to transfer
to an {\em absolute} address, while the branch instructions cause the
execution to transfer to an address calculated {\em relative} to the
current address.  For example, consider the following instructions:

\begin{center}
\begin{tabular}{|l|p{5.0in}|}
\hline
	{\tt j 12}	& Continue executing at location 12
			in memory. \\
\hline

	{\tt b 12}	& Continue executing at the twelth instruction
			past the current instruction. \\
\hline
\end{tabular}
\end{center}

Which of these instructions is more appropriate in a particular
context depends on a number of factors.  It is much easier to write
relocatable code using branches, but often more intuitive to write
simple code using jumps.  Human coders usually find jumps easier to
understand, while compilers and other automatic code generators find
it easier to use the branching instructions.

One particular difficulty with using the branching instructions is
that some of the instructions in the assembly language expand to more
than one hardware instruction, and the number of instructions in the
expansion can depend on several things.  For example, in order to know
how many instructions an {\tt lc} will really require, it is necessary
to know how large the constant is.  This makes using the branch
instructions difficult unless you entirely avoid using the synthetic
instructions that can expand to more than one size -- easy for a code
generator to do, but awkward for a human.
 
\subsection{Labels}
\label{Labels-subsec}
\index{labels}

In order to use a jump instruction, we need to know the address of
the location in memory that we want to jump to.  Keeping track of
the numeric addresses in memory of the instructions that we want to
jump to is troublesome and tedious at best -- a small error can make
our program misbehave in strange ways, and if we change the program at
all by inserting or removing instructions, we will have have to
carefully recompute all of these addresses and then change all of the
instructions that use these addresses.  This is much more than most
humans can reasonably keep track of.  Luckily, the computer is very
good at keeping track of details like this, and so the Ant-32
assembler provides {\em labels}, a way to provide a human-readable
shorthand for addresses.

A {\em label} is a symbolic name for an address in memory.  In
Ant-32 assembler, a {\em label definition} is an identifier followed
by a colon.  Ant-32 identifiers use the same conventions as
Python, Java, C, C++, and many other contemporary languages:

\begin{itemize}

\item Ant-32 identifiers must begin with an underscore, an
	uppercase character (A-Z) or a lowercase character (a-z).

\item Following the first character there may be zero or more
	underscores, or uppercase, lowercase, or numeric (0-9)
	characters.  No other characters can appear in an identifier.

\item Although there is no intrinsic limit on the length of
	Ant-32 identifiers, some Ant-32 tools may reject
	identifiers longer than 100 characters.

\end{itemize}

The definition of a label must be the first item on a line, and must
begin in the ``zero column'' (immediately after the left margin). 
Label definitions {\em cannot} be indented, but all other non-comment
lines {\em must} be.

Since label definitions must begin in column zero, only one label
definition is permitted on each line of assembly language, but a
location in memory may have more than one label.  Giving the same
location in memory more than one label can be very useful.  For
example, the same location in your program may represent the end of
several nested ``if'' statements, so you may find it useful to give
this instruction several labels corresponding to each of the nested
``if'' statements.

When a label appears alone on a line, it refers to the following
memory location.  This is often good style, since it allows the use of
long, descriptive labels without disrupting the indentation of the
program.  It also leaves plenty of space on the line for the
programmer to write a comment describing what the label is used for,
which is very important since even relatively short assembly language
programs may have a large number of labels.

Because labels represent addresses, they are effectively constants. 
Therefore, we can use {\tt lc} to load the address represented by a
label into a register, in the same manner as we loaded the constants
1 and 2 into registers in the {\tt add.asm} program.

\subsection{Jumping Using Labels}
\index{jump with labels}

Using the comparison and jump instructions and labels we can do
what we want in the {\tt larger.asm} program.  Since the jump
instructions take a register containing an address as their first
argument, we need to somehow load the address represented by the label
into a register.  We do this by using the {\tt lc} ({\em load
constant}) command.  The {\tt larger.asm} program illustrates how this
is done.

\input{Tut32/larger}

Note that Ant-32 does not have an instruction to {\em
copy} or {\em move} the contents of one register to another.  We can
achieve the same result, however, by adding zero to the source
register and saving the result in the destination register.  (There
are several other instructions we could use in a similar manner to
achieve the same result, but using addition is straightforward.)

We can use the {\tt add} instruction and use the zero register ({\tt
ze}) to supply a zero.  Alternatively, we can use the {\tt addi}
instruction.  The {\tt addi} instruction (and the other arithmetic
instructions that end in ``{\tt i}'') are called {\em immediate}
instructions because one of their operands is a constant.

\subsection{Running {\tt larger.asm} Using {\tt ad32}}

Like the previous example program, we need to assemble {\tt
larger.asm}, using {\tt aa32}, to create the file {\tt larger.a32},
before we can run the program.  Once the program is assembled, we can
run it using either {\tt ant32} or {\tt ad32}.  Unfortunately, this
program isn't very interesting -- since it never loads any values into
registers {\tt g1} and {\tt g2}, the result will always be the same. 
In a real program, we would take the numbers from the user at
runtime -- but unfortunately, reading in numbers is actually a
complicated exercise by itself.  Luckily, we can use the debugger to
load values into registers, and this will allow us to test the
logic of our program.

The {\tt lc} debugger command mimics the {\tt lc} mnemonic in
the assembly language.  For example, the command

\begin{verbatim}
                lc g1, 10
\end{verbatim}

loads the number 10 into register {\tt g1}.

To test our program, we can use the {\tt lc} command to load numbers
into registers {\tt g1} and {\tt g2}, the {\tt r} command to run the
program, and then the {\tt p} command to see the result.  An entire
such debugger session is shown in Figure \ref{ad32-lc-fig}.  The user
commands are shown in a bold font.  Note that {\tt ad32} prints the
address of the next instruction to be executed and the source code for
that instruction (unless the processor is halted), before each prompt.

\begin{figure}
\hrule
\caption{\label{ad32-lc-fig}
	Using {\tt lc} to initialize registers in {\tt ad32}.
	User input is shown in bold font.}

{\small
\begin{alltt}
0x80000000:              lc      g4, $g2_larger  # put the address of g2_larger into g4
>> {\bf lc g1, 100}

0x80000000:              lc      g4, $g2_larger  # put the address of g2_larger into g4
>> {\bf lc g2, 200}

0x80000000:              lc      g4, $g2_larger  # put the address of g2_larger into g4
>> {\bf r}
PC = 0x80000024, Status = CPU Halted
HALTED at (0x80000028)

>> {\bf p g0}
g0 :  hex: 0x000000c8  dec:         200  
\end{alltt}
}
\hrule
\end{figure}

\section{Strings and {\tt cout}: {\tt hello.asm}}
\index{hello.asm}
\index{cout}
\label{hello-sec}
\label{load-store-sec}

The next program that we will write is the ``Hello World'' program,
a program that simply prints the message ``Hello World'' to the screen
and then halts.

Ant-32 includes a very simple text-based console, with
instructions to read and write single characters.  The instruction for
writing a single character is named {\tt cout} (for {\em console
output}).

Because there is no way in Ant-32 to print out more than one
character at a time, we must use a loop to print out each character of
the string, starting at the beginning and continuing until we reach
the end of the string.

The string ``{\tt Hello World}'' is not part of the instructions of
the program, but it is part of the memory used by the program.  The
assembler places all data values (not instructions) after all of the
instructions in memory.

The way that the initial contents of data memory are defined is via
the {\tt .byte} directive.  {\tt .byte} looks like an instruction that
takes as many as eight 8-bit constants, but it is not an instruction
at all.  Instead, it is a directive to the assembler to fill in the
next available locations in memory with the given values.

Data and instructions are seperated by using two assembler directives: 
{\tt .data} and {\tt .text}.  The {\tt .data} directive tells the
assembler to assemble the subsequent lines into the data area, and the
{\tt .text} directive tells the assembler to assemble the subsequent
lines into the {\em text} or instruction memory.  In the assembled
version of your program, all of the text is placed at the beginning,
and all of the data is placed immediately after the text.

Note that the assembler assumes that the program starts with
instructions, so it is not necessary for the first line of the program
to be a {\tt .text}.  (Since none of the earlier examples in this
document used any data memory at all, they didn't need either the {\tt
.text} or {\tt .data} directives, but almost all the programs we will
see from this point forward will use them.)
 
In our programs, we will use the following convention for ASCII
strings:  a {\em string} is a sequence of characters terminated by a 0
byte.  For example, the string ``hi'' would be represented by the
three characters `h', `i', and 0.  Using a 0 byte to mark the end of
the string is a convenient method, used by several contemporary
languages.

The program {\tt hello.asm} is an example of how to use labels and
treat characters in memory as strings:

\index{hello.asm (complete listing)}
\input{Tut32/hello}

The label {\tt str\_data} is the symbolic representation of the
memory location where the string begins in data memory.

\section{Character I/O: {\tt echo.asm}}
\index{echo.asm}
\label{echo-sec}
\index{character I/O}
\index{cin}

Now that we have mastered character output, we'll turn our attention
to reading and writing single characters.  The program we'll write in
this section simply echoes whatever you type to it, until EOI ({\em end
of input}) is reached.

The instruction for reading a character from the console is named {\tt
cin} (for {\em console input}).  The way that EOI is detected in
Ant-32 is that when the EOI is reached, any attempt to use {\tt cin}
to read more input will immediately fail, and a negative value will be
placed in the destination register to indicate that there was an
error.  (If the {\tt cin} succeeds, then the destination register gets
a value between 0 and 255.)

Therefore, our program will loop, continually using {\tt cin} to read
characters, and checking after each {\tt cin} to see whether or not
the EOI has been reached.

\input{Tut32/echo}

%%% end of tut32.tex

