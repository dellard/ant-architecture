% $Id: func.tex,v 1.8 2002/04/17 20:05:59 ellard Exp $

\documentclass[makeidx,psfig]{article}
\usepackage{ifthen}
\usepackage{makeidx}
\usepackage{psfig}

% % macros.tex
%
% Definitions of commands extending Latex for the CS51 Project Book.
%
% Bob Walton
% Dec 7, 1993

\setcounter{secnumdepth}{5}
\setcounter{tocdepth}{5}

% \AND			\wedge, the math AND operator
% \OR			\vee, the math OR operator
% \NOT			\sim, the math NOT operator
% \IMPLIES		\Rightarrow, the math IMPLIES operator
% \EQUIV		\equiv, the math EQUIVALENT-TO operator
%
% \contributors{<contributor>, <contributor>, ...}
%			Specifies comma separated list of contributors to
%			be acknowledged in the preface.
%
% \begin{history}{<filename>}
% <history>		Displays <history> as a quote if \makehistory
% \end{history}		has been included in the input.  Histories are
%			used to indicate author, date, and changes to
%			project documentation files.  Histories are not
%			printed in the copy of the project book received
%			by students.
%
%			The <filename> is the name of the file whose
%			history is being given.  It is printed at the
%			beginning of the history.  The extension is
%			assumed to be .tex, and is not included in
%			<filename>.
%
%			\end{history} must appear literally without
%			any spaces just like \end{verbatim}.
%			
% \EOL			Indicates a spot in a word where a line break
%			is OK.  E.g. {\tt car-\EOL stream}.
%
% \makehistory		Causes histories to be displayed.
%
% \inputfigure{<file>}{<height>}
%			Inputs a postscript figure from the indicated
%			file and makes it the indicated height.
%
%			Extra vertical space will be put before \inputfigure
%			only if a blank line proceeds it.  Extra vertical
%			space will be put after \inputfigure only if
%			a blank line follows it.
%
% \cbox{<code>} 	Displays <code> as code names or phrases
%			embedded in text.  Similar to \verb but
%			always terminates with }.
%
%			Restriction: \cbox cannot appear in any macro
%			argument.
%
% \$<code>$		Same as \cbox{<code>} except <code> is terminated
%			by a $.  Cannot appear in any macro argument.
%
% \dollar		The $ character.  Same as \$ in normal Latex,
%			(we redefine \$ above to allow \$<code>$).
%
% \cindex{<translated-code>}
%			Similar to index, and in fact equivalent to:
%
%			  \index{<translated-code>@\tt <translated-code>}
%
%			where <translated-code> is <code> with the special
%			characters translated for latex: namely
%				``\'' is denoted by ``$\backslash$''
%				``^'' is denoted by ``\^{~}''
%				``~'' is denoted by ``$\sim$''
%				``$'' is denoted by ``\dollar''
%				``#'' is denoted by ``\#''
%				``%'' is denoted by ``\%''
%				``&'' is denoted by ``\&''
%				``_'' is denoted by ``\_''
%				``{'' is denoted by ``\{''
%				``}'' is denoted by ``\}''
%
% \ckey{<code>}		Same as \cbox but puts <code> in index as per
%			\index.
%
%			Warning: because of limitations of latex, <code>
%			is communicated to the index as {\tt <code>},
%			and therefore <code> cannot contain special
%			characters.  If you need special characters,
%			use \ickey.
%
% \ickey{<code>}{<translated-code>}
%			Must be used in place of \ckey when <code>
%			contains characters that cannot be included
%			in the scope of \tt.
%
%			Same as \ckey but calls \cindex with
%			<translated-code> instead of <code>.
%
% \key{<text>}		Applies \em or \bf to <text> and also puts <text>
%			in the index.
%
% \ikey{<text>}{<itext>}
%			Applies \em or \bf to the first argument <text>,
%			and puts the second argument <itext> in the index.
%
% \begin{code}{<name>}	Displays <code>.  Similar to \begin{verbatim}
% <code>		but indents or centers <code>.
% \end{code}
%			\end{code} must appear literally without
%			any spaces just like \end{verbatim}.
%
%			<name> is ignored during text processing.  It is
%			used to extract the <code> from the file so it can
%			be tested.
%
%			Extra vertical space will be put before \begin{code}
%			only if a blank line proceeds it.  Extra vertical
%			space will be put after \end{code} only if
%			a blank line follows it.
%
% \begin{code*}{<name>}	Ditto but displays spaces as square cups, like
% <code>		verbatim*.
% \end{code*}		WARNING: this macro has NEVER WORKED.
%
% \begin{labpar}{<label>}
%			Begin labeled paragraph, which is an indented
%			paragraph with <label> in the left margin.
% \end{labpar}		End labeled paragraph.
%
% \begin{indentpar}	Begin indented paragraph ({list} with no label).
% \end{indentpar}	End indented paragraph.
%
% \spc			A temporary length command usable in conjunction
%			with \settowidth.
%
% \begin{problems}	Begin a list of problems.
% \problem		Start the next problem.
% \end{problems}	End a list of problems.
%
% \begin{moreproblems}	Continue a list of problems.
%			(Just like problems environment,
%			 but does not reset problem number counter.)
% \end{moreproblems}	End a continued list of problems.
%
% \begin{subproblems}	Begin a list of subproblems.
% \subproblem		Start the next subproblem.
% \end{subproblems}	End a list of subproblems.
%
% \begin{boxpicture}{<x-size>}{<y-size>}
% <picture commands>	Like \begin{picture} put sets the size of one
% \end{boxpicture}	unit so that 100 units is the side of a square
%			box suitable from making dotted pair diagrams.
%			<x-size> and <y-size> are the sizes of the
%			picture in these units.  (0,0) is the lower
%			left corner, (<x-size>-1,<y-size>-1) is the
%			upper right corner.
%
% \cons{<x>}{<y>}	Draws a CONS cell consisting of two side-by-side
%			boxes.  The Lower left coordinate of the left
%			box is  (<x>,<y>), while the lower left coordinate
%			of the right box is (<x>+100,<y>).
%
% \nil{<x>}{<y>}	Draws the lower left to upper right slant line
%			that indicates a box points at NIL.  <x>,<y> are
%			the lower left coordinates of the slant line.
%
% \sym{<x>}{<y>}{<name>}
%			Draws a symbol, which is just the symbol <name>,
%			a piece of text, centered in the box whose lower
%			left coordinates are <x>, <y>.
%
% \lab{<x>}{<y>}{<pos>>}{<text>}
%			Ditto, but positions the <text>:
%			    Against the left side of the box if <pos> = l.
%			    Against the right side of the box if <pos> = r.
%			    Against the top of the box if <pos> = t.
%			    Against the bottom of the box if <pos> = b.
%
%			<pos> can consist of two letters, one for
%			horizontal control and one for vertical control.
%			The default is to center.
%			
%
% \cpointer{<x1>}{<y1>}{<x2>}{<y2>}
%			Draws a pointer from a CONS cell box (either left
%			(car) or right (cdr)) to another box.  (<x1>,<y1>)
%			are the coordinates of the CONS cell box (its
%			lower leftmost point, actually), and (<y1>,<y2>)
%			are the coordinates of the target box (also
%			its lower leftmost point).  The pointer actually
%			goes from the middle of the CONS box toward the
%			middle of the target box, but stops at the
%			boundary of the target box.
%
%			Restriction: pointers must be horizontal pointing
%			right; vertical pointing down; or at 45 degree
%			angles pointing either down left of down right.
%
% \spointer{<x1>}{<y1>}{<x2>}{<y2>}
%			Ditto but the pointer, instead of starting from
%			the middle of the box, starts from the edge of the
%			box.  Used to point from a symbol to any box.
%
% \sline{<x1>}{<y1>}{<x2>}{<y2>}
%			Ditto but a line is drawn instead of a pointer
%			(i.e. there is no arrowhead).
%

\newcommand{\AND}{\wedge}
\newcommand{\OR}{\vee}
\newcommand{\NOT}{\sim}
\newcommand{\IMPLIES}{\Rightarrow}
\newcommand{\EQUIV}{\equiv}

\newcommand{\EOL}{\penalty \exhyphenpenalty}

\newcount\ATCATCODE
\ATCATCODE=\catcode`@

\catcode `@=11	% @ is now a letter

% The following are alterations of latex.tex macros.

% Variation on @ifnextchar:
\long\def\ifnexttoken#1#2#3{\let\@tempe #1\def\@tempa{#2}\def\@tempb{#3}%
	\futurelet\@tempc\@ifnch}

\def\inputfigure#1#2{\ifvmode \vspace{2ex}\else \par\fi
		    \centerline{\psfig{figure=#1,height=#2}}
		    \ifnexttoken\par{\vspace{2ex}}{}}

\begingroup \catcode `|=0 \catcode `[= 1
\catcode`]=2 \catcode `\{=12 \catcode `\}=12
\catcode`\\=12
|gdef|@xcode#1\end{code}%
	[#1|end[code]|ifnexttoken|par[|vspace[2ex]][]]
|gdef|@sxcode#1\end{code*}%
	[#1|end[code*]|ifnexttoken|par[|vspace[2ex]][]]
|long|gdef|@xhistory#1\end{history}[|end[history]]
|endgroup

\newlength{\codewidth}

\def\code #1{\ifvmode \vspace{2ex}\else \par\fi % must be before setlength
	     \setlength{\codewidth}{\linewidth}%
	     \addtolength{\codewidth}{-\parindent}%
	     \begin{minipage}[t]{\codewidth}%\small
	     \@verbatim \frenchspacing\@vobeyspaces \@xcode}
\def\endcode{\endtrivlist\if@endpe\@doendpe\fi\end{minipage}}

\@namedef{code*} #1{%
	     \ifvmode \vspace{2ex}\else \par\fi % must be before setlength
	     \setlength{\codewidth}{\linewidth}%
	     \addtolength{\codewidth}{-\parindent}%
	     \begin{minipage}[t]{\codewidth}%\small
	     \@verbatim \@xscode}
\@namedef{endcode*}{\endtrivlist\if@endpe\@doendpe\fi\end{minipage}}

\let\dollar\$
% \begingroup \catcode `|=0 \catcode `[= 1
% \catcode`]=2 \catcode `\{=12 \catcode `\}=12 \catcode `$=12
% \catcode`\\=12
% |gdef|cbox #1[|verb }#1}]
% |gdef|$[|verb $]
% |endgroup

\gdef\cindex #1{\index{#1@{\tt #1}}}
\gdef\ckey #1{\cbox{#1}\index{#1@{\tt #1}}}
\gdef\ickey #1#2{\cbox{#1}\index{#2@{\tt #2}}}

\gdef\key #1{{\em #1}\index{#1}}
\gdef\ikey #1#2{{\em #1}\index{#2}}

\def\history#1{\begingroup \@noligs \let\do\@makeother \dospecials \@xhistory}
\def\endhistory{\endgroup}
\def\makehistory{\def\history##1{\begin{quote} \small {\bf ##1}:}
		 \def\endhistory{\end{quote}}}

\def\contributors #1{\def\contributorlist{#1}}

% End of altered latex.tex macros.

\catcode `@=\ATCATCODE	% @ is now restored

\newenvironment{labpar}[1]%
	{\begin{list}{#1}{\settowidth{\leftmargin}{#1}%
			  \addtolength{\leftmargin}{\labelsep}%
			  \settowidth{\labelwidth}{#1}%
			  \setlength{\partopsep}{\parskip}%
			  \setlength{\parskip}{0in}%
			  \setlength{\topsep}{0in}%
			  \item}}%
	{\end{list}}

\newenvironment{indentpar}%
	{\begin{list}{}{}\item}%
	{\end{list}}

\newcounter{pnumber}
\newenvironment{problems}%
	{\begin{list}{\arabic{pnumber}.}{\usecounter{pnumber}}}%
	{\end{list}}
\newcommand{\problem}{\item}

\newcounter{mpnumber}
\newenvironment{moreproblems}%
	{\setcounter{mpnumber}{\value{pnumber}}%
	 \begin{list}{\arabic{pnumber}.}{\usecounter{pnumber}}%
	 \setcounter{pnumber}{\value{mpnumber}}}%
	{\end{list}}

\newcounter{spnumber}
\newenvironment{subproblems}%
	{\begin{list}{(\alph{spnumber})}{\usecounter{spnumber}}}%
	{\end{list}}
\newcommand{\subproblem}{\item}

\newlength{\spc}

\newcount\TCOUNTX
\newcount\TCOUNTY
\newcount\TCOUNTZ

\newenvironment{boxpicture}[2]%
	{\setlength{\unitlength}{0.002in}
	 \begin{picture}(#1,#2)
	}%
        {\end{picture}}

\newcommand{\cons}[2]
	{\put(#1,#2){\begin{picture}(200,100)
		     \put(0,0){\framebox(200,100){}}
		     \put(100,0){\line(0,1){100}}
		     \end{picture}
		    }
	}

\newcommand{\sym}[3]
	{\put(#1,#2){\makebox(100,100){\tt #3}}
        }

\newcommand{\lab}[4]
	{\put(#1,#2){\makebox(100,100)[#3]{\tt #4}}
        }

\newcommand{\nil}[2]
        {\put(#1,#2){\line(1,1){100}}
        }


\newcommand{\cpointer}[4]
       	{\TCOUNTX=#1
	 \advance\TCOUNTX by 50
	 \TCOUNTY=#2
	 \advance\TCOUNTY by 50

	 \ifnum#2=#4 {\TCOUNTZ=#3
		      \advance\TCOUNTZ by -\TCOUNTX
		      \put(\number\TCOUNTX,\number\TCOUNTY)
				{\vector(1,0){\TCOUNTZ}}
		     }
	\else {
		\ifnum#1<#3 {\TCOUNTZ=#3
		     	     \advance\TCOUNTZ by -\TCOUNTX
		     	     \put(\number\TCOUNTX,\number\TCOUNTY)
					{\vector(1,-1){\TCOUNTZ}}
		    	    } 
		\fi
		\ifnum#1=#3 {\TCOUNTZ=#4
			     \advance\TCOUNTZ by 100
			     \multiply\TCOUNTZ by -1
			     \advance\TCOUNTZ by \TCOUNTY
		     	     \put(\number\TCOUNTX,\number\TCOUNTY)
					{\vector(0,-1){\TCOUNTZ}}
		    	    } 
		\fi
		\ifnum#1>#3 {\TCOUNTZ=#3
			     \advance\TCOUNTZ by 100
			     \multiply\TCOUNTZ by -1
		     	     \advance\TCOUNTZ by \TCOUNTX
		     	     \put(\number\TCOUNTX,\number\TCOUNTY)
					{\vector(-1,-1){\TCOUNTZ}}
		    	    } 
		\fi
	      }

	\fi
        }

\newcommand{\spointer}[4]
       	{\TCOUNTX=#1
	 \advance\TCOUNTX by 50
	 \TCOUNTY=#2
	 \advance\TCOUNTY by 50

	 \ifnum#2=#4 {\TCOUNTZ=#3
		      \advance\TCOUNTX by 50
		      \advance\TCOUNTZ by -\TCOUNTX
		      \put(\number\TCOUNTX,\number\TCOUNTY)
				{\vector(1,0){\TCOUNTZ}}
		     }
	\else {
		\ifnum#1<#3 {\TCOUNTZ=#3
			     \advance\TCOUNTX by 50
			     \advance\TCOUNTY by -50
		     	     \advance\TCOUNTZ by -\TCOUNTX
		     	     \put(\number\TCOUNTX,\number\TCOUNTY)
					{\vector(1,-1){\TCOUNTZ}}
		    	    } 
		\fi
		\ifnum#1=#3 {\TCOUNTZ=#4
			     \advance\TCOUNTY by -50
			     \advance\TCOUNTZ by 100
			     \multiply\TCOUNTZ by -1
			     \advance\TCOUNTZ by \TCOUNTY
		     	     \put(\number\TCOUNTX,\number\TCOUNTY)
					{\vector(0,-1){\TCOUNTZ}}
		    	    } 
		\fi
		\ifnum#1>#3 {\TCOUNTZ=#3
			     \advance\TCOUNTX by -50
			     \advance\TCOUNTY by -50
			     \advance\TCOUNTZ by 100
			     \multiply\TCOUNTZ by -1
		     	     \advance\TCOUNTZ by \TCOUNTX
		     	     \put(\number\TCOUNTX,\number\TCOUNTY)
					{\vector(-1,-1){\TCOUNTZ}}
		    	    } 
		\fi
	      }

	\fi
        }

\newcommand{\sline}[4]
       	{\TCOUNTX=#1
	 \advance\TCOUNTX by 50
	 \TCOUNTY=#2
	 \advance\TCOUNTY by 50

	 \ifnum#2=#4 {\TCOUNTZ=#3
		      \advance\TCOUNTX by 50
		      \advance\TCOUNTZ by -\TCOUNTX
		      \put(\number\TCOUNTX,\number\TCOUNTY)
				{\line(1,0){\TCOUNTZ}}
		     }
	\else {
		\ifnum#1<#3 {\TCOUNTZ=#3
			     \advance\TCOUNTX by 50
			     \advance\TCOUNTY by -50
		     	     \advance\TCOUNTZ by -\TCOUNTX
		     	     \put(\number\TCOUNTX,\number\TCOUNTY)
					{\line(1,-1){\TCOUNTZ}}
		    	    } 
		\fi
		\ifnum#1=#3 {\TCOUNTZ=#4
			     \advance\TCOUNTY by -50
			     \advance\TCOUNTZ by 100
			     \multiply\TCOUNTZ by -1
			     \advance\TCOUNTZ by \TCOUNTY
		     	     \put(\number\TCOUNTX,\number\TCOUNTY)
					{\line(0,-1){\TCOUNTZ}}
		    	    } 
		\fi
		\ifnum#1>#3 {\TCOUNTZ=#3
			     \advance\TCOUNTX by -50
			     \advance\TCOUNTY by -50
			     \advance\TCOUNTZ by 100
			     \multiply\TCOUNTZ by -1
		     	     \advance\TCOUNTZ by \TCOUNTX
		     	     \put(\number\TCOUNTX,\number\TCOUNTY)
					{\line(-1,-1){\TCOUNTZ}}
		    	    } 
		\fi
	      }

	\fi
        }

%%% 02/26/94
%%% An attempt at writing my own macros...

\newcommand{\codesize}{\small}

%%% algorithm environment:
%%%     Arg 1 is some description of the algorithm.
%%%     Arg 2 is the reference name of the algorithm.
\newenvironment{algorithm}[2]{
        \begin{figure}[hbtp]
        \hrule
	\vspace{2mm}
        \begin{alghead}
        \label{#2}
        {\em #1}
        \end{alghead}
        % \hrule
}{
        \vspace{3mm}
        \hrule
        \end{figure}
}

%%% dan-code environment:
%%%     Arg 1 is the name of the program or code fragment.
%%%     Arg 2 is the reference name of the code.
%%%     Arg 3 is any additional text to add to the preface of the code.
\newenvironment{dan-code}[3]{
        \newpage
        \section{{\tt #1}}
        \label{#2}
        \index{#1 (complete listing)}
        #3
        \vspace{3mm}
        \hrule
        \vspace{3mm}
        \footnotesize
}{
        \normalsize
        \vspace{3mm}
        \hrule
}

\newcommand{\marginnote}[1]{\marginpar{{\bf {\footnotesize #1}}}}

%% The variables must be set before any of the following macros
%% can be used.  They should be assigned in the root of the document,
%% not here, however.
%%
%% UseDefaultAntOpcodes - True if the Ant uses the default opcodes
%%	(whatever they happen to be at any given instant...) instead
%%	of some variation.  Currently, the only variation is for QRR,
%%	which omits sys and adds halt and in/out.
%%
%% AllowUnwriteableDes - The current ANT faults if the program tries
%%	to write r0 or r1.  If this bool is true, then the processor
%%	does not fault.  I just silently ignores the write, leaving
%%	r0 with zero (always) and r1 with whatever it would have been
%%	had the destination register been something else...
%%
%%	As if 1/19, the default is to fault.  However, this is very
%%	likely to change in the future.

\newboolean{UseDefaultAntOpcodes}
\newboolean{AllowUnwriteableDes}

%%% end of dan.tx

% $Id: ant-macros.tex,v 1.7 2002/01/02 02:13:47 ellard Exp $

\newcommand{\ThreeRegisterOp}[6]{
	\vspace{0.1in}
	\begin{tabular}{p{0.6in}p{2.4in}|p{0.6in}|p{0.6in}|p{0.6in}|p{0.6in}|}
	\cline{3-6}
	\fbox{\tt\Large #1} & {\bf #2} &
			#3 & {\em #4} & {\em #5} & {\em #6} \\
	\cline{3-6}
	\end{tabular}
	\vspace{0.1in}
}

% TwoRegisterOp is actually exactly the same as ThreeRegisterOp right now.
% It exists only to clarify the spec (and because perhaps in the future
% we will distinguish them).

\newcommand{\TwoRegisterOp}[6]{\ThreeRegisterOp{#1}{#2}{#3}{#4}{#5}{#6}}

\newcommand{\OneRegisterOp}[5]{
	\vspace{0.1in}
	\begin{tabular}{p{0.6in}p{2.4in}|p{0.6in}|p{0.6in}|p{1.4in}|}
	\cline{3-5}
	\fbox{\tt\Large #1}	& {\bf #2} &
			#3 & {\em #4} & {\em #5} \\
	\cline{3-5}
	\end{tabular}
	\vspace{0.1in}
}

\newcommand{\MaxIntWord}{{\bf\tt MAX\_INT8}}
\newcommand{\MinIntWord}{{\bf\tt MIN\_INT8}}
\newcommand{\MaxUIntWord}{{\bf\tt MAX\_UINT8}}
\newcommand{\MinUIntWord}{{\bf\tt MIN\_UINT8}}
\newcommand{\MaxUIntHWord}{{\bf\tt MAX\_UINT4}}
\newcommand{\MinUIntHWord}{{\bf\tt MIN\_UINT4}}

\newcommand{\Reg}[1]{{\bf\tt R({\em #1})}}
\newcommand{\Mem}[1]{{\bf\tt MEMORY(#1)}}

\newcommand{\tw}[1]{\tt {#1}}



\newtheorem{alghead}{Algorithm}[section]
\newtheorem{codehead}{Program}[section]

\makeindex

\setlength{\textheight}{8.0in}
\setlength{\textwidth}{6.5in}
\setlength{\oddsidemargin}{0.0in}
\setlength{\evensidemargin}{0.0in}
\raggedbottom

\title{Functions in {\sc Ant-8}}

\begin{document}

In this document we will see how to build modular code and how
functions and methods in higher-level languages can be written in {\sc
Ant-8}.

\section{Avoiding Repeated Code}

Recall the program {\tt hello.asm} (described in Section \ref{hello}). 
Now imagine that you had a program that printed two strings.  We could
write such a program by making a copy of most of the {\tt hello.asm}
program so that there were two loops, one for printing the first
string, and the other for printing the second, but this seems a little
wasteful.  Now imagine that we wanted to print three or four different
strings-- we could make three or four copies of this code, with slight
changed to print each string, but this would soon become ridiculous.

\begin{figure}
\caption{Source code of {\tt print-2.asm}}
\hrule
\input{Tutorial/print-2}
\hrule
\end{figure}

What we would like to do instead is discover how we can write code so
that similar functionality is implemented once, instead of many times.

In this example, one way we could do this would be to make an extra
loop, outside the printing loop, that printed each string in turn. 
This approach would work for this example, but it does not work in
general because it assumes that all we are doing is printing strings. 
In the more general case, there might be a lot of other things that
are done by the program, and they might be different from loop to
loop.

What we would like is for the printing loop to be something we can use
any time we need to print a string, no matter what else the program
does.  In order to accomplish this, we need to learn some new
techniques.

\begin{figure}
\caption{Source code of {\tt print-3.asm}}
\hrule
\input{Tutorial/print-3}
\hrule
\end{figure}

\subsection{The Return Address}

The most important thing to note about this goal is how execution gets
back to where it was before it jumped to the printing code.  This is
accomplished by noting that after a {\tt jmp} instruction is executed,
register {\tt r1} contains the address of the instruction that would
have been executed had the been an ordinary instruction (not a branch
or a jump).  This means that after the program jumps to the code for
printing out the string, register {\tt r1} contains the address that
we want to branch back to after the printing is finished.  This is
called the {\em return address}.  It is important to grab the return
address out of register {\tt r1} immediately, since many instructions
change {\tt r1}.

In this code, we grab the return address and store it in register {\tt
r5}.  When we the code is finished printing the string, it branches
back to this address by using the {\tt beq} instruction.

\subsection{Saving and Restoring State}

The code in {\tt print-3} almost accomplishes our goal, but it has an
important flaw-- it changes the contents of registers {\tt r2} through
{\tt r5}.  Therefore, we can't just use this code wherever we like,
because when the function returns the values of these registers may be
changed.  Note that register {\tt r1} will also be changed by this
piece of code, but since many operations change {\tt r1} anyway, we
don't care so much about this.

One solution to this problem would be to rest of our program so that
it didn't use these registers for anything (as is true in {\tt
print-3}), but this approach quickly becomes unworkable for programs
that have more than a few very simple functions-- there simply aren't
enough registers.

A much more general solution is to {\em preserve} the contents of the
registers used by the function (in this case, registers {\tt r2}
through {\tt r4}) by storing them to memory whenever the function is
called, and then {\em restore} them by loading their values back into
these registers just before the function jumps back to the return
address.  To do this, all we need to do is set aside a small amount of
memory to store the contents of these registers in.  In this case,
the body of the {\tt print\_str} function uses registers {\tt r2} - {\tt r4},
so we need three bytes of memory to store these values.

This solution still isn't perfect, however, because we need to use one
register to load the address of {\tt print\_str\_mem} into!  Since we
use this register to compute where the registers we're saving are
stored in memory, it is overwritten before it can be saved. 
Therefore, not {\em all} registers can be preserved and restored using
this scheme.

We have the same problem with storing the return address.  This value
must be left in a register, so the program can do the {\tt beq} to
return from the function.  We could use another register for this, but
it turns out that this is unnecessary.  By being careful and storing
the return address in memory, along with the values of registers {\tt
r2} through {\tt r4}, we can get away with using just one ``scratch''
register, register {\tt r15}.

In program {\tt print-4}, the four bytes of memory after the {\tt
print\_str\_mem} are used to store the preserved values (the return
address, and the contents of registers {\tt r2} - {\tt r4}).

\begin{figure}
\caption{The source code of {\tt print-4.asm}}
\hrule
\input{Tutorial/print-4}
\hrule
\end{figure}

\section{Recursive Functions}

The method of preserving the values of the registers used by a
function in explicit memory locations, as done in {\tt print-4} has
serious drawbacks.  First, it requires that memory be set aside to
store the registers for each function, and this memory is always set
aside for this purpose, even when none of the functions are being
called.

More importantly, however, it cannot be used to implement {\em
recursive} functions.  A recursive function is a function that calls
itself (either indirectly or directly).

For this section, the recursive function we will use is the function
for computing the $n$'th Fibonacci number.  The sequence of Fibonacci 
numbers is defined as:

\begin{itemize}

	\item	Fib(0) = 1
	\item	Fib(1) = 1
	\item	Fib(n) = Fib(n-1) + Fib(n-2) if $n > 1$.

\end{itemize}

\subsection{Using Memory as a Stack}

Instead of setting aside a specific area memory for each function, we
will set aside a pool of memory and use it as temporary storage for
all of the functions.  At any given moment, we will keep track of what
part of the memory is being used, and what parts are unused.

Keeping track of what parts are used and unused would seem like a
tedious and difficult exercise, but it is not, thanks to a key
observation about the way that functions execute-- if function A calls
function B, then function B must return before function A.  Therefore,
we can simply organize our pool of memory as an array.  When we call
function A, we can set aside as much of the array as A needs, starting
at the beginning of the array.  When function B is called, we can put
its temporary storage immediately after the storage from A.  All we
really need to keep track of is how much of the array is in use at any
time.

This data structure, where temporary function records are stacked on
top of each other, is called a {\em stack}.  The common operations on
a stack are to {\em push} a value, which means to add it to the end of
the stack, and to {\em pop} a value, which means to remove it from the
end.

\subsection{General Function Linkage}

\subsubsection{Calling a Function}

Before jumping or branching to a function, ...

\subsubsection{Function Preamble}

\begin{enumerate}

\item {\bf Save the return address in register {\tt r4}.}

	The {\tt jmp} or branch instruction that invokes the function
	saves the return address in register {\tt r1}.  Many
	instructions overwrite register {\tt r1}, so we must extract
	the return address from {\tt r1} before executing any of them,
	or else the return address will be lost.  To simplify things,
	we might as well do this immediately.

	By convention, we temporarily save the return address in
	register {\tt r4}.


\item {\bf Preserve the return address.}

	Store the return address onto the stack.  Register {\tt r2}
	is used as the stack pointer.

\item {\bf Preserve the registers.}

	Store each of the registers that we want to restore later onto
	the stack.

\item {\bf Increment the stack pointer.}

	Move the stack pointer up, so that if any other functions
	are called they start with the stack in the right place.

\end{enumerate}

\subsubsection{Returning From a Function}

\begin{enumerate}

\item	{\bf Put the return value (if any) into register {\tt r3}.}

	If the function returns a value, by convention the caller will
	expect to find it in register {\tt r3}.

\item	{\bf Decrement the stack pointer.}

	Move the stack pointer back to its previous position,
	deallocating the current stack frame.

\item	{\bf Restore the return address.}

	By convention, we load the return address into register {\tt
	r4}.

\item	{\bf Restore the registers.}

	Load the contents of each of the preserved registers back into
	the registers, from the stack.  For each {\tt st1} instruction
	in the function preamble, there must be a corresponding {\tt
	ld1}.

\item	{\bf Branch to the return address.}

	Using {\tt beq}, branch back to the return address, which
	by convention was stored in {\tt r4}.

\end{enumerate}

\subsection{Optimized Function Linkage}

The general method of building stack frames described in the previous
section is always correct, but frequently it is far from optimal.  For
example, in the {\tt fib} function, we always save all of the
registers that {\em might} potentially be used by the function, even
though many of these registers are not used.  The base case of the
recursion occurs in more than half of the calls to {\tt fib}, so more
than half of the time this is wasted effort.  It would be more
efficient to treat the base case separately from the recursive case,
and only do all the work of preserving the registers when actually
necessary.

An example is shown in {\tt fib-2.asm}.  The {\tt fib} function is
very simple, and doesn't preserve many registers, but the basic idea
of handling the base case separately is illustrated.

% \begin{figure}
% \caption{The source code of {\tt fib-2.asm}}
% \hrule
% \input{Tutorial/fib-2}
% \hrule
% \end{figure}

\end{document}
