% $Id: info.tex,v 1.14 2001/07/06 03:51:40 ellard Exp $

\renewcommand{\INSTmnemonic}	{info}
\renewcommand{\INSTdesc}	{Information Request}
\renewcommand{\INSTopcode}	{0x11}
\renewcommand{\INSToptype}	{OneReg}
\renewcommand{\INSTfieldA}	{des}
\renewcommand{\INSTfieldB}	{const16}

\renewcommand{\INSTsemantic}	{

	The value placed in \Reg{des} depends on the value of {\em
	const16}.  See figure \ref{info req figure} on page
	\pageref{info req figure} for a complete listing of the valid
	values and their meanings.

\begin{figure}[ht]
\caption{Information Request: values placed in des register}
\label{info req figure}
\begin{center}
\begin{tabular}{|l|p{4.in}|}
\hline
{\em const16}	&{\em value placed in des register}\\
\hline
\hline
0	& The number of general-purpose registers

	This count does not include the kernel registers, exception
	registers, or cycle counters.  In current implementations it
	is fixed at 64, but future implementations may allow values
	from 8 to 240.  \\
\hline
1	& {\em Reserved} \\
\hline
2	& The number of TLB entries
\\
\hline
3	& The number of bits used by {\tt srand} (0..96):
	if {\tt srand} is unimplemented, then this value is zero.
\\
\hline
4	& {\em Reserved} \\
\hline
5	& Optional feature support: \\
 	& bits 0..3: number of {\tt cin} channels (0 to 15) \\
	& bits 4..7: number of {\tt cout} channels (0 to 15) \\
 	& bit 8:  1 if {\tt rand} is supported \\
 	& bit 9:  1 if {\tt srand} is supported \\
\hline
6	& Processor manufacturer ID code

	ID codes 0..255 are reserved for the Ant development team.  \\
\hline
7	& Processor spec version:
	the version of the processor specification to which
	processor conforms. \\
	& bits 24..31: major version number \\
	& bits 16..23: minor version number \\
	& bits 8..15: release number \\
	& bits 0..7: release status (0xA = $\alpha$, 0xB = $\beta$, 0xF = stable) \\
	& For example, the processor version for the implementation
	of version 3.1.0a of Ant32 is 
	{\tt 0x0301000A}. \\
\hline
8	& CPU number (always 0 on a uniprocessor) \\
\hline
9	& Processor implementation version

	The meaning of this value is defined by the processor
	manufacturer.  \\
\hline

\end{tabular}
\end{center}

Note:  for all other values of {\em const16}, 0 is placed in the
destination register.

\end{figure}

% \end{description}

}

% \renewcommand{\INSTprose}	{ }

\renewcommand{\INSTexceptions}	{}
