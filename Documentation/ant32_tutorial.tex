% 07/09/99
% $Id: ant32_tutorial.tex,v 1.7 2002/04/17 20:05:59 ellard Exp $

\documentclass[11pt,makeidx,psfig]{book}
\usepackage{ifthen}
\usepackage{makeidx}
\usepackage{alltt}

% % macros.tex
%
% Definitions of commands extending Latex for the CS51 Project Book.
%
% Bob Walton
% Dec 7, 1993

\setcounter{secnumdepth}{5}
\setcounter{tocdepth}{5}

% \AND			\wedge, the math AND operator
% \OR			\vee, the math OR operator
% \NOT			\sim, the math NOT operator
% \IMPLIES		\Rightarrow, the math IMPLIES operator
% \EQUIV		\equiv, the math EQUIVALENT-TO operator
%
% \contributors{<contributor>, <contributor>, ...}
%			Specifies comma separated list of contributors to
%			be acknowledged in the preface.
%
% \begin{history}{<filename>}
% <history>		Displays <history> as a quote if \makehistory
% \end{history}		has been included in the input.  Histories are
%			used to indicate author, date, and changes to
%			project documentation files.  Histories are not
%			printed in the copy of the project book received
%			by students.
%
%			The <filename> is the name of the file whose
%			history is being given.  It is printed at the
%			beginning of the history.  The extension is
%			assumed to be .tex, and is not included in
%			<filename>.
%
%			\end{history} must appear literally without
%			any spaces just like \end{verbatim}.
%			
% \EOL			Indicates a spot in a word where a line break
%			is OK.  E.g. {\tt car-\EOL stream}.
%
% \makehistory		Causes histories to be displayed.
%
% \inputfigure{<file>}{<height>}
%			Inputs a postscript figure from the indicated
%			file and makes it the indicated height.
%
%			Extra vertical space will be put before \inputfigure
%			only if a blank line proceeds it.  Extra vertical
%			space will be put after \inputfigure only if
%			a blank line follows it.
%
% \cbox{<code>} 	Displays <code> as code names or phrases
%			embedded in text.  Similar to \verb but
%			always terminates with }.
%
%			Restriction: \cbox cannot appear in any macro
%			argument.
%
% \$<code>$		Same as \cbox{<code>} except <code> is terminated
%			by a $.  Cannot appear in any macro argument.
%
% \dollar		The $ character.  Same as \$ in normal Latex,
%			(we redefine \$ above to allow \$<code>$).
%
% \cindex{<translated-code>}
%			Similar to index, and in fact equivalent to:
%
%			  \index{<translated-code>@\tt <translated-code>}
%
%			where <translated-code> is <code> with the special
%			characters translated for latex: namely
%				``\'' is denoted by ``$\backslash$''
%				``^'' is denoted by ``\^{~}''
%				``~'' is denoted by ``$\sim$''
%				``$'' is denoted by ``\dollar''
%				``#'' is denoted by ``\#''
%				``%'' is denoted by ``\%''
%				``&'' is denoted by ``\&''
%				``_'' is denoted by ``\_''
%				``{'' is denoted by ``\{''
%				``}'' is denoted by ``\}''
%
% \ckey{<code>}		Same as \cbox but puts <code> in index as per
%			\index.
%
%			Warning: because of limitations of latex, <code>
%			is communicated to the index as {\tt <code>},
%			and therefore <code> cannot contain special
%			characters.  If you need special characters,
%			use \ickey.
%
% \ickey{<code>}{<translated-code>}
%			Must be used in place of \ckey when <code>
%			contains characters that cannot be included
%			in the scope of \tt.
%
%			Same as \ckey but calls \cindex with
%			<translated-code> instead of <code>.
%
% \key{<text>}		Applies \em or \bf to <text> and also puts <text>
%			in the index.
%
% \ikey{<text>}{<itext>}
%			Applies \em or \bf to the first argument <text>,
%			and puts the second argument <itext> in the index.
%
% \begin{code}{<name>}	Displays <code>.  Similar to \begin{verbatim}
% <code>		but indents or centers <code>.
% \end{code}
%			\end{code} must appear literally without
%			any spaces just like \end{verbatim}.
%
%			<name> is ignored during text processing.  It is
%			used to extract the <code> from the file so it can
%			be tested.
%
%			Extra vertical space will be put before \begin{code}
%			only if a blank line proceeds it.  Extra vertical
%			space will be put after \end{code} only if
%			a blank line follows it.
%
% \begin{code*}{<name>}	Ditto but displays spaces as square cups, like
% <code>		verbatim*.
% \end{code*}		WARNING: this macro has NEVER WORKED.
%
% \begin{labpar}{<label>}
%			Begin labeled paragraph, which is an indented
%			paragraph with <label> in the left margin.
% \end{labpar}		End labeled paragraph.
%
% \begin{indentpar}	Begin indented paragraph ({list} with no label).
% \end{indentpar}	End indented paragraph.
%
% \spc			A temporary length command usable in conjunction
%			with \settowidth.
%
% \begin{problems}	Begin a list of problems.
% \problem		Start the next problem.
% \end{problems}	End a list of problems.
%
% \begin{moreproblems}	Continue a list of problems.
%			(Just like problems environment,
%			 but does not reset problem number counter.)
% \end{moreproblems}	End a continued list of problems.
%
% \begin{subproblems}	Begin a list of subproblems.
% \subproblem		Start the next subproblem.
% \end{subproblems}	End a list of subproblems.
%
% \begin{boxpicture}{<x-size>}{<y-size>}
% <picture commands>	Like \begin{picture} put sets the size of one
% \end{boxpicture}	unit so that 100 units is the side of a square
%			box suitable from making dotted pair diagrams.
%			<x-size> and <y-size> are the sizes of the
%			picture in these units.  (0,0) is the lower
%			left corner, (<x-size>-1,<y-size>-1) is the
%			upper right corner.
%
% \cons{<x>}{<y>}	Draws a CONS cell consisting of two side-by-side
%			boxes.  The Lower left coordinate of the left
%			box is  (<x>,<y>), while the lower left coordinate
%			of the right box is (<x>+100,<y>).
%
% \nil{<x>}{<y>}	Draws the lower left to upper right slant line
%			that indicates a box points at NIL.  <x>,<y> are
%			the lower left coordinates of the slant line.
%
% \sym{<x>}{<y>}{<name>}
%			Draws a symbol, which is just the symbol <name>,
%			a piece of text, centered in the box whose lower
%			left coordinates are <x>, <y>.
%
% \lab{<x>}{<y>}{<pos>>}{<text>}
%			Ditto, but positions the <text>:
%			    Against the left side of the box if <pos> = l.
%			    Against the right side of the box if <pos> = r.
%			    Against the top of the box if <pos> = t.
%			    Against the bottom of the box if <pos> = b.
%
%			<pos> can consist of two letters, one for
%			horizontal control and one for vertical control.
%			The default is to center.
%			
%
% \cpointer{<x1>}{<y1>}{<x2>}{<y2>}
%			Draws a pointer from a CONS cell box (either left
%			(car) or right (cdr)) to another box.  (<x1>,<y1>)
%			are the coordinates of the CONS cell box (its
%			lower leftmost point, actually), and (<y1>,<y2>)
%			are the coordinates of the target box (also
%			its lower leftmost point).  The pointer actually
%			goes from the middle of the CONS box toward the
%			middle of the target box, but stops at the
%			boundary of the target box.
%
%			Restriction: pointers must be horizontal pointing
%			right; vertical pointing down; or at 45 degree
%			angles pointing either down left of down right.
%
% \spointer{<x1>}{<y1>}{<x2>}{<y2>}
%			Ditto but the pointer, instead of starting from
%			the middle of the box, starts from the edge of the
%			box.  Used to point from a symbol to any box.
%
% \sline{<x1>}{<y1>}{<x2>}{<y2>}
%			Ditto but a line is drawn instead of a pointer
%			(i.e. there is no arrowhead).
%

\newcommand{\AND}{\wedge}
\newcommand{\OR}{\vee}
\newcommand{\NOT}{\sim}
\newcommand{\IMPLIES}{\Rightarrow}
\newcommand{\EQUIV}{\equiv}

\newcommand{\EOL}{\penalty \exhyphenpenalty}

\newcount\ATCATCODE
\ATCATCODE=\catcode`@

\catcode `@=11	% @ is now a letter

% The following are alterations of latex.tex macros.

% Variation on @ifnextchar:
\long\def\ifnexttoken#1#2#3{\let\@tempe #1\def\@tempa{#2}\def\@tempb{#3}%
	\futurelet\@tempc\@ifnch}

\def\inputfigure#1#2{\ifvmode \vspace{2ex}\else \par\fi
		    \centerline{\psfig{figure=#1,height=#2}}
		    \ifnexttoken\par{\vspace{2ex}}{}}

\begingroup \catcode `|=0 \catcode `[= 1
\catcode`]=2 \catcode `\{=12 \catcode `\}=12
\catcode`\\=12
|gdef|@xcode#1\end{code}%
	[#1|end[code]|ifnexttoken|par[|vspace[2ex]][]]
|gdef|@sxcode#1\end{code*}%
	[#1|end[code*]|ifnexttoken|par[|vspace[2ex]][]]
|long|gdef|@xhistory#1\end{history}[|end[history]]
|endgroup

\newlength{\codewidth}

\def\code #1{\ifvmode \vspace{2ex}\else \par\fi % must be before setlength
	     \setlength{\codewidth}{\linewidth}%
	     \addtolength{\codewidth}{-\parindent}%
	     \begin{minipage}[t]{\codewidth}%\small
	     \@verbatim \frenchspacing\@vobeyspaces \@xcode}
\def\endcode{\endtrivlist\if@endpe\@doendpe\fi\end{minipage}}

\@namedef{code*} #1{%
	     \ifvmode \vspace{2ex}\else \par\fi % must be before setlength
	     \setlength{\codewidth}{\linewidth}%
	     \addtolength{\codewidth}{-\parindent}%
	     \begin{minipage}[t]{\codewidth}%\small
	     \@verbatim \@xscode}
\@namedef{endcode*}{\endtrivlist\if@endpe\@doendpe\fi\end{minipage}}

\let\dollar\$
% \begingroup \catcode `|=0 \catcode `[= 1
% \catcode`]=2 \catcode `\{=12 \catcode `\}=12 \catcode `$=12
% \catcode`\\=12
% |gdef|cbox #1[|verb }#1}]
% |gdef|$[|verb $]
% |endgroup

\gdef\cindex #1{\index{#1@{\tt #1}}}
\gdef\ckey #1{\cbox{#1}\index{#1@{\tt #1}}}
\gdef\ickey #1#2{\cbox{#1}\index{#2@{\tt #2}}}

\gdef\key #1{{\em #1}\index{#1}}
\gdef\ikey #1#2{{\em #1}\index{#2}}

\def\history#1{\begingroup \@noligs \let\do\@makeother \dospecials \@xhistory}
\def\endhistory{\endgroup}
\def\makehistory{\def\history##1{\begin{quote} \small {\bf ##1}:}
		 \def\endhistory{\end{quote}}}

\def\contributors #1{\def\contributorlist{#1}}

% End of altered latex.tex macros.

\catcode `@=\ATCATCODE	% @ is now restored

\newenvironment{labpar}[1]%
	{\begin{list}{#1}{\settowidth{\leftmargin}{#1}%
			  \addtolength{\leftmargin}{\labelsep}%
			  \settowidth{\labelwidth}{#1}%
			  \setlength{\partopsep}{\parskip}%
			  \setlength{\parskip}{0in}%
			  \setlength{\topsep}{0in}%
			  \item}}%
	{\end{list}}

\newenvironment{indentpar}%
	{\begin{list}{}{}\item}%
	{\end{list}}

\newcounter{pnumber}
\newenvironment{problems}%
	{\begin{list}{\arabic{pnumber}.}{\usecounter{pnumber}}}%
	{\end{list}}
\newcommand{\problem}{\item}

\newcounter{mpnumber}
\newenvironment{moreproblems}%
	{\setcounter{mpnumber}{\value{pnumber}}%
	 \begin{list}{\arabic{pnumber}.}{\usecounter{pnumber}}%
	 \setcounter{pnumber}{\value{mpnumber}}}%
	{\end{list}}

\newcounter{spnumber}
\newenvironment{subproblems}%
	{\begin{list}{(\alph{spnumber})}{\usecounter{spnumber}}}%
	{\end{list}}
\newcommand{\subproblem}{\item}

\newlength{\spc}

\newcount\TCOUNTX
\newcount\TCOUNTY
\newcount\TCOUNTZ

\newenvironment{boxpicture}[2]%
	{\setlength{\unitlength}{0.002in}
	 \begin{picture}(#1,#2)
	}%
        {\end{picture}}

\newcommand{\cons}[2]
	{\put(#1,#2){\begin{picture}(200,100)
		     \put(0,0){\framebox(200,100){}}
		     \put(100,0){\line(0,1){100}}
		     \end{picture}
		    }
	}

\newcommand{\sym}[3]
	{\put(#1,#2){\makebox(100,100){\tt #3}}
        }

\newcommand{\lab}[4]
	{\put(#1,#2){\makebox(100,100)[#3]{\tt #4}}
        }

\newcommand{\nil}[2]
        {\put(#1,#2){\line(1,1){100}}
        }


\newcommand{\cpointer}[4]
       	{\TCOUNTX=#1
	 \advance\TCOUNTX by 50
	 \TCOUNTY=#2
	 \advance\TCOUNTY by 50

	 \ifnum#2=#4 {\TCOUNTZ=#3
		      \advance\TCOUNTZ by -\TCOUNTX
		      \put(\number\TCOUNTX,\number\TCOUNTY)
				{\vector(1,0){\TCOUNTZ}}
		     }
	\else {
		\ifnum#1<#3 {\TCOUNTZ=#3
		     	     \advance\TCOUNTZ by -\TCOUNTX
		     	     \put(\number\TCOUNTX,\number\TCOUNTY)
					{\vector(1,-1){\TCOUNTZ}}
		    	    } 
		\fi
		\ifnum#1=#3 {\TCOUNTZ=#4
			     \advance\TCOUNTZ by 100
			     \multiply\TCOUNTZ by -1
			     \advance\TCOUNTZ by \TCOUNTY
		     	     \put(\number\TCOUNTX,\number\TCOUNTY)
					{\vector(0,-1){\TCOUNTZ}}
		    	    } 
		\fi
		\ifnum#1>#3 {\TCOUNTZ=#3
			     \advance\TCOUNTZ by 100
			     \multiply\TCOUNTZ by -1
		     	     \advance\TCOUNTZ by \TCOUNTX
		     	     \put(\number\TCOUNTX,\number\TCOUNTY)
					{\vector(-1,-1){\TCOUNTZ}}
		    	    } 
		\fi
	      }

	\fi
        }

\newcommand{\spointer}[4]
       	{\TCOUNTX=#1
	 \advance\TCOUNTX by 50
	 \TCOUNTY=#2
	 \advance\TCOUNTY by 50

	 \ifnum#2=#4 {\TCOUNTZ=#3
		      \advance\TCOUNTX by 50
		      \advance\TCOUNTZ by -\TCOUNTX
		      \put(\number\TCOUNTX,\number\TCOUNTY)
				{\vector(1,0){\TCOUNTZ}}
		     }
	\else {
		\ifnum#1<#3 {\TCOUNTZ=#3
			     \advance\TCOUNTX by 50
			     \advance\TCOUNTY by -50
		     	     \advance\TCOUNTZ by -\TCOUNTX
		     	     \put(\number\TCOUNTX,\number\TCOUNTY)
					{\vector(1,-1){\TCOUNTZ}}
		    	    } 
		\fi
		\ifnum#1=#3 {\TCOUNTZ=#4
			     \advance\TCOUNTY by -50
			     \advance\TCOUNTZ by 100
			     \multiply\TCOUNTZ by -1
			     \advance\TCOUNTZ by \TCOUNTY
		     	     \put(\number\TCOUNTX,\number\TCOUNTY)
					{\vector(0,-1){\TCOUNTZ}}
		    	    } 
		\fi
		\ifnum#1>#3 {\TCOUNTZ=#3
			     \advance\TCOUNTX by -50
			     \advance\TCOUNTY by -50
			     \advance\TCOUNTZ by 100
			     \multiply\TCOUNTZ by -1
		     	     \advance\TCOUNTZ by \TCOUNTX
		     	     \put(\number\TCOUNTX,\number\TCOUNTY)
					{\vector(-1,-1){\TCOUNTZ}}
		    	    } 
		\fi
	      }

	\fi
        }

\newcommand{\sline}[4]
       	{\TCOUNTX=#1
	 \advance\TCOUNTX by 50
	 \TCOUNTY=#2
	 \advance\TCOUNTY by 50

	 \ifnum#2=#4 {\TCOUNTZ=#3
		      \advance\TCOUNTX by 50
		      \advance\TCOUNTZ by -\TCOUNTX
		      \put(\number\TCOUNTX,\number\TCOUNTY)
				{\line(1,0){\TCOUNTZ}}
		     }
	\else {
		\ifnum#1<#3 {\TCOUNTZ=#3
			     \advance\TCOUNTX by 50
			     \advance\TCOUNTY by -50
		     	     \advance\TCOUNTZ by -\TCOUNTX
		     	     \put(\number\TCOUNTX,\number\TCOUNTY)
					{\line(1,-1){\TCOUNTZ}}
		    	    } 
		\fi
		\ifnum#1=#3 {\TCOUNTZ=#4
			     \advance\TCOUNTY by -50
			     \advance\TCOUNTZ by 100
			     \multiply\TCOUNTZ by -1
			     \advance\TCOUNTZ by \TCOUNTY
		     	     \put(\number\TCOUNTX,\number\TCOUNTY)
					{\line(0,-1){\TCOUNTZ}}
		    	    } 
		\fi
		\ifnum#1>#3 {\TCOUNTZ=#3
			     \advance\TCOUNTX by -50
			     \advance\TCOUNTY by -50
			     \advance\TCOUNTZ by 100
			     \multiply\TCOUNTZ by -1
		     	     \advance\TCOUNTZ by \TCOUNTX
		     	     \put(\number\TCOUNTX,\number\TCOUNTY)
					{\line(-1,-1){\TCOUNTZ}}
		    	    } 
		\fi
	      }

	\fi
        }

%%% 02/26/94
%%% An attempt at writing my own macros...

\newcommand{\codesize}{\small}

%%% algorithm environment:
%%%     Arg 1 is some description of the algorithm.
%%%     Arg 2 is the reference name of the algorithm.
\newenvironment{algorithm}[2]{
        \begin{figure}[hbtp]
        \hrule
	\vspace{2mm}
        \begin{alghead}
        \label{#2}
        {\em #1}
        \end{alghead}
        % \hrule
}{
        \vspace{3mm}
        \hrule
        \end{figure}
}

%%% dan-code environment:
%%%     Arg 1 is the name of the program or code fragment.
%%%     Arg 2 is the reference name of the code.
%%%     Arg 3 is any additional text to add to the preface of the code.
\newenvironment{dan-code}[3]{
        \newpage
        \section{{\tt #1}}
        \label{#2}
        \index{#1 (complete listing)}
        #3
        \vspace{3mm}
        \hrule
        \vspace{3mm}
        \footnotesize
}{
        \normalsize
        \vspace{3mm}
        \hrule
}

\newcommand{\marginnote}[1]{\marginpar{{\bf {\footnotesize #1}}}}

%% The variables must be set before any of the following macros
%% can be used.  They should be assigned in the root of the document,
%% not here, however.
%%
%% UseDefaultAntOpcodes - True if the Ant uses the default opcodes
%%	(whatever they happen to be at any given instant...) instead
%%	of some variation.  Currently, the only variation is for QRR,
%%	which omits sys and adds halt and in/out.
%%
%% AllowUnwriteableDes - The current ANT faults if the program tries
%%	to write r0 or r1.  If this bool is true, then the processor
%%	does not fault.  I just silently ignores the write, leaving
%%	r0 with zero (always) and r1 with whatever it would have been
%%	had the destination register been something else...
%%
%%	As if 1/19, the default is to fault.  However, this is very
%%	likely to change in the future.

\newboolean{UseDefaultAntOpcodes}
\newboolean{AllowUnwriteableDes}

%%% end of dan.tx

% $Id: ant-macros.tex,v 1.7 2002/01/02 02:13:47 ellard Exp $

\newcommand{\ThreeRegisterOp}[6]{
	\vspace{0.1in}
	\begin{tabular}{p{0.6in}p{2.4in}|p{0.6in}|p{0.6in}|p{0.6in}|p{0.6in}|}
	\cline{3-6}
	\fbox{\tt\Large #1} & {\bf #2} &
			#3 & {\em #4} & {\em #5} & {\em #6} \\
	\cline{3-6}
	\end{tabular}
	\vspace{0.1in}
}

% TwoRegisterOp is actually exactly the same as ThreeRegisterOp right now.
% It exists only to clarify the spec (and because perhaps in the future
% we will distinguish them).

\newcommand{\TwoRegisterOp}[6]{\ThreeRegisterOp{#1}{#2}{#3}{#4}{#5}{#6}}

\newcommand{\OneRegisterOp}[5]{
	\vspace{0.1in}
	\begin{tabular}{p{0.6in}p{2.4in}|p{0.6in}|p{0.6in}|p{1.4in}|}
	\cline{3-5}
	\fbox{\tt\Large #1}	& {\bf #2} &
			#3 & {\em #4} & {\em #5} \\
	\cline{3-5}
	\end{tabular}
	\vspace{0.1in}
}

\newcommand{\MaxIntWord}{{\bf\tt MAX\_INT8}}
\newcommand{\MinIntWord}{{\bf\tt MIN\_INT8}}
\newcommand{\MaxUIntWord}{{\bf\tt MAX\_UINT8}}
\newcommand{\MinUIntWord}{{\bf\tt MIN\_UINT8}}
\newcommand{\MaxUIntHWord}{{\bf\tt MAX\_UINT4}}
\newcommand{\MinUIntHWord}{{\bf\tt MIN\_UINT4}}

\newcommand{\Reg}[1]{{\bf\tt R({\em #1})}}
\newcommand{\Mem}[1]{{\bf\tt MEMORY(#1)}}

\newcommand{\tw}[1]{\tt {#1}}


% $Id: ant-spec.tex,v 1.7 2002/01/02 02:53:15 ellard Exp $
%
% Macros for printing the info for each instruction in a regular manner.
%
% This is a useful hack.  Instead of just scribbling out the
% spec/description/etc for each opcode as it appears in each place in
% the document, we put a little effort into trying to make this more
% modular.  For each instruction, we define a bunch of attribute variables--
% such as its mnemonics, numeric opcode, brief description, etc.  Then these
% can be invoked at will.
%
% The shortcoming is that these attributes are all bound to global variables,
% so there can only be one value visible at any instant.
%
% So, here's how it typically works:
%
% 1. Unset all the variables, using the \INSTreset command, defined here.
%
% 2. Set all the variables (or the subset of variables that are relevant)
%	to whatever instruction you are documenting).  This is typically
%	accomplished by \input'ing one of the files in Arch32/ or Arch/.
%
% 3. Display the values in whatever canonical style you wish.  Typically
%	only a subset of the variables is shown.  One canonical style is
%	defined here: 


\newcommand{\INSTmnemonic}	{}	% asm mnemonic
\newcommand{\INSToptype}	{}	% type of op.  Must be one of:
					% ThreeReg, TwoReg, OneReg
\newcommand{\INSTopcode}	{}	% numeric value of opcode
\newcommand{\INSTfieldA}	{}	% First field of instruction
\newcommand{\INSTfieldB}	{}	% Second field of instruction
\newcommand{\INSTfieldC}	{}	% Third field of instruction
\newcommand{\INSTdesc}		{}	% brief mnemonic description
\newcommand{\INSTsemantic}	{}	% Concise semantics.
\newcommand{\INSTprose}		{}	% Prose explanation.
\newcommand{\INSTexceptions}	{}	% Exceptions the inst can generate


% Should always do an \INSTreset before setting up the values for
% a new op.  Otherwise, the previous values will linger, and if they
% are not reset can pop up in unexpected places.

\newcommand{\INSTreset}		{
	\renewcommand{\INSTmnemonic}	{}
	\renewcommand{\INSTdesc}	{}
	\renewcommand{\INSTopcode}	{}
	\renewcommand{\INSToptype}	{}
	\renewcommand{\INSTfieldA}	{}
	\renewcommand{\INSTfieldB}	{}
	\renewcommand{\INSTfieldC}	{}
	\renewcommand{\INSTsemantic}	{}
	\renewcommand{\INSTprose}	{}
	\renewcommand{\INSTexceptions}	{}
}

\newcommand{\INSTunusedField}	{\bf 0}

\newcommand{\INSTshowExceptions}	{
	\ifthenelse{\equal{}{\INSTexceptions}}{}{
		\\ {\bf Exceptions: } \INSTexceptions
	}
}

% INSTdisplay: How to display the summary line for an instruction.
%
% This uses the ThreeRegisterOp, TwoRegisterOp, and OneRegisterOp macros
% defined in ant-macros.tex.

\newcommand{\INSTdisplay}{
	\ifthenelse{\equal{ThreeReg}{\INSToptype}}{
		\ThreeRegisterOp
				{\INSTmnemonic}{\INSTdesc}{\INSTopcode}
				{\INSTfieldA}{\INSTfieldB}{\INSTfieldC}
		\INSTshowExceptions
	}
	{
	\ifthenelse{\equal{TwoReg}{\INSToptype}}{
		\TwoRegisterOp
				{\INSTmnemonic}{\INSTdesc}{\INSTopcode}
				{\INSTfieldA}{\INSTfieldB}{\INSTfieldC}
		\INSTshowExceptions
	}
	{
	\ifthenelse{\equal{OneReg}{\INSToptype}}{
		\OneRegisterOp
				{\INSTmnemonic}{\INSTdesc}{\INSTopcode}
				{\INSTfieldA}{\INSTfieldB}
		\INSTshowExceptions
	}
	{
		{\INSTmnemonic}:
		Oops!  Someone made a mistake in the spec!
	}}}
}

% The macro for printing an op spec in the architecture definition for
% Ant32:  reads in the corresponding file, and then displays the
% interesting parts.
% 
% Doesn't work properly with synonyms (yet).  Makes it easy (easier,
% anyway) to reformat the specs, which is likely in the future.

\newcommand{\INSTspec}[1]	{
	\INSTreset		% Reset all the variables to their defaults

	\input{Arch32/#1}	% Read in the instruction template.

	\INSTdisplay		% Emit the summary display line

	\INSTsemantic		% Emit the semantics, if any.

	\INSTprose		% Emit the prose, if any.
}

% end of ant-spec.tex

% $Id: ant32-ver.tex,v 1.3 2002/01/09 21:36:30 ellard Exp $
%
% Version of Ant32.

\newcommand{\antVersion}{Revision 3.1.0b}



\newtheorem{alghead}{Algorithm}[chapter]
\newtheorem{codehead}{Program}[chapter]

\makeindex

\setlength{\textheight}{8.0in}
\setlength{\textwidth}{6.5in}
\setlength{\oddsidemargin}{0.0in}
\setlength{\evensidemargin}{0.0in}
\raggedbottom

\title{\Huge\bf Ant-32 \\ Assembly Language Tutorial \\
		(for version \input{../CurrVersion}) }

\author{Daniel J. Ellard, Penelope A. Ellard}

\begin{document}

\frontmatter
\maketitle

{\bf Copyright 2001-2002 by the President and Fellows of Harvard College. }

\tableofcontents

\chapter{Preface}

This document contains a brief tutorial for Ant-32 assembly
language programming, the assembly language utilities, and a
description of the general Ant-32 instruction set architecture. 
A complete specification of the Ant-32 architecture, including the
exception handling mechanisms, the MMU, and related issues, is given
in the {\em The Ant-32 Architecture (version \antVersion)},
which is provided as part of the Ant distribution and is also
available from the Ant web site ({\tt www.ant.harvard.edu}).

The Ant-32 architecture is a 32-bit RISC architecture designed
specifically for pedagogical purposes.  It is intended to be useful to
teaching a broad variety of topics, including machine architecture,
assembly language programming, compiler code generation, operating
systems, and VLSI circuit design and implementation.

\section{Outline}

\begin{itemize}

\item Chapter \ref{ant-asm-tut-intro} gives a tutorial for
	Ant-32 assembly language programming.  After reading this
	chapter, the reader should be able to write simple
	Ant-32 assembly language programs.

\item Chapter \ref{advanced-programming-chap} gives a tutorial for
	implementing function calls and related techniques in
	Ant-32 assembly language.

\item Chapter \ref{ant32-inst} gives a summary of the Ant-32
	instruction set.

\item Appendix \ref{rom-chapter} describes the default machine
	configuration, including the basic boot sequence and utility
	functions provided in the ROM.

\item Appendix \ref{ant32-asm-chapter} documents the assembler
	directives provided by the {\tt aa32} assembler.

\end{itemize}

\mainmatter

% DJE, PAE
%
% 02/20/94 - Created for MIPS/SPIM, then later ported to Ant,
% then Ant8.
%
% 11/02/01 - Retrofitted to Ant-32.
%
% $Id: tut32.tex,v 1.14 2002/04/22 16:32:22 ellard Exp $

\chapter{Ant-32 Assembly Language Programming }
\label{ant-asm-tut-intro}

This chapter is a tutorial for the basics of Ant-32 assembly
language programming and the Ant-32 environment.  This section
covers the basics of the Ant-32 assembly language, including
arithmetic operations, simple I/O, conditionals, loops, and accessing
memory.

\section{What is Assembly Language?}
\index{assembly}

Computer instructions are represented, in a computer, as sequences of
bits.  Generally, this is the lowest possible level of representation
for a program -- each instruction is equivalent to a single,
indivisible action of the CPU.  This representation is called {\em
machine language}, and it is the only form that can be ``understood''
directly by the computer.

A slightly higher-level representation (and one that
is much easier for humans to use) is called {\em assembly language}.
Assembly language is closely related to machine language,   
and there is usually a straightforward way to translate
programs written in assembly language into machine language.
(This translation is usually implemented by a program called
an {\em assembler}.)
Assembly language is usually a direct translation of the
machine language; one instruction in assembly language
corresponds to one instruction in the machine language.

Because of the close relationship between machine and assembly
languages, each different machine architecture usually has its own
unique assembly language (in fact, a particular architecture may have
several).

\section{Getting Started with Ant-32 Assembly: {\tt add.asm}}
\index{add.asm}

To get our feet wet, we'll write an assembly language program named
{\tt add.asm} that computes the sum of 1 and 2.  Although this task is
very simple, in order to accomplish it we will need to explore several
key concepts in Ant-32 assembly language programming.

\subsection{Registers}

Like many modern CPU architectures, the Ant-32 CPU can only
operate directly on data that is stored in special locations called
{\em registers}.  The Ant-32 hardware architecture has 64
general-purpose registers.  However, some of these registers are
reserved for use by the assembler, and some are reserved for other
special purposes.

In the Ant-32 software architecture, there are 56
general-purpose registers available.  These are named {\tt g0} through
{\tt g55}.  Each of these registers can hold a single 32-bit value.

One of the registers that is defined to have a special meaning is
the {\em zero register} ({\tt ze}), which always contains the constant
zero.  Any values can be assigned to {\tt ze}, but the assignment has
no effect.

While most modern computers have many megabytes of memory, it is
unusual for a computer to have more than a few dozen registers.  Since
most computer programs use much more data than can fit into these
registers, it is usually necessary to juggle the data back and forth
between memory and the registers, where it can be operated upon by the
CPU.  (The first few programs that we write will only use registers,
but in section \ref{load-store-sec} the use of memory is introduced.)

\subsection{Commenting}
\index{commenting}

Before we start to write the executable statements of our program,
it is important to write a comment that describes what the
program is supposed to do, and what algorithm will be used to
accomplish this task.  In the Ant-32 assembly language, any text
between a pound sign ({\tt \#}) and the subsequent newline is
considered to be a comment, and is ignored by the assembler.  Good
comments are absolutely essential!  Assembly language programs are
notoriously difficult to read unless they are well organized and
properly documented.

Therefore, we start by writing the following:

{\codesize
\begin{verbatim}
# Dan Ellard
# add.asm-- A program that computes the sum of 1 and 2,
#       leaving the result in register g0.
# Registers used:
# g0 - used to hold the result.

# end of add.asm
\end{verbatim}}

Even though this program doesn't actually do anything yet, at least
anyone reading our program will know what this program is {\em
supposed} to do, and perhaps who to blame if it doesn't work.

Unlike programs written in higher level languages, it is usually
appropriate to comment every line of an assembly language program,
often with seemingly redundant comments.  Uncommented code that seems
obvious when you write it will be baffling a few hours later.  While a
well-written but uncommented program in a high level language might be
relatively easy to read and understand, even the most well-written
assembly code is unreadable without appropriate comments.  Some
programmers prefer to add comments that paraphrase the steps performed
by the assembly instructions in a higher-level language.

We are not finished commenting this program, but we've done all that
we can do until we know a little more about how the program will
actually work.

\subsection{Finding the Right Instructions}

Next, we need to figure out what instructions the computer will need
to execute in order to add two numbers.  (Since the Ant-32
architecture has relatively few instructions, it won't be long before
you have memorized the names of all of the frequently-occurring
instructions, but when you are getting started you'll need to spend
some time browsing through the list of instructions, looking for ones
that you can use to do what you want.) A summary of the user-level
instructions is given in Chapter \ref{ant32-inst}, on page
\pageref{ant32-inst}.

Scanning through the list of instructions, we find the the {\tt add}
instruction, which adds two numbers together.
The {\tt add} instruction takes three operands, which must appear
in the following order:

\begin{enumerate}

\item A register that will be used to hold the result of the addition. 
	For our program, this will be {\tt g0}.

\item A register that contains the first number to be added. 
	Therefore, we're going to have to place the value 1 into a
	register before we can use it as an operand of {\tt add}. 
	Checking the list of registers used by this program (which is
	an essential part of the commenting) we select {\tt g1}, and
	make note of this in the comments.

\item A register that holds the second number to be added.  We're also
	going to have to place the value 2 into a register before we
	can use it as an operand of {\tt add}.  Checking the list of
	registers used by this program we select {\tt g2}, and make
	note of this in the comments.

\end{enumerate}

We now know how we can add the numbers, but we have to figure out how
to place 1 and 2 into the appropriate registers.  To do this, we can
use the {\tt lc} ({\em load constant}) instruction, which places a
constant into a register.  Therefore, we arrive at the following
sequence of instructions:

{\codesize
\begin{verbatim}
# Dan Ellard
# add.asm-- A program that computes the sum of 1 and 2,
#       leaving the result in register g0.
# Registers used:
# g0 - used to hold the result.
# g1 - used to hold the constant 1.
# g2 - used to hold the constant 2.

        lc      g1, 1           # g1 = 1
        lc      g2, 2           # g2 = 2
        add     g0, g1, g2      # g0 = g1 + g2.

# end of add.asm
\end{verbatim}}

It is important to note that the {\tt lc} instruction is not
always implemented by a single Ant-32 instruction.  The {\tt lc}
instruction can handle any 32-bit constant, but the Ant-32
hardware architecture only contains instructions for dealing directly
with 16-bit constants.  In the case where the constant has a magnitude
too large to fit into 16 bits, the assembler expands the {\tt lc}
instruction into two real instructions.

For the small constants in this program, we could use {\tt lcl} (a
native instruction) instead of {\tt lc}, but it's easier to simply
always use {\tt lc} and let the assembler decide how to handle it.

%%% &&& Not done here.
%%% should there be a way to disable pseudo-ops?

\subsection{Completing the Program}

These three instructions perform the calculation that we want, but
they do not really form a complete program.  We have told the
processor what we want it to do, but we have not told it to stop
after it has done it!

Ant-32 programs always begin executing at the first instruction
in the program.  There is no rule for where the program ends, however,
and if not told otherwise the Ant-32 processor will read past
the end of the program, interpreting whatever it finds as instructions
and trying to execute them.  It might seem sensible (or obvious) that
the processor should stop executing when it reaches the ``end'' of the
program (in this case, the {\tt add} instruction on the last line),
but there are some situations where we might want the program to
continue past the ``end'' of the program, or stop before it reaches
the end.  Therefore, the Ant-32 architecture contains an
instruction named {\tt halt} that {\em halts} the processor.

The {\tt halt} instruction does not take any operands.  (For more
information about {\tt halt}, consult Section \ref{halt-inst-sec} on
page \pageref{halt-inst-sec}.)

\index{add.asm (complete listing)}
% $Id

\ThreeRegisterOp{add}{Addition}{\OPCODE}{des}{src1}{src2}

\index{add@{\tt add}!for Ant-8}

8-bit integer addition (with overflow).

\begin{enumerate}

\item Let {\em temp} $~ \leftarrow$  \Reg{src1} + \Reg{src2}.

	\Reg{src1} and \Reg{src2} are treated as
	8-bit 2's complement integers.

\item Detect whether overflow/underflow has taken place:

\begin{tabular}{lrl}

\Reg{1} $~ \leftarrow$ & 1  & if \MaxIntWord $< ~$ {\em temp} \\

\Reg{1} $~ \leftarrow$ & 0  &
	if \MinIntWord $\leq$ {\em temp} $\leq ~$ \MaxIntWord \\

\Reg{1} $~ \leftarrow$ & -1 & if {\em temp} $< ~$ \MinIntWord \\

\end{tabular}

\item \Reg{des} $\leftarrow$ the lower eight bits of {\em temp}.

\end{enumerate}



\subsection{The Format of Ant-32 Assembly Programs}

As you read {\tt add.asm}, you may notice several formatting
conventions -- every instruction is indented, and each line contains at
most one instruction.  These conventions are {\em not} simply a matter
of style, but are actually part of the definition of the Ant-32
assembly language.

The first rule of Ant-32 assembly formatting is that
instructions {\em must} be indented.  Comments do not need to be
indented, but all of the code itself must be.  The second rule of
Ant-32 assembly formatting is that only one instruction can appear on
a each line.  (There are a few additional rules, but these will not be
important until section \ref{Labels-subsec}.)

Unlike many programming languages, where the use of whitespace and
formatting is largely a matter of style, in Ant-32 assembly
language some use of whitespace is required.

\subsection{Assembling and Running Ant-32 Assembly Language Programs}
\index{ad32}
\index{aa32}
\index{ant32}

At this point, we should have a complete program.  Now, it's time to
run it and see what happens.

The principal way of running an Ant-32 program is to use the
command-line tools:  the assembler {\tt aa32}, the debugger {\tt ad32}
and VM {\tt ant32}.

\subsubsection{Using the Command-line Tools}
\index{aa32}

Before the command-line tools can run on a program, the program must be
written in a file.  This file must be plain text, and by convention
Ant-32 assembly language files have a suffix of {\tt .asm}.  In
this example, we will assume that the file {\tt add.asm} contains a
copy of the {\tt add} program listed earlier.

Before we can run the program, we must {\em assemble} it.  The
assembler translates the program from the assembly language
representation to the machine language representation.  The assembler
for Ant-32 is called {\tt aa32}, so the appropriate command
would be:

\index{aa32}
\begin{verbatim}
                aa32 add.asm
\end{verbatim}

This will create a file named {\tt add.a32} that contains the
Ant-32 machine-language representation of the program in {\tt
add.asm} (and some additional information that is used by the
debugger).

Now that we have the assembled version of the program, we can test it
by loading it into the Ant-32 debugger in order to execute it. 
The name of the Ant-32 debugger is {\tt ad32}, so to run the
debugger, use the {\tt ad32} command followed by the name of the machine
language file to load.  For example, to run the program that we just
wrote and assembled:

\index{ad32}
\begin{verbatim}
                ad32 add.a32
\end{verbatim}

After starting, the debugger will display the following
prompt: {\tt >>}.
Whenever you see the {\tt >>} prompt, you know that the debugger
is waiting for you to specify a command for it to execute.

Once the program is loaded, you can use the {\tt r} (for {\em run})
command to run it:
\begin{verbatim}
                >> r
\end{verbatim}

The program runs, and then the debugger indicates that it is ready to
execute another command.  Since our program is supposed to
leave its result in register {\tt g0}, we can verify that the
program is working by asking the debugger to print out the contents of
the registers using the {\tt p} (for {\em print}) command,
to see if it contains the result we expect:

{\codesize
\begin{verbatim}

>> p 
g0 :  00000003 00000001 00000002 00000000 00000000 00000000 00000000 00000000
g8 :  00000000 00000000 00000000 00000000 00000000 00000000 00000000 00000000
g16:  00000000 00000000 00000000 00000000 00000000 00000000 00000000 00000000
g24:  00000000 00000000 00000000 00000000 00000000 00000000 00000000 00000000
g32:  00000000 00000000 00000000 00000000 00000000 00000000 00000000 00000000
g40:  00000000 00000000 00000000 00000000 00000000 00000000 00000000 00000000
g48:  00000000 00000000 00000000 00000000 00000000 00000000 00000000 00000000
ra:   00000000
sp:   00000000
fp:   00000000
\end{verbatim}}

The {\tt p} command displays the contents of all of the registers. 
The first column shows what registers are displayed on that line.  For
example, the first line lists the values in registers {\tt g0} through
{\tt g7}.  The register values are printed in hexadecimal.

To print the value of particular registers, specify the names of those
registers as part of the {\tt p} command.  For example, to print the
values of only {\tt g0}, {\tt g1}, and {\tt g2}:

{\codesize
\begin{verbatim}
>> p g0, g1, g2
g0 :  hex: 0x00000003  dec:           3  ascii: '\003'
g1 :  hex: 0x00000001  dec:           1  ascii: '\001'  
g2 :  hex: 0x00000002  dec:           2  ascii: '\002'
\end{verbatim}}

Note that the format of the display is different when the {\tt p}
command includes specific registers.  First the hexadecimal
representation of the value in the register is printed, then the
decimal representation, and finally the {\sc ASCII} representation (if
the value is in the ASCII range).  If the {\sc ASCII} value is
printable, the corresponding character is displayed.  Otherwise, the
value is shown as a 3-digit octal number (as shown in this example).

Using the {\tt p} command, we can examine the registers to make sure
that the calculation was carried out properly.  Then we can use the
{\tt q} command to exit the debugger.

{\tt ad32} includes a number of features that will make debugging your
Ant-32 assembly language programs much easier.  Type {\tt h}
(for {\em help}) at the {\tt >>} prompt for a full list of the {\tt
ad32} commands, or consult {\tt ad32\_notes.html} for more information.

\index{ant32}

Once your program is debugged, you can use the {\tt ant32} program to
execute your {\tt .a32} files.  {\tt ant32} simply runs an Ant-32
program and then exits.

\section{Branches, Jumps, and Conditional Execution: {\tt larger.asm}}
\index{jumping}
\index{larger.asm}

The next piece of code that we will write will compare two numbers
(stored in registers {\tt g1} and {\tt g2}) and put the larger of the
two in register {\tt g0}.

The basic structure of this program is similar to the one used by
{\tt add.asm}, except that we're computing the maximum rather than the
sum of two numbers.  The difference is that the behavior of this
program depends upon the values in {\tt g1} and {\tt g2}, which are
unknown when the program is written.  The program must be able to
decide whether to execute instructions to copy the number from {\tt
g1} into {\tt g0}, or copy the number from {\tt g2} into {\tt g0}. 
This is known as {\em conditional execution} -- whether or not certain
parts of program are executed depends on a condition that is not known
when the program is written.

\subsection{Comparison Instructions}

Our program requires a way to compare two integers to determine
whether the first is larger than the second.  Fortunately, the
Ant-32 instruction set contains several instructions that make
comparing integers easy:

\vspace{3mm}
\begin{center}
\begin{tabular}{|l|l|}
\hline
{\tt eq}  & Equal \\
{\tt gts} & Greater Than (signed) \\
{\tt ges} & Greater Than or Equal (signed) \\
{\tt gtu} & Greater Than (unsigned) \\
{\tt geu} & Greater Than or Equal (unsigned) \\
\hline
\end{tabular}
\end{center}
\vspace{3mm}

The result of a comparison operation is that 1 is placed in the
destination register if the condition is true, 0 otherwise.
For example,

        \begin{verbatim}
                gts     g0, g1, g2
        \end{verbatim}

will cause {\tt g0} to get the value 1 if the value in register {\tt
g1} is greater than the value in register {\tt g2} (when the values
are interpreted as signed numbers).

\subsection{Branching and Jumping}

Ant-32 contains instructions that allow the programmer to
specify that execution should {\em branch} (or {\em jump}) to a
location other than the next instruction, or continue with the next
instruction, based on the value stored in a register.  These
instructions allow conditional execution to be implemented in assembly
language (although not nearly as succinctly as the methods provided in
higher-level languages).


In Ant-32 assembler, there are several jump instructions.  The
one we will focus on for this program is {\tt jez}, which stands for
{\em jump if equal zero}.  The format of {\tt jez} is:

\begin{alltt}
                jez     {\em des}, {\em cond}, {\em addr} \end{alltt}

where {\em des}, {\em cond}, and {\em addr} are the names of
registers.  If the value in the {\em cond} register is zero, then
execution will jump to the address specified by the {\em addr}
register; otherwise, execution will continue with the next
instruction.  In either case, the address of the currently executing
instruction is stored in the {\em des} register.  (Capturing the
address of the {\tt jez} instruction in the {\em des} register makes
it possible to use {\tt jez} to implement function calls, as discussed
in Chapter \ref{advanced-programming-chap}.)

In addition to {\tt jez}, Ant-32 includes several other jump
constructs, such as {\tt jnz} ({\em jump if not equal zero}), {\tt
jezi}, {\tt jnzi}, and {\tt j} (an unconditional jump).

In addition to the jump instructions, Ant-32 provides several
branching instructions, such as {\tt bez} and {\tt bnz} ({\em branch
if equal/not equal zero}), and {\tt bezi}, {\tt bnzi}, and {\tt b}.

There is a potential for confusion between the terms ``branching'' and
``jumping''.  In their common usage as verbs to describe what happens
in a program, they are nearly synonymous.  In the actual hardware,
however, there are two distinct kinds of instructions, which implement
this notion in very different ways, and the distinction between them
in very important.  The jump instructions cause execution to transfer
to an {\em absolute} address, while the branch instructions cause the
execution to transfer to an address calculated {\em relative} to the
current address.  For example, consider the following instructions:

\begin{center}
\begin{tabular}{|l|p{5.0in}|}
\hline
	{\tt j 12}	& Continue executing at location 12
			in memory. \\
\hline

	{\tt b 12}	& Continue executing at the twelth instruction
			past the current instruction. \\
\hline
\end{tabular}
\end{center}

Which of these instructions is more appropriate in a particular
context depends on a number of factors.  It is much easier to write
relocatable code using branches, but often more intuitive to write
simple code using jumps.  Human coders usually find jumps easier to
understand, while compilers and other automatic code generators find
it easier to use the branching instructions.

One particular difficulty with using the branching instructions is
that some of the instructions in the assembly language expand to more
than one hardware instruction, and the number of instructions in the
expansion can depend on several things.  For example, in order to know
how many instructions an {\tt lc} will really require, it is necessary
to know how large the constant is.  This makes using the branch
instructions difficult unless you entirely avoid using the synthetic
instructions that can expand to more than one size -- easy for a code
generator to do, but awkward for a human.
 
\subsection{Labels}
\label{Labels-subsec}
\index{labels}

In order to use a jump instruction, we need to know the address of
the location in memory that we want to jump to.  Keeping track of
the numeric addresses in memory of the instructions that we want to
jump to is troublesome and tedious at best -- a small error can make
our program misbehave in strange ways, and if we change the program at
all by inserting or removing instructions, we will have have to
carefully recompute all of these addresses and then change all of the
instructions that use these addresses.  This is much more than most
humans can reasonably keep track of.  Luckily, the computer is very
good at keeping track of details like this, and so the Ant-32
assembler provides {\em labels}, a way to provide a human-readable
shorthand for addresses.

A {\em label} is a symbolic name for an address in memory.  In
Ant-32 assembler, a {\em label definition} is an identifier followed
by a colon.  Ant-32 identifiers use the same conventions as
Python, Java, C, C++, and many other contemporary languages:

\begin{itemize}

\item Ant-32 identifiers must begin with an underscore, an
	uppercase character (A-Z) or a lowercase character (a-z).

\item Following the first character there may be zero or more
	underscores, or uppercase, lowercase, or numeric (0-9)
	characters.  No other characters can appear in an identifier.

\item Although there is no intrinsic limit on the length of
	Ant-32 identifiers, some Ant-32 tools may reject
	identifiers longer than 100 characters.

\end{itemize}

The definition of a label must be the first item on a line, and must
begin in the ``zero column'' (immediately after the left margin). 
Label definitions {\em cannot} be indented, but all other non-comment
lines {\em must} be.

Since label definitions must begin in column zero, only one label
definition is permitted on each line of assembly language, but a
location in memory may have more than one label.  Giving the same
location in memory more than one label can be very useful.  For
example, the same location in your program may represent the end of
several nested ``if'' statements, so you may find it useful to give
this instruction several labels corresponding to each of the nested
``if'' statements.

When a label appears alone on a line, it refers to the following
memory location.  This is often good style, since it allows the use of
long, descriptive labels without disrupting the indentation of the
program.  It also leaves plenty of space on the line for the
programmer to write a comment describing what the label is used for,
which is very important since even relatively short assembly language
programs may have a large number of labels.

Because labels represent addresses, they are effectively constants. 
Therefore, we can use {\tt lc} to load the address represented by a
label into a register, in the same manner as we loaded the constants
1 and 2 into registers in the {\tt add.asm} program.

\subsection{Jumping Using Labels}
\index{jump with labels}

Using the comparison and jump instructions and labels we can do
what we want in the {\tt larger.asm} program.  Since the jump
instructions take a register containing an address as their first
argument, we need to somehow load the address represented by the label
into a register.  We do this by using the {\tt lc} ({\em load
constant}) command.  The {\tt larger.asm} program illustrates how this
is done.

\input{Tut32/larger}

Note that Ant-32 does not have an instruction to {\em
copy} or {\em move} the contents of one register to another.  We can
achieve the same result, however, by adding zero to the source
register and saving the result in the destination register.  (There
are several other instructions we could use in a similar manner to
achieve the same result, but using addition is straightforward.)

We can use the {\tt add} instruction and use the zero register ({\tt
ze}) to supply a zero.  Alternatively, we can use the {\tt addi}
instruction.  The {\tt addi} instruction (and the other arithmetic
instructions that end in ``{\tt i}'') are called {\em immediate}
instructions because one of their operands is a constant.

\subsection{Running {\tt larger.asm} Using {\tt ad32}}

Like the previous example program, we need to assemble {\tt
larger.asm}, using {\tt aa32}, to create the file {\tt larger.a32},
before we can run the program.  Once the program is assembled, we can
run it using either {\tt ant32} or {\tt ad32}.  Unfortunately, this
program isn't very interesting -- since it never loads any values into
registers {\tt g1} and {\tt g2}, the result will always be the same. 
In a real program, we would take the numbers from the user at
runtime -- but unfortunately, reading in numbers is actually a
complicated exercise by itself.  Luckily, we can use the debugger to
load values into registers, and this will allow us to test the
logic of our program.

The {\tt lc} debugger command mimics the {\tt lc} mnemonic in
the assembly language.  For example, the command

\begin{verbatim}
                lc g1, 10
\end{verbatim}

loads the number 10 into register {\tt g1}.

To test our program, we can use the {\tt lc} command to load numbers
into registers {\tt g1} and {\tt g2}, the {\tt r} command to run the
program, and then the {\tt p} command to see the result.  An entire
such debugger session is shown in Figure \ref{ad32-lc-fig}.  The user
commands are shown in a bold font.  Note that {\tt ad32} prints the
address of the next instruction to be executed and the source code for
that instruction (unless the processor is halted), before each prompt.

\begin{figure}
\hrule
\caption{\label{ad32-lc-fig}
	Using {\tt lc} to initialize registers in {\tt ad32}.
	User input is shown in bold font.}

{\small
\begin{alltt}
0x80000000:              lc      g4, $g2_larger  # put the address of g2_larger into g4
>> {\bf lc g1, 100}

0x80000000:              lc      g4, $g2_larger  # put the address of g2_larger into g4
>> {\bf lc g2, 200}

0x80000000:              lc      g4, $g2_larger  # put the address of g2_larger into g4
>> {\bf r}
PC = 0x80000024, Status = CPU Halted
HALTED at (0x80000028)

>> {\bf p g0}
g0 :  hex: 0x000000c8  dec:         200  
\end{alltt}
}
\hrule
\end{figure}

\section{Strings and {\tt cout}: {\tt hello.asm}}
\index{hello.asm}
\index{cout}
\label{hello-sec}
\label{load-store-sec}

The next program that we will write is the ``Hello World'' program,
a program that simply prints the message ``Hello World'' to the screen
and then halts.

Ant-32 includes a very simple text-based console, with
instructions to read and write single characters.  The instruction for
writing a single character is named {\tt cout} (for {\em console
output}).

Because there is no way in Ant-32 to print out more than one
character at a time, we must use a loop to print out each character of
the string, starting at the beginning and continuing until we reach
the end of the string.

The string ``{\tt Hello World}'' is not part of the instructions of
the program, but it is part of the memory used by the program.  The
assembler places all data values (not instructions) after all of the
instructions in memory.

The way that the initial contents of data memory are defined is via
the {\tt .byte} directive.  {\tt .byte} looks like an instruction that
takes as many as eight 8-bit constants, but it is not an instruction
at all.  Instead, it is a directive to the assembler to fill in the
next available locations in memory with the given values.

Data and instructions are seperated by using two assembler directives: 
{\tt .data} and {\tt .text}.  The {\tt .data} directive tells the
assembler to assemble the subsequent lines into the data area, and the
{\tt .text} directive tells the assembler to assemble the subsequent
lines into the {\em text} or instruction memory.  In the assembled
version of your program, all of the text is placed at the beginning,
and all of the data is placed immediately after the text.

Note that the assembler assumes that the program starts with
instructions, so it is not necessary for the first line of the program
to be a {\tt .text}.  (Since none of the earlier examples in this
document used any data memory at all, they didn't need either the {\tt
.text} or {\tt .data} directives, but almost all the programs we will
see from this point forward will use them.)
 
In our programs, we will use the following convention for ASCII
strings:  a {\em string} is a sequence of characters terminated by a 0
byte.  For example, the string ``hi'' would be represented by the
three characters `h', `i', and 0.  Using a 0 byte to mark the end of
the string is a convenient method, used by several contemporary
languages.

The program {\tt hello.asm} is an example of how to use labels and
treat characters in memory as strings:

\index{hello.asm (complete listing)}
\input{Tut32/hello}

The label {\tt str\_data} is the symbolic representation of the
memory location where the string begins in data memory.

\section{Character I/O: {\tt echo.asm}}
\index{echo.asm}
\label{echo-sec}
\index{character I/O}
\index{cin}

Now that we have mastered character output, we'll turn our attention
to reading and writing single characters.  The program we'll write in
this section simply echoes whatever you type to it, until EOI ({\em end
of input}) is reached.

The instruction for reading a character from the console is named {\tt
cin} (for {\em console input}).  The way that EOI is detected in
Ant-32 is that when the EOI is reached, any attempt to use {\tt cin}
to read more input will immediately fail, and a negative value will be
placed in the destination register to indicate that there was an
error.  (If the {\tt cin} succeeds, then the destination register gets
a value between 0 and 255.)

Therefore, our program will loop, continually using {\tt cin} to read
characters, and checking after each {\tt cin} to see whether or not
the EOI has been reached.

\input{Tut32/echo}

%%% end of tut32.tex



\chapter{Advanced Ant-32 Programming}
\label{advanced-programming-chap}
% $Id: reg32.tex,v 1.18 2002/04/22 16:32:21 ellard Exp $
% $Date: 2002/04/22 16:32:21 $

\section{Introduction}

Any of the general registers in the Ant-32 architecture can, in
general, be used in whatever way the programmer wishes.  The
architecture imposes no restrictions or limitations (apart from the
restriction that the zero register always contains the constant 0, and
that for the operations that take a register pair as an operand, the
register pair must begin with an even-numbered register).

Most software architectures, however, include some conventions about
the use of specific registers.  These conventions are principally
focussed on supporting features of high-level languages, such as
functions, recursion, and separate compilation.

In order to facilitate the implementation of higher-level software
architectures using Ant-32, the Ant-32 tools support two register
names and conventions.

The first is a very simple model, useful for introductory programming
courses and demonstrating how function calls and recursion can be
implemented.  This convention is the focus of the rest of this document.

The second is a more advanced model, which refines the simple model in
a manner that allows for more efficient code.  It is described only
briefly in this document.

\section{Simple Register Use Conventions}

The simple register use conventions implement a straight-forward stack
architecture.  The conventions are outlined in Figure
\ref{simple-conventions-fig}, and described in more detail below.

\begin{figure}
\caption{\label{simple-conventions-fig} Simplified Register Use Conventions}

\begin{center}
\begin{tabular}{|l|l|p{2.0in}|}

\hline
	{\bf Mnemonic}	& {\bf Registers} & {\bf Description} \\
\hline
\index{ze - the zero register}
	{\tt ze}	& {\tt r0}		& Always zero \\
\hline
\index{ra - the return address}
	{\tt ra}	& {\tt r1}		& Return address \\
\hline
\index{sp - the stack pointer}
	{\tt sp}	& {\tt r2}		& Stack pointer \\
\hline
\index{fp - the frame pointer}
	{\tt fp}	& {\tt r3}		& Frame pointer \\
\hline
	{\tt g0-g55}	& {\tt r4 - r59}	& General-purpose registers \\
\hline
\index{u0-u3 - scratch registers}
	{\tt u0-u3}	& {\tt r60 - r63}	& Reserved registers \\
\hline

\end{tabular}
\end{center}

\end{figure}

\subsection{{\tt ze} - The zero register}

	The {\tt ze} register is simply register zero, which always
	contains the number zero.

\subsection{{\tt ra} - The Return Address}

	The {\tt ra} register is used to store the return address of
	the most recent function call.

\subsection{{\tt sp} - The Stack Pointer}

	{\tt sp} is used as the {\em stack pointer}.  The stack grows
	``downward''; a push moves the stack pointer to a numerically
	lower address, and a pop moves the stack pointer toward
	numerically greater address.

	The Ant-32 architecture does not contain native push or pop
	instructions, and these operations require more than one
	instruction to execute.  The push and pop operations, for
	example, can be coded as shown in Figure
	\ref{simple-push-pop}.

	In general, {\tt sp} points to the ``top'' of the stack
	(although this may seem somewhat confusing, since the stack
	grows downward -- so the top of the stack is located at the lowest
	address).  This convention can be relaxed in order to
	implement groups of push or pop operations (see Figure
	\ref{combined-push-pop}), as long as the stack pointer is
	never moved past any values that are still on the stack.


	\begin{figure}
	\hrule
	\caption{\label{simple-push-pop}
			Implementing {\tt push} and {\tt pop}}

        \begin{verbatim}
                # push register g0:
                subi    sp, sp, 4
                st4     g0, sp, 0

                # pop into register g1:
                ld4     g1, sp, 0
                addi    sp, sp, 4
        \end{verbatim}
	\hrule
	\end{figure}

	The Ant-32 assembler provides macro implemenations of {\tt
	push} and {\tt pop}, using this method.

	\begin{figure}
	\hrule
	\caption{\label{combined-push-pop}
			Combining Multiple Push or Pop Operations}
			\vspace{3mm}

	For consecutive pushes and pops, it can increase code
	efficiency to reduce the number of {\tt addi} and {\tt subi}
	instructions by aggregating the movement of the stack pointer,
	as shown in the following code fragment.

	\begin{verbatim}
                # push registers g0, g1, g2:
                subi    sp, 12
                st4     g0, sp, 8
                st4     g1, sp, 4
                st4     g2, sp, 0

                # pop into registers g3, g4, g5:
                ld4     g3, sp, 0
                ld4     g4, sp, 4
                ld4     g5, sp, 8
                addi    sp, 12
        \end{verbatim}
	\hrule
	\end{figure}

\subsection{{\tt fp} - The Frame Pointer}

The {\tt fp} register is used as a {\em frame pointer}.  The frame is
often used to implement activation records, or simplify the
implementation of function calls.

\subsection{{\tt g0}-{\tt g55} - General-Purpose Registers}

These registers are free to be used for any purpose.

\subsection{{\tt u0}-{\tt u3} - Reserved Registers}

These registers are reserved for use by the assembler.  They are used
as scratch space for the expansion of macros.  They should not be used
for any other purpose, and programs should never make any assumptions
about their contents.

\section{Function Calls}
\index{function calls}

This section describes how the stack pointer, frame pointer, and
return address registers can be used to implement the abstraction of
function calls.  The description is divided into four steps:

\begin{enumerate}

\item	Preparing to call the function and performing the call.

\item	Function preamble.

\item	Preparing to return from the function.

\item	Cleaning up after the function call.

\end{enumerate}

\subsection{Preparing to Call: Using {\tt call}}
\label{reg32-call-sec}
\index{call}

\begin{enumerate}

\item All of the {\tt g-}registers whose values need to be preserved
	are pushed onto the stack.  The order that they are pushed
	onto the stack is up to the caller.

	Before the function call takes place, the caller must save any
	registers that contain necessary values, because otherwise the
	function might overwrite these values.

\item The arguments to the function are pushed onto the stack, in the
	reverse order that they appear (from right to left).

	The stack only contains whole words (32-bit values).  If the
	arguments to the function are 8 or 16-bit values, then they
	are still pushed as the lower 8 or 16 bits of a 32-bit value,
	requiring four bytes of storage.  It is the responsibility of
	the called function to ignore the extra bits.

\item Jump or branch to the function (using {\tt jez}, {\tt jnz}, {\tt
	bez}, or {\tt bnz}), specifying the return address register
	{\tt ra} as the destination register.

\end{enumerate}

Note that the last step can be accomplished with the {\tt call} macro.

\subsection{Handling the Call: Using {\tt entry}}
\index{entry}
\label{reg32-entry-sec}

\begin{enumerate}

\item The current value of the {\tt fp} and {\tt ra} registers
	are pushed onto the stack.

\item The frame pointer gets a copy of the stack pointer.

\item The stack pointer is decremented by the size of the local frame. 
	The area of memory thus allocated between the stack pointer
	and the frame pointer is used for local storage -- for example,
	the local variables of the current function.

	Note that the local frame size must always be a multiple of 4,
	so that the stack pointer is always aligned properly on a
	4-byte boundary.

\end{enumerate}

These steps can be accomplished by using the {\tt entry} macro.  This
macro takes a single constant argument, which is the size of the stack
frame to create.

After this preamble is finished, the stack contains the information
about the function call in the order shown in Figure
\ref{simple-stack-call}.

\begin{figure}
\caption{\label{simple-stack-call} Stack at start of call.}

\begin{center}

\begin{tabular}{|l|l|l|}
\hline
{\bf Address}	& {\bf Contents} & {\bf Description} \\
\hline
$\vdots$	& {\tt g0} $\cdots$ {\tt g55}	&
		Saved copies of {\tt g-}registers. \\
$\vdots$	& $\vdots$			& \\
\hline
${\tt fp} + 8 + (N ~ \times 4 )$ 	& $arg_{N}$	& \\
$\vdots$	& $\vdots$			& 
		Arguments to the function. \\
${\tt fp} + 8$	& $arg_{0}$		& \\

\hline
${\tt fp} + 4$	& {\tt fp}			&
		The saved value of the {\tt fp}. \\
\hline
${\tt fp} + 0$	& {\tt ra}			&
		The saved value of the {\tt ra}. \\
\hline
${\tt fp} - 4$	& 				& \\
$\vdots$	& $\vdots$			&
		{\em local variables}		\\
${\tt fp} - (4 + (M ~ \times 4))$	&	& \\
\hline
\end{tabular}

\end{center}

\end{figure}

Note that the function can always access its arguments and local
variables via fixed offsets relative to the frame pointer, and the
stack pointer is free to move.  For example, the first argument
($arg_{0}$) is accessible at the address ${\tt fp} + 8$, while the
second argument is at address ${\tt fp} + 12$, and so forth.

During a function call, the stack pointer can be used to manage the
allocation of dynamic but function-private storage.  If the storage
requirements of the function can be computed in advance, however, it
can be just as convenient to allocate this space from the frame.

\subsection{Returning from a Call: Using {\tt return}}
\index{return}
\label{reg32-return-sec}

\begin{enumerate}

\item The return value (if any) is put into register {\tt g0}.

	Functions that return multiple values, or a single value that
	is too large to fit into a single register, use a more
	complicated method for returning their values.  This method is
	not documented here.

\item The stack pointer is reset to contain a copy of the frame
	pointer.

\item The return address is popped into {\tt ra}, and then
	the {\tt ra} register is incremented by 4.

	This increment is necessary because when the function is
	called via a jump or branch instruction, {\tt ra} gets the
	address of the instruction that performed the call.  The
	address we want to return to is the address of the instruction
	after the call.

\item The frame pointer is popped into {\tt fp}.

	At this point, the stack pointer is in the same position as it
	was before the function was called.

\item Use the {\tt jez} instruction to jump to the {\tt ra}.

\end{enumerate}

For a function that returns a single value, the {\tt return} macro is
provided to perform all of these steps.  The single operand to the
{\tt return} macro can be the name of the register that contains the value
to return, or the constant to return.

\subsection{Handling the Return}

When the execution resumes in the caller, the stack is exactly the
same as it was before the jump to the caller.  All that remains is to
save the results, and restore the rest of the environment to the way
it was before the call took place.  This can be done by popping the
parameters and then by popping the saved {\tt g}-registers.  Once the
stack is restored, execution can resume as normal.

\section{Examples of Functions}

\newpage
Program {\tt add-func.asm} gives a very simple example of a
function that takes two arguments and returns their sum.

\vspace{3mm}
\hrule
\index{add-func.asm}
\input{Tut32/add-func}

\newpage
Program {\tt fibonacci.asm} gives an example of a recursive
function.

\vspace{3mm}
\hrule
\index{fibonacci.asm}
\input{Tut32/fibonacci}

\newpage
\section{Advanced Register Use Conventions}
\index{function calls, optimized}

\subsection{Optimizing Saving and Restoring of Registers}

The function calling conventions described in the first part of this
chapter can result in very inefficient code.  For example, imagine
that we have a function $\alpha$ that calls function $\beta$.  Before
$\alpha$ calls $\beta$, it has to save all the registers it is using. 
If $\alpha$ uses many registers, and $\beta$ only uses a few, then it
may be that many of $\alpha$'s registers didn't need to be saved,
because their values weren't modified by $\beta$ at all.

One solution to this particular problem is to change the
responsibility for saving the registers to the called function -- in
this case, $\beta$ would be responsible for saving and restoring the
few registers that it uses.  Unfortunately, in the opposite case,
where $\alpha$ only uses a few registers and $\beta$ uses many, then
this approach results in the same kind of inefficiency as we saw
initially.

Ideally, each function would have its own set of registers available
for its exclusive use.  Unfortunately, this is impossible:  typical
programs have thousands of functions but processors only have dozens
of registers -- and even if a huge number of registers were available,
recursive functions would still be a problem.

However, there is a relatively straightforward way to solve most of
this problem, by dividing the register set into two groups -- one group
which is caller-saved (like all the registers in the earlier
convention) and a second which is callee-saved.  Ideally, functions
that call other functions will use the callee-saved registers, and
{\em leaf functions} (functions that do not call other functions) or
the base case code of recursive functions will use the caller-saved
registers.  If, in our previous example, $\beta$ is a leaf function,
then if $\alpha$ uses only callee-saved registers, and $\beta$ uses
only caller-saved registers, then no registers will need to be saved
at all.

\subsection{Optimizing Parameter Passing}

Another cause of inefficiency in the normal function call conventions
is the pushing of the parameters onto the stack, and then accessing them
via the frame pointer.   In terms of the number of instructions executed,
this convention is not terribly inefficient -- but in terms of the {\em kind}
of instructions executed, it can be very slow.  Passing the parameters on
the stack means storing to memory and then loading from memory, and on
most modern processors accessing memory is at least an order of magnitude 
slower than accessing values in registers.

Therefore, to optimize the passing of parameters, we reserve a small
number of registers to use for passing parameters.  If there are more
parameters than will fit in these registers, the remainder are passed
on the stack as before.  Studies of existing bodies of software have
shown, however, that six (or even four) argument registers are
sufficient for an overwhelming majority of common functions.

\subsection{The Advanced Conventions}

\begin{figure}
\caption{\label{advanced-conventions-fig} Advanced Register Use Conventions}

\begin{center}
\begin{tabular}{|l|l|p{2.0in}|}

\hline
	{\bf Mnemonic}	& {\bf Registers} & {\bf Description} \\
\hline
	{\tt ze}	& {\tt r0}		& Always zero \\
\hline
	{\tt ra}	& {\tt r1}		& Return address \\
\hline
	{\tt sp}	& {\tt r2}		& Stack pointer \\
\hline
	{\tt fp}	& {\tt r3}		& Frame pointer \\
\hline
	{\tt v0-1}	& {\tt r4 - r5}		& Returned values \\
\hline
	{\tt a0-5}	& {\tt r6 - r11}	& Argument registers \\
\hline
	{\tt s0-23}	& {\tt r12 - r35}	& Callee-saved \\
\hline
	{\tt t0-23}	& {\tt r36 - r59}	& Caller-saved \\
\hline
	{\tt u0-3}	& {\tt r60 - r63}	& Reserved for the assembler \\
\hline

\end{tabular}
\end{center}

\end{figure}

The conventions are similar to the previous, except that the {\tt
g}-registers have been partitioned into four different kinds of
registers:  return value registers, argument value registers, saved
registers, and temporary registers.  These registers are described in
more detail below.

\subsubsection{Return Value Registers: {\tt v0} - {\tt v1}}

Values returned from a function.  If the return value of the function
requires more than two registers to express, the remainder of the
return value is returned via the stack.

\subsubsection{Argument Value Registers: {\tt a0} - {\tt a5}}

Parameters to a function.  If the function has more than six
parameters, then the additional parameters are pushed onto the stack,
in the opposite order that they appear (right to left).

\subsubsection{Callee Saved Registers:  {\tt s0} - {\tt s23}}

If any of these registers are used by a function, then the function is
responsible for saving their original values and then restoring them
when the function returns.

How the values are preserved and restored is up to the implementation. 
For implementations of languages that permit recursive or reentrant
functions, using the stack is an appropriate method.

\subsubsection{Temporary (Caller Saved) Registers:  {\tt t0} - {\tt t23}}

If any of these registers contains live values when a function is
called, they are preserved by the caller and then restored after the
function has returned.

How the values are preserved and restored is up to the implementation. 
For implementations of languages that permit recursive or reentrant
functions, using the stack is an appropriate method.

The distinction between the saved registers and the temporary
registers allows some useful optimizations, especially with leaf
functions (functions that do not call any other functions) or the base
case of recursive functions.  If these functions use can manage to
exclusively use {\tt t}-registers, and their callers use only {\tt
s}-registers, then these calls do not require saving and restoring any
registers:  it is the responsibility of the caller to save any {\tt
t}-registers it needs, and the callee to save any {\tt s}-registers it
needs, so if the caller only uses {\tt s}-registers and the callee
only uses {\tt t}-registers, a significant reduction in the overhead
of function calls is obtained.



% $Id: t32-inst.tex,v 1.11 2002/04/16 15:19:09 ellard Exp $

% Local defs
% Might eventually have to move these to a global file.

\newcommand{\rdes}{{\em rdes}}
\newcommand{\srcA}{{\em src1}}
\newcommand{\srcB}{{\em src2}}
\newcommand{\constB}{{\em const8}}
\newcommand{\constH}{{\em const16}}
\newcommand{\const}{{\em const32}}
\newcommand{\SYN}{$\bullet$}
\newcommand{\itablehead}
	{\begin{tabular}{|p{0.1in}p{0.8in}p{1.3in}|p{3.4in}|}
	\hline
	& {\bf Mnemonic}	& {\bf Operands}	& {\bf Description} \\
	}

\chapter{Ant-32 Instruction Set Summary}
\label{ant32-inst}

\section{Notation}

The notations used to describe the instructions are summarized below.

\begin{center}
\begin{tabular}{|l|p{4.0in}|}
\hline
\Reg{\em x}		& The value stored in register {\em x}.
			\\
\constB			& Any 8-bit constant.
			\\
\constH			& Any 16-bit constant.
			\\
\const			& Any 32-bit constant.  A label can be used
				as a 32-bit constant.
			\\
\SYN			& An instruction description that begins with
				a \SYN\ symbol indicates
				that the instruction is {\em
				synthetic} (see Section
				\ref{t32-inst-syn-sec}). \\

\hline
\end{tabular}
\end{center}

\section{Differences Between Assembly Language and Machine Language}
\label{t32-inst-syn-sec}

The Ant-32 assembly language is closely related to the Ant-32
machine language, and there is always a simple mapping from
instructions in the assembly language to the corresponding machine
instructions.  All of the machine language instructions are directly
expressible in assembly language, but the assembly language also
provides a slightly higher-level abstraction of the machine (called
{\em synthetic instructions}) in order to reduce the tedium of
programming in Ant-32 assembly language, and provides several directives
to the assembler 


In the tables of instructions that follow in this chapter,
instructions that begin with a \SYN\ are {\em synthetic instructions}.
Synthetic instructions fall into two categories: mnemonic names for
operations directly supported by the hardware, and names for sequences
of instructions that implement operations not directly supported by
the hardware.

For an example of the first type, consider the {\tt mov} instruction. 
The Ant-32 hardware does not implement such an instruction, but
the same functionality can be achieved by using the {\tt add}
instruction with the zero register as one of the operands.\footnote{The
{\tt mov} instruction can also be implemented in many other ways.}

\begin{verbatim}
                # copy the contents of g1 into g0
                mov     g0, g1

                # This is the same as writing:
                add     g0, g1, ze
\end{verbatim}

As an example of the second type, consider the {\tt lc} (load
constant) instruction.  The Ant-32 hardware does not implement
any method to load a 32-bit constant into a register, but the same
effect can be achieved by using an {\tt lcl} (which loads a 16-bit
constant into the lower 16 bits of a register, performing sign
extension), followed by an {\tt lch} (which loads a 16-bit constant
into the upper 16 bits of a register).

\begin{verbatim}
                # load constant 0x12345678 into g0
                lc      g0, 0x12345678

                # this is the same as writing:
                lcl     g0, 0x5678
                lch     g0, 0x1234
\end{verbatim}

In many cases, the synthetic instructions have the same form as native
instructions.  For example, {\tt addi} (add immediate) exists in the
native instruction set, but only for 8-bit constants.  The assembler
will allow {\tt addi} to take a 32-bit constant, however, by using a
synthetic sequence of instructions to implement the desired
functionality.  Note that the assembler chooses the best way to
synthesize the instruction-- for example, different sequences will be
created to implement {\tt addi} depending on whether the constant
requires 8, 16, or 32-bits to express.

\section{Loading Constants}

\itablehead

\hline
\index{lch}
	& {\tt lch} &	\rdes, \constH &
		Load \constH\ into the top (high-order) 16 bits of \rdes. \\
\index{lcl}
	& {\tt lcl} &	\rdes, \constH &
		Load \constH\ into the lower 16 bits of \rdes,
		and perform sign extension to fill in the top
		16 bits of \rdes. \\
\index{lc}
\SYN	& {\tt lc} &	\rdes, \const &
		Load the \const\ into \rdes. \\
\hline

\end{tabular}

\section{Arithmetic Operations}

\itablehead

\hline
\index{add}
	& {\tt add} &	\rdes, \srcA, \srcB	&
		\rdes\ gets \Reg{\srcA}\ + \Reg{\srcB}. \\
\index{addi}
	& {\tt addi} &	\rdes, \srcA, \constB	&
		\rdes\ gets \Reg{\srcA}\ + \constB. \\
\SYN	& {\tt addi} &	\rdes, \srcA, \const	&
		\rdes\ gets \Reg{\srcA}\ + \const. \\
\index{sub}
	& {\tt sub} &	\rdes, \srcA, \srcB	&
		\rdes\ gets \Reg{\srcA}\ - \Reg{\srcB}. \\
\index{subi}
	& {\tt subi} &	\rdes, \srcA, \constB	& \rdes\ gets \Reg{\srcA}\ - \constB. \\
\SYN	& {\tt subi} &	\rdes, \srcA, \const	& \rdes\ gets \Reg{\srcA}\ - \const. \\
\index{mul}
	& {\tt mul} &	\rdes, \srcA, \srcB	& \rdes\ gets \Reg{\srcA}\ $\times$ \Reg{\srcB}. \\
\index{muli}
	& {\tt muli} &	\rdes, \srcA, \constB 	& \rdes\ gets \Reg{\srcA}\ $\times$ \constB. \\
\SYN	& {\tt muli} &	\rdes, \srcA, \const 	& \rdes\ gets \Reg{\srcA}\ $\times$ \const. \\
\index{div}
	& {\tt div} &	\rdes, \srcA, \srcB 	& \rdes\ gets \Reg{\srcA}\ $/$ \Reg{\srcB}. \\
\index{divi}
	& {\tt divi} &	\rdes, \srcA, \constB 	& \rdes\ gets \Reg{\srcA}\ $/$ \constB. \\
\SYN	& {\tt divi} &	\rdes, \srcA, \const 	& \rdes\ gets \Reg{\srcA}\ $/$ \const. \\
\index{mod}
	& {\tt mod} &	\rdes, \srcA, \srcB 	& \rdes\ gets \Reg{\srcA}\ {\bf modulo} \Reg{\srcB}. \\
\index{modi}
	& {\tt modi} &	\rdes, \srcA, \constB 	& \rdes\ gets \Reg{\srcA}\ {\bf modulo} \constB. \\
\SYN	& {\tt modi} &	\rdes, \srcA, \const 	& \rdes\ gets \Reg{\srcA}\ {\bf modulo} \const. \\
\hline
\end{tabular}
\vspace{3mm}

The ``o'' arithmetic operations are similar to the ordinary arithmetic
operations, except that they include the calculation of the
``overflow'', if any, from the operations.  For these operations,
\rdes\ must be an even-numbered register.  The result of the operation
is stored in registers \rdes\ and \rdes\ $ + 1$.  Consult the
architecture reference for more information.

\vspace{3mm}
\noindent
\itablehead
\hline
\index{addo}
	& {\tt addo} &	\rdes, \srcA, \srcB &
		Add with overflow. \\
\index{addio}
	& {\tt addio} &	\rdes, \srcA, \constB &
		Add immediate with overflow. \\
\SYN	& {\tt addio} &	\rdes, \srcA, \const & \\
\index{subo}
	& {\tt subo} &	\rdes, \srcA, \srcB &
		Subtract with overflow. \\
\index{subio}
	& {\tt subio} &	\rdes, \srcA, \constB &
		Subtract immediate with overflow. \\
\SYN	& {\tt subio} &	\rdes, \srcA, \const & \\
\index{mulo}
	& {\tt mulo} &	\rdes, \srcA, \srcB &
		Multiply with overflow. \\
\index{mulio}
	& {\tt mulio} &	\rdes, \srcA, \constB &
		Multiply immediate with overflow. \\
\SYN	& {\tt mulio} &	\rdes, \srcA, \const & \\
\hline
\end{tabular}

\section{Logical Bit Operations}

\itablehead

\hline
\index{and}
	& {\tt and} &	\rdes, \srcA, \srcB &
		\rdes\ gets the bitwise {\sc and} of \Reg{\srcA}\ and \Reg{\srcB}. \\
\SYN \index{andi}
	& {\tt andi} &	\rdes, \srcA, \const &
		\rdes\ gets the bitwise {\sc and} of \Reg{\srcA}\ and \const. \\
\index{nor}
	& {\tt nor} &	\rdes, \srcA, \srcB &
		\rdes\ gets the bitwise {\sc nor} of \Reg{\srcA}\ and \Reg{\srcB}. \\
\SYN \index{nori}
	& {\tt nori} &	\rdes, \srcA, \const &
		\rdes\ gets the bitwise {\sc nor} of \Reg{\srcA}\ and \const. \\
\index{or}
	& {\tt or} &	\rdes, \srcA, \srcB &
		\rdes\ gets the bitwise {\sc or} of \Reg{\srcA}\ and \Reg{\srcB}. \\
\SYN \index{ori}
	& {\tt ori} &	\rdes, \srcA, \const &
		\rdes\ gets the bitwise {\sc or} of \Reg{\srcA}\ and \const. \\
\index{xor}
	& {\tt xor} &	\rdes, \srcA, \srcB &
		\rdes\ gets the bitwise {\sc xor} of \Reg{\srcA}\ and \Reg{\srcB}. \\
\SYN \index{xori}
	& {\tt xori} &	\rdes, \srcA, \const &
		\rdes\ gets the bitwise {\sc xor} of \Reg{\srcA}\ and \const. \\

\hline
\end{tabular}

\section{Bit Shifting Operations}

\itablehead

\hline
\index{shl}
	& {\tt shl} &	\rdes, \srcA, \srcB &
			Shift \Reg{\srcA}\ left by \Reg{\srcB}\ bits. \\
\index{shli}
	& {\tt shli} &	\rdes, \srcA, \constB &
			Shift \Reg{\srcA}\ left by \constB\ bits. \\
\SYN	& {\tt shli} &	\rdes, \srcA, \const &
			Shift \Reg{\srcA}\ left by \const\ bits. \\
\index{shr}
	& {\tt shr} &	\rdes, \srcA, \srcB &
			Shift \Reg{\srcA}\ right by \Reg{\srcB}\ bits. \\
\index{shru}
	& {\tt shru} &	\rdes, \srcA, \srcB &
			Unsigned shift \Reg{\srcA}\ right
			by \Reg{\srcB}\ bits. \\
\index{shri}
	& {\tt shri} &	\rdes, \srcA, \constB &
			Shift \Reg{\srcA}\ right by \constB\ bits. \\
\SYN	& {\tt shri} &	\rdes, \srcA, \const &
			Shift \Reg{\srcA}\ right by \const\ bits. \\
\index{shrui}
	& {\tt shrui} &	\rdes, \srcA, \constB &
			Unsigned shift \Reg{\srcA}\ right by \constB\ bits. \\
\SYN	& {\tt shrui} &	\rdes, \srcA, \const &
			Unsigned shift \Reg{\srcA}\ right by \const\ bits. \\
\hline
\end{tabular}
\vspace{3mm}

The left shift operation shifts the bits ``left'', towards the more
significant bits, filling in the least significant bits with zeros. 
The right shift operations shift the bits toward the least significant
bits.  If the operation is ``unsigned'' then zeros are used to fill in
the most significant bits, but if the operation is not ``unsigned''
then a copy of the most significant bit in the \srcA\ register is
used to fill these bits.

\section{Load/Store Operations}

\itablehead
\hline
\index{st1}
	& {\tt st1} &	\srcA, \srcB, \constB &
			Store the least significant byte of \Reg{\srcA}\
			to the address \Reg{\srcB} + \constB. \\
\index{st4}
	& {\tt st4} &	\srcA, \srcB, \constB &
			Store \Reg{\srcA}
			to the address \Reg{\srcB} + \constB. \\
\index{ld1}
	& {\tt ld1} &	\rdes, \srcA, \constB &
			Load the byte at address \Reg{\srcA} + \constB\
			into \rdes.   The byte is sign-extended to 32-bits. \\
\index{ld4}
	& {\tt ld4} &	\rdes, \srcA, \constB &
			Load the word at address \Reg{\srcA} + \constB\
			into \rdes.  \\
\index{ex4}
	& {\tt ex4} &	\rdes, \srcA, \constB &
			Exchange the contents of register \rdes\ and
			the word at address \Reg{\srcA} + \constB\
			\\
\hline
\end{tabular}

\section{Comparison Instructions}

\itablehead
\hline
\index{eq}
	& {\tt eq} &	\rdes, \srcA, \srcB &
		\rdes\ gets 1 if \Reg{\srcA}\ $==$ \Reg{\srcB}, 0 otherwise. \\
\index{ges}
	& {\tt ges} &	\rdes, \srcA, \srcB &
		\rdes\ gets 1 if \Reg{\srcA}\ $\ge$ \Reg{\srcB}, 0 otherwise.
		The comparison uses signed numbers. \\
\index{gts}
	& {\tt gts} &	\rdes, \srcA, \srcB &
		\rdes\ gets 1 if \Reg{\srcA}\ $>$ \Reg{\srcB}, 0 otherwise.
		The comparison uses signed numbers. \\
\index{geu}
	& {\tt geu} &	\rdes, \srcA, \srcB &
		Like {\tt ges}, but using unsigned numbers. \\
\index{gtu}
	& {\tt gtu} &	\rdes, \srcA, \srcB &
		Like {\tt gts}, but using unsigned numbers. \\
\SYN \index{les}
	& {\tt les} &	\rdes, \srcA, \srcB &
		\rdes\ gets 1 if \Reg{\srcA}\ $\le$ \Reg{\srcB}, 0 otherwise.
		The comparison uses signed numbers. \\
\SYN \index{lts}
	& {\tt lts} &	\rdes, \srcA, \srcB &
		\rdes\ gets 1 if \Reg{\srcA}\ $<$ \Reg{\srcB}, 0 otherwise.
		The comparison uses signed numbers. \\
\SYN \index{leu}
	& {\tt leu} &	\rdes, \srcA, \srcB &
		Like {\tt les}, but using unsigned numbers. \\
\SYN \index{ltu}
	& {\tt ltu} &	\rdes, \srcA, \srcB &
		Like {\tt lts}, but using unsigned numbers. \\
\hline
\end{tabular}

\section{Branch and Jump Instructions}

\itablehead
\hline
\index{jez}
	& {\tt jez} &	\rdes, \srcA, \srcB &
			If \Reg{\srcA}\ is zero,
			jump to the address \Reg{\srcB}.
			\rdes\ gets the address of the current instruction.
			\\
\index{jnz}
	& {\tt jnz} &	\rdes, \srcA, \srcB &
			If \Reg{\srcA}\ is not zero,
			jump to the address \Reg{\srcB}.
			\rdes\ gets the address of the current instruction.
			\\
\SYN \index{jezi}
	& {\tt jezi} &	\rdes, \srcA, \const &
			If \Reg{\srcA}\ is zero, jump to \const.
			\rdes\ gets the value of the current instruction.
			\\
\SYN \index{jnzi}
	& {\tt jnzi} &	\rdes, \srcA, \const &
			If \Reg{\srcA}\ is not zero, jump to \const.
			\rdes\ gets the value of the current instruction.
			\\
\index{bez}
	& {\tt bez} &	\rdes, \srcA, \srcB &
			If \Reg{\srcA}\ is zero, branch to the address of
			the current instruction plus \Reg{\srcB}.
			\rdes\ gets the address of the current instruction.
			\\
\index{bnz}
	& {\tt bnz} &	\rdes, \srcA, \srcB &
			If \Reg{\srcA}\ is not zero, branch to the address of
			the current instruction plus \Reg{\srcB}.
			\rdes\ gets the address of the current instruction.
			\\
\index{bezi}
	& {\tt bezi} &	\srcA, \constH &
			If \Reg{\srcA}\ is zero, branch to
			the address of the current instruction
			$+ ~ (4 ~ \times$ \constH$)$.
			\\
\index{bnzi}
	& {\tt bnzi} &	\srcA, \constH &
			If \Reg{\srcA}\ is not zero, branch to
			the address of the current instruction
			$+ ~ (4 ~ \times$ \constH$)$.
			\\
\hline

\end{tabular}

\section{Console I/O Instructions}

\itablehead

\hline
\index{cin}
	& {\tt cin} &	\rdes &
			Read a character from the console into \rdes. \\
\index{cout}
	& {\tt cout} &	\srcA &
			Write the character \Reg{\srcA}\ to the console. \\
\hline

\end{tabular}

\section{Halting}
\label{halt-inst-sec}

\itablehead

\hline
\index{halt}
	& {\tt halt} & 	&	Stop the processor. \\
\hline

\end{tabular}

\section{Artificial Instructions}

\itablehead

\hline
\SYN \index{mov}
	& {\tt mov} & \rdes, \srcA	& Copy \Reg{\srcA}\ to \rdes. \\
\SYN \index{j}
	& {\tt j} & \const	& Jump to \const. \\
\SYN	& {\tt j} & \srcA	& Jump to \Reg{\srcA}. \\
\SYN \index{b}
	& {\tt b} & \const	& Branch to \const. \\
\SYN	& {\tt b} & \srcA 	& Branch to \Reg{\srcA}. \\
\hline
\SYN \index{push}
	& {\tt push} & \srcA  	& Push \Reg{\srcA}\ onto the stack. \\
\SYN \index{pop}
	& {\tt pop} & \rdes 	& Pop the stack into \rdes. \\
\hline
\SYN	& {\tt call}	& \const	& Call a function
				(see Section \ref{reg32-call-sec}). \\
\SYN	& {\tt entry}	& \const	& Create a stack frame
				(see Section \ref{reg32-entry-sec}). \\
\SYN	& {\tt return}	& \srcA	& Return \Reg{\srcA}\ from a function
				(see Section \ref{reg32-return-sec}). \\
\SYN	& {\tt return}	& \const	& Return the \const\ from
				a function
				(see Section \ref{reg32-return-sec}). \\
\hline
\end{tabular}



\appendix

\chapter{The Default Machine and ROM Routines}
\label{rom-chapter}
% 12/28/01
% $Id: rom32.tex,v 1.3 2002/04/16 15:19:09 ellard Exp $

\section{Introduction}

This section describes the default Ant-32 implementation and the
default ROM.  The default ROM supplied with the implementation of the
Ant-32 architecture contains a boot routine for initializing the
machine and several utility functions to simplify writing small
programs.

\section{Hardware Overview}

The default machine has 4 megs of physical RAM, contiguous from
physical address 0 to physical address {\tt 0x3fffff}.  Because of the
manner in which physical memory is addressed in (unmapped) system
mode, this means that this RAM appears to begin at virtual address
{\tt 0x80000000} and ends at {\tt 0x803fffff}.  Other RAM sizes are
possible, however, so it is a mistake to assume that this is always
the amount of RAM available.

In addition to the RAM, there are 4 pages (16K) of ROM located at the
top of the physical address space.  This small area of memory is where
the default ROM is located.

The details of the ROM are best described in the source code for the
ROM itself, and readers interested in more detail should refer to it. 
The source for the default ROM is provided as part of the normal
Ant-32 distribution.

\section{Initialization}

When the machine is booted, if the memory image was constructed in the
usual fashion (as described in {\tt aa32\_notes.html}), a short
initialization routine located in the ROM is called before execution
continues with the main program.

The boot ROM assumes that the memory image for the main program has
already been loaded into memory, starting at memory location {\tt
0x80000000}.

Note that there is nothing sacred here-- all of the initializations
done here can be overridden by the main program.  The purpose of the
routine supplied in the ROM is simply to supply reasonable defaults so
that it is, for many purposes, unnecessary to override anything.  The
only really important thing that the main code needs to take into
account is that exceptions are enabled by the ROM.

The steps taken by the initialization routine are as follows:

\begin{enumerate}

\item	{\em Determine the size of physical memory.}

	This is done by iteratively probing each page of RAM, starting
	at physical location 0 and continuing until either the address
	space of physical RAM is exhausted (at 1 Gbyte) or an invalid
	page is encountered. 

\item	{\em Initialize the {\tt sp} and {\tt fp} registers.}

	The frame pointer and stack pointer are initialized to point
	to the ``top'' of physical memory (via addresses in the
	unmapped segment).  Note that because of the way that the
	stack operations are implemented, the initial location pointed
	to by the frame pointer and stack pointer is actually one word
	{\em past} the end of physical memory.

	The initialization code assumes that it knows how much RAM is
	actually present.  It is possible to write this routine in
	such a way that it first detects how much memory there is in
	the machine, but this has not been implemented yet.
	
\item	{\em Prepare for Exceptions, and Zero the Cycle Counters}

	First, the exception handler is set to the address of a
	routine located in the ROM (named {\tt antSysRomEH}) that
	prints an error message and halts if a run-time exception
	occurs.  This is a minimal exception handler (since it
	doesn't really ``handle'' exceptions, it just makes the
	results a little less messy).

	Next, the {\em exception disable} flag is cleared, permitting
	exceptions to occur.

	Finally, the cycle counters and registers used by the
	probing routines are set to zero, 

\item	{\em Call the Main Code}

	The call to the main code of the program is implemented in the
	same manner as a zero-argument function call, so that if the
	``main'' of the program returns, this code will be able to
	properly halt the machine.

\end{enumerate}

\section{ROM Routines}

The functions in the ROM use the calling conventions described in
Chapter \ref{advanced-programming-chap}.  For routines that require
more than one parameter, the parameters are listed in the order that
they should be pushed onto the stack.

\subsection{Memory Management}

\begin{description}

\item[{\tt antSysSbrkInit}]

	Set the initial address of the boundary (aka {\em break})
	between preallocated memory and memory available for dynamic
	memory allocation.

	Note that all memory {\em after} this boundary (at higher
	addresses) is implicitly assumed to be available for dynamic
	allocation, which is not a completely accurate assumption,
	because the stack is also located in this region, and grows
	down towards the break.  If the break and the frontier of the
	stack cross, disaster is very likely.  Detecting this
	situation without adding costly overhead to every {\tt push}
	requires advanced techniques not described here.

\item[{\tt antSysSbrk}]

	Takes a single argument {\em size}, which is the amount to
	move the break.  The previous value of the break is returned.

	If the {\em size} is positive, the break is advanced,
	effectively allocating memory.  If the {\em size} is negative,
	memory is deallocated.

	Note that the {\em size} is always rounded up (towards
	positive infinity) to the nearest multiple of 4, in order to
	ensure that the break is always properly aligned for any
	memory access operation.  This can cause confusing behavior
	when trying to deallocate a small amount of memory.  For
	example, using {\tt antSysSbrk} with a size of 1 advances the
	break by 4 bytes (allocating 3 extra bytes), but using {\tt
	antSysSbrk} with a value of -1 does not move the break at all,
	so no memory is actually deallocated. 

\end{description}

\subsection{Simple I/O Routines}

\begin{description}

\item[{\tt antSysPrintString}]

	Print the zero-terminated {\sc ASCII} string pointed to by the
	argument.

\item[{\tt antSysPrintSDecimal}]

	Print the argument as a 32-bit signed decimal integer.

\item[{\tt antSysPrintUDecimal}]

	Print the argument as a 32-bit unsigned decimal integer.

\item[{\tt antSysPrintHex}]

	Print the argument as a 32-bit hexadecimal integer.

\item[{\tt antSysReadLine}]

	Read characters until end-of-line or end-of-input is reached. 
	(The behavior mimics the {\tt fgets} function from the
	standard C library.)

	This routine takes two parameters, which are pushed onto the
	stack in the following order:

	\begin{description}

	\item[buffer length] The maximum number of characters to read
		from the console.

	\item[buffer address] The address of the buffer to place the
		characters read from the console.

	\end{description}

\item[{\tt antSysReadDecimal}]

	Read characters from the console and interpret them as a
	32-bit signed decimal number, which is returned.

	Invalid input characters (such as non-digit characters, or a
	number too large to represent in 32 bits) will result in an
	arbitrary value being returned.  No error checking is
	performed.

\item[{\tt antSysReadHex}]
\index{antSysReadHex}

	Read characters from the console and interpret them as a
	32-bit hexadecimal number, which is returned.

	Invalid input characters (such as non-hex characters, or a
	number too large to represent in 32 bits) will result in an
	arbitrary value being returned.  No error checking is
	performed.

\end{description}



\chapter{Ant-32 Assembler Reference}
\label{ant32-asm-chapter}
% $Id: ant32-asm.tex,v 1.4 2002/04/17 20:05:58 ellard Exp $

\section{Comments and Whitespace}

A comment begins with a \verb$#$ and continues until the following
end-of-line.  The only exception to this is when the \verb$#$
character appears as part of an ASCII character constant (as described
in section \ref{data-const-sec}).

Once comments have been removed, any line that is not indented defines
a {\em label}.  The name of the label begins with the first character
of the line, and continues until a colon ({\tt :}) has been reached. 
All other lines must be indented.  The recommended level of
indentation is at least one tab-stop; additional indentation may be
used, at the discretion of the programmer, to clarify the program
structure.

\section{Summary of Directives}

\vspace{3mm}
\noindent
\begin{tabular}{|ll|p{4.0in}|}
\hline
{\bf Name}      & {\bf Parameters}      & {\bf Description}     \\
\hline
{\tt .text}	&					&
		Assemble the following assembly language statements
		as program instructions.  (This is the default.)
		\\
{\tt .data}	&					&
		Assemble the following assembly language statements
		as data.
		\\
\hline
{\tt .define}	& {\em name}, {\em value}		&
		Bind the {\em value} to the {\em name}.
		\\
\hline
{\tt .byte}     & {\em byte1, $\cdots$, byteN }		&
		Assemble the given byte values.
		\\
{\tt .word}     & {\em word1, $\cdots$, wordN }		&
		Assemble the given word (4-byte) values.
		\\
{\tt .ascii}    & {\tt "{\em string}"}			&
		Assemble the given string.  The string is
		not zero-terminated.
		\\
{\tt .asciiz}   & {\tt "{\em string}"}			&
		Assemble the given string, including the
		a zero-terminating byte.
		\\
\hline
{\tt .align}	& {\em size}				&
		Force alignment to the next address
		used by the assembler to the given {\em size},
		skipping over memory if needed.
		\\
\hline
\end{tabular}
\vspace{3mm}

% \section{The {\tt .text} and {\tt .data} Directives}

\section{Constants}
\label{data-const-sec}

Several Ant-32 assembly instructions contain 8, 16, or 32-bit constants.
A 32-bit constant can be specified in a variety of ways:
as decimal, octal, hexadecimal, or binary numbers, {\sc ASCII} codes (using
the same conventions as C), or labels.  Examples are shown in the
following table:

\begin{center}
\begin{tabular}{|l|l|l|}
\hline
Representation	& Value	& Decimal Value \\
\hline
{\em Decimal (base 10)}		&	{\tt 65}	&	65 \\
{\em Hexadecimal (base 16)}	&	{\tt 0x41}	&	65 \\
{\em Octal (base 8)}		&	{\tt 0101}	&	65 \\
{\em Binary (base 2)}		&	{\tt 0b01000001}&	65 \\
{\em {\sc ASCII}}		&	{\tt 'A'}	&	65 \\
\hline
{\em Decimal (base 10)}		&	{\tt 10}	&	10 \\
{\em Hexadecimal (base 16)}	&	{\tt 0xa}	&	10 \\
{\em Octal (base 8)}		&	{\tt 012}	&	10 \\
{\em Binary (base 2)}		&	{\tt 0b1010}	&	10 \\
{\em {\sc ASCII}}		&	{\tt '\verb$\$n'}	&	10 \\
\hline
\end{tabular}
\end{center}
\vspace{3mm}

The value of a label is the index of the subsequent instruction in
instruction memory for labels that appear in the code, or the index of
the subsequent {\tt .byte}, {\tt .word}, or {\tt .ascii} item for
labels that appear in the data.

The 8 and 16-bit constants can be specified in all the same ways as
the 32-bit constants {\em except} for labels, which are always 32
bits.

\section{Symbolic Constants}
\label{data-symconst-sec}
\index{.define}

Constants can be given symbolic names via the {\tt .define} directive. 
This can result in substantially more readable code.  The first
operand of the {\tt .define} directive is the symbolic name for the
constant, and the second value is an integer constant.  Unfortunately,
the integer constant must not be a label or another symbolic constant.

\vspace{3mm}
{\codesize
\begin{verbatim}
        .define ROWS, 10        # Defining ROWS to be 10
        .define COLS, 10        # Defining COLS to be 10

        lc      g2, ROWS        # Using ROWS as a constant
        addi    g3, g3, COLS    # Using COLS as a constant
\end{verbatim}}
\vspace{3mm}

Note that {\tt .define}'d constants can be redefined at any point.

\section{The {\tt .byte}, {\tt .word}, and {\tt .ascii} Directives}
\label{data-directive-sec}
\label{byte-figure}

The {\tt .byte} and {\tt .word} directives are used to specify data
values to be assembled into the next available locations in memory.
{\tt .byte} is used to assemble bytes, and {\tt .word} is used to
assemble 32-bit values.

\vspace{3mm}
\noindent
\begin{tabular}{|ll|p{4.0in}|}
\hline
{\bf Name}      & {\bf Parameters}      & {\bf Description}     \\
\hline
{\tt .byte}     & {\em byte1, $\cdots$, byteN }   &
		Assemble the given bytes (8-bit values) into the
		next available locations in the data segment.  As many
		as 8 bytes can be specified on the same line.  Bytes
		may be specified as hexadecimal, octal, binary, decimal
		or character constants.
                \\
{\tt .word}     & {\em word1, $\cdots$, wordN }   &
		Assemble the given words (32-bit values) into the
		next available locations in the data segment.  As many
		as 8 words can be specified on the same line.  Words
		may be specified as labels, hexadecimal, octal, binary, decimal
		or character constants.
                \\
{\tt .ascii}	& {\tt {\em "string"}}		&
		Assemble the given string (which must be enclosed
		in double quotes) as a sequence of 8-bit ASCII values.
		Note that a terminating zero is {\em not}
		added to string by {\tt .ascii}, and must be
		placed there explicitly if desired.
		\\
{\tt .asciiz}	& {\tt {\em "string"}}		&
		Assemble the given string (which must be enclosed
		in double quotes) as a sequence of 8-bit ASCII values.
		Unlike {\tt .ascii},
		a terminating zero byte is added to the end of the string.
		\\

\hline
\end{tabular}
\vspace{3mm}

\section{{\tt .align}}
\index{.align}

The Ant-32 architecture only allows memory references that are {\em
aligned} according to their size:  4-byte word reads and writes must
always be aligned on 4-byte boundaries (their address must always be
divisible by 4).  Byte reads and writes do not have any alignment
restrictions, since all addresses are divisible by 1.

The {\tt .align} directive is used to ensure that an address is
divisible by an arbitrary amount.  The {\tt .align} directive is used
to ensure that addresses are properly aligned.  The {\tt .align}
directive causes the assembler to skip to the next address which is a
multiple of its argument {\em size}.  (If the current address is a
multiple of the {\em size}, then no skip is needed.)

For example, to ensure that the address of a {\tt .word}
is aligned in a 4-byte boundary after an {\tt .ascii} string:

\begin{verbatim}
        .ascii  "hello"
        .align  4	# make sure that xxx is aligned on a word boundary
xxx :   .word   100
\end{verbatim}

This will ensure the address {\tt xxx} is aligned on a 4-byte boundary.

{\tt .align} can also be used to align on other boundaries, such as
page boundaries (by using a size of 4096).

Note that the alignment adjustment is done {\em after} the rest of the
line is processed, and therefore it is usually incorrect to
put a label definition on the same line as a {\tt .align}, because
the label will be assigned to a possibly misaligned address.  For example:

\begin{verbatim}
xxx:    .align  4       # WRONG: xxx might not be aligned
        .word   100     # xxx might not be the address of this word.

        .align  4       # RIGHT: yyy will be aligned properly
yyy:    .word   100     # yyy will be the address of this word
\end{verbatim}



\backmatter

\printindex

\end{document}
