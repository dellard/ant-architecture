% $Id: ant8.tex,v 1.10 2002/10/10 02:38:49 ellard Exp $

\section{Overview}

This document describes the architecture of the 8-bit {\sc Ant-8} processor.

{\sc Ant-8} is a very small and simple processor.

It contains 16 user-visible registers:  14 general-purpose registers
and two special registers.

The instruction set consists of 16 instructions.


\section{{\sc Ant-8} Memory Organization}

The memory of the {\sc Ant-8} processor consists of 256 8-bit
bytes.  This memory is shared by the instructions and the
data.

The {\sc Ant-8} architecture is a load/store architecture; the only
instructions that can access memory are the {\em load} and
{\em store} instructions.  All other operations access only
registers.

\section{The {\sc Ant-8} Register Set}

The {\sc Ant-8} processor has 16 registers that can be accessed directly by
the programmer.  In {\sc Ant-8} assembler, they are named {\tt r0} through
{\tt r15}.  In {\sc Ant-8} machine language, they are the 4-bit numbers 0
through 15, and are usually written as a single hexadecimal digit
({\tt 0x0} through {\tt 0xf}).

Registers {\tt r2} through {\tt r15} are general purpose registers. 
These registers can be used as both the source and destination
registers in any of the instructions that use source and destination
registers; they are read/write registers.

Registers {\tt r0} and {\tt r1} are not general purpose registers and
can be used only as source registers; they are read-only registers. 
{\tt r0} always contains the constant zero, and {\tt r1} is used to
hold values related to the result of the previous instruction.  Some
instructions do not modify {\tt r1} at all, but several use {\tt r1}
to store useful results.

If an instruction attempts to write a value to either {\tt r0}
or {\tt r1}, the instruction executes in the normal manner,
but no changes are made to the register.

The program counter (or {\sc PC}) is a special 8-bit register that
contains the offset (or index) into memory of the next instruction to
execute.  Each instruction is 2 bytes long, and each instruction must
begin on an even address.  Note that the offset is interpreted as an
unsigned number and therefore ranges from $0 \cdots \MaxUIntWord$. 
The {\sc PC} is not directly accessible to the program.

\section{Execution of Programs}

Programs are executed in the following manner:

\subsection{Initialization}

\begin{enumerate}


	\item Each location in memory is filled with zero.

	\item All of the registers are set to zero.

	\item The program counter (PC) is set to zero.

	\item The program is loaded into memory from a file.

		See section \ref{executable-file-sec} for information
		about the program file format.

	\item The fetch and execute loop (described in Section
		\ref{fetch-and-exec}) is executed until the program
		halts via the {\tt hlt} instruction, or because the
		execution encounters an error.
	
		Whenever the {\sc Ant-8} processor halts due to any error,
		it dumps core to a file named {\tt ant8.core}.

\end{enumerate}

\subsection{The Fetch and Execute Loop}
\label{fetch-and-exec}

\begin{enumerate}

\item Fetch the instruction at the offset in memory indicated
	by the {\sc PC}.

\item Set {\sc PC} $\leftarrow$ {\sc PC} $+ ~ 2$.

\item Execute the instruction.

	\begin{enumerate}
 
	\item Get the value of the source registers (if any).

	\item Perform the specified operation.

	\item Place the result, if any, into the destination register.

	\item Update register {\tt r1}, if necessary.

	\item Update the {\sc PC}, if necessary (only for branching or
			jumping instructions:  {\tt beq}, {\tt bgt}, and
			{\tt jmp} instructions).

	\end{enumerate}

\end{enumerate}

\section{The Instruction Set}

\subsection{Instruction Formats}
  
Most of the instructions have the following general instruction format:

\newcommand{\OPCODE}{{\em Operator \newline (4 bits)}}

\begin{description}

\item \ThreeRegisterOp{Op}{Description}
	{\parbox[t]{0.7in} \OPCODE }
	{\parbox[t]{0.7in}{Register 1 \\ (4 bits) }}
	{\parbox[t]{0.7in}{Register 2 \\ (4 bits) }}
	{\parbox[t]{0.7in}{Register 3 \\ (4 bits) }}
\end{description}

The lc (load constant), inc (increment), and I/O
({\tt in}/{\tt out}) instructions have the following format:

\begin{description}
\item \OneRegisterOp{Op}{Description}
	{\parbox[t]{0.7in} \OPCODE }
	{\parbox[t]{0.7in}{Register 1 \\ (4 bits) }}
	{\parbox[t]{0.7in}{Constant \\ (8 bits) }}
\end{description}


The final two exceptions to this are the {\tt ld1} (load) and {\tt
st1} (store) instructions, which have the following format:

\begin{description}
\item \TwoRegisterOp{Op}{Description}
	{\parbox[t]{0.7in} \OPCODE }
	{\parbox[t]{0.7in}{Register 1 \\ (4 bits) }}
	{\parbox[t]{0.7in}{Register 2 \\ (4 bits) }}
	{\parbox[t]{0.7in}{Constant \\ (4 bits) }}
\end{description}

The 4 bits of the {\em Operator} are a 1-digit hexadecimal number that
represents the name of the operator or instruction.  The 4 bits of the
Register(s) are the number of the register (i.e., {\tt 0x3} will
represent register 3).
 
\subsection{Notation}

The notation we will use to describe the operands of the
instructions is given in Figure \ref{operand-types-table}.

\begin{figure}[h]
\caption{ \label{operand-types-table} {\sc Ant-8} Machine Language Operand Types}
\vspace{3mm}
\begin{center}
\begin{tabular}{|lp{4.5in}|}
\hline

{\em des}       & Must always be a register index.  \\
{\em reg}       & Must always be a register index. \\
{\em src1}      & Must always be a register index. \\
{\em src2}      & Must always be a register index. \\
{\em const8}	& Must be an 8-bit constant
		($\MinIntWord \cdots \MaxIntWord$). \\
{\em uconst8}	& Must be an 8-bit constant
		($\MinUIntWord \cdots \MaxUIntWord$). \\
{\em uconst4}	& Must be a 4-bit constant integer
			($\MinUIntHWord \cdots \MaxUIntHWord$). \\
\hline
\end{tabular}

\vspace{3mm}
\begin{tabular}{|lrl|}
\hline
\MaxIntWord	&	127	& Maximum signed 8-bit integer. \\
\MinIntWord	&	-128	& Minimum signed 8-bit integer. \\
\hline
\MaxUIntWord	&	255	& Maximum unsigned 8-bit integer. \\
\MinUIntWord	&	0	& Minimum unsigned 8-bit integer. \\
\hline
\MaxUIntHWord	&	15	& Maximum unsigned 4-bit integer. \\
\MinUIntHWord	&	0	& Minimum unsigned 4-bit integer. \\
\hline
\end{tabular}
\end{center}
\vspace{3mm}
\end{figure}

Note that the same register can serve as both a source and destination
in one command.  For instance, you can double the contents of a
register by adding that register to itself and putting the result back
in that register, all in one command.

\subsection{Instruction Descriptions}

\begin{description}

\item	\renewcommand{\OPCODE}{{\tt 0 0 0 0}} % $Id

\OneRegisterOp{hlt}{Halt}{\OPCODE}{\em unused}{\em unused}

\index{hlt@{\tt halt} for Ant-8}

Dump core to {\tt ant8.core}, 
and halt the processor. PC should be the value of the last executed instruction.



\item	\renewcommand{\OPCODE}{{\tt 0 0 0 1}} % $Id: lc.tex,v 1.3 2002/04/16 01:02:41 ellard Exp $

\OneRegisterOp{lc}{Load constant}{\OPCODE}{des}{const8}

\index{lc@{\tt lc}!for Ant-8}

Load an 8-bit constant into a register.

\Reg{des} $\leftarrow$ {\em const8}.


\item	\renewcommand{\OPCODE}{{\tt 0 0 1 0}} % $Id: inc.tex,v 1.3 2002/04/16 01:02:40 ellard Exp $

\OneRegisterOp{inc}{Increment}{\OPCODE}{des}{const8}

\index{inc@{\tt inc}!for Ant-8}

Increment the value of a register by a constant.

\begin{enumerate}

\item Let {\em temp} $\leftarrow$  \Reg{des} + {\em const8}.

	\Reg{des} and {\em const8} are treated as
	8-bit 2's complement integers.

\item \Reg{des} $\leftarrow$ the lower eight bits of {\em temp}.

\item Detect whether overflow/underflow has taken place:

\begin{tabular}{lrl}

\Reg{1} $\leftarrow$ & 1  & if \MaxIntWord $<$ {\em temp} \\

\Reg{1} $\leftarrow$ & 0  &
		if \MinIntWord $\leq$ {\em temp} $\leq$ \MaxIntWord \\

\Reg{1} $\leftarrow$ & -1 & if {\em temp} $<$ \MinIntWord \\

\end{tabular}

\end{enumerate}


\item	\renewcommand{\OPCODE}{{\tt 0 0 1 1}} % $Id: jmp.tex,v 1.4 2002/04/16 01:02:40 ellard Exp $

\OneRegisterOp{jmp}{Jump}{\OPCODE}{(ignored)}{const8}

\index{jmp@{\tt jmp}!for Ant-8}

Jump to a constant address.

\begin{enumerate}

\item \Reg{1} $\leftarrow$ {\sc PC}.

	Note that {\sc PC} has already been incremented, and so the
	value left in \Reg{1} is set to the address of the instruction
	following the currently executing instruction.

\item {\sc PC} $\leftarrow$ {\em const8}.

\end{enumerate}

{\em const8} is treated as an unsigned 8-bit integer.


\item	\renewcommand{\OPCODE}{{\tt 0 1 0 0}} % $Id

\ThreeRegisterOp{beq}{Branch if equal}{\OPCODE}{reg1}{reg2}{reg3}

\index{beq@{\tt beq}!for Ant-8}

Branch if equal.

\begin{enumerate}

\item Branch to \Reg{reg1} if \Reg{reg2} is equal to \Reg{reg3}:

	\begin{tabular}{lll}

	{\sc PC} $~ \leftarrow$ & \Reg{reg1} &
			if \Reg{reg2} is equal to \Reg{reg3} \\

	{\sc PC} $~ \leftarrow$ & {\sc PC} & otherwise. \\

	\end{tabular}

\end{enumerate}


\item	\renewcommand{\OPCODE}{{\tt 0 1 0 1}} % $Id

\ThreeRegisterOp{bgt}{Branch if greater}{\OPCODE}{reg1}{reg2}{reg3}

\index{bgt@{\tt bgt}!for Ant-8}

Branch if greater than.
     
\begin{enumerate}

\item Branch to \Reg{reg1} if \Reg{reg2} $>$ \Reg{reg3}:

	\begin{tabular}{lll}

	{\sc PC} $\leftarrow$ & \Reg{reg1} & if \Reg{reg2} $>$ \Reg{reg3} \\

	{\sc PC} $\leftarrow$ & {\sc PC} & otherwise. \\

	\end{tabular}

	The contents of registers {\em reg2} and {\em reg3} are treated
	as 8-bit two's complement integers.

\end{enumerate}


\item	\renewcommand{\OPCODE}{{\tt 0 1 1 0}} % $Id: ld1.tex,v 1.9 2001/07/05 14:01:17 ellard Exp $

\renewcommand{\INSTmnemonic}	{ld1}
\renewcommand{\INSTdesc}	{Load Byte}
\renewcommand{\INSTopcode}	{0xE0}
\renewcommand{\INSToptype}	{TwoReg}
\renewcommand{\INSTfieldA}	{des}
\renewcommand{\INSTfieldB}	{src1}
\renewcommand{\INSTfieldC}	{const8}

\renewcommand{\INSTsemantic}	{
	\Reg{des} $\leftarrow$ \Mem{\Reg{src1} + {\em const8}}

}

\renewcommand{\INSTprose}	{

	Load sign-extended 8-bit byte.

}


\renewcommand{\INSTexceptions}	{
	Bus error
	}

\item	\renewcommand{\OPCODE}{{\tt 0 1 1 1}} % $Id: st1.tex,v 1.3 2002/10/09 23:53:08 ellard Exp $

\ThreeRegisterOp{st1}{Store one byte}{\OPCODE}{reg}{src1}{uconst4}

\index{st1@{\tt st1}!for Ant-8}

Store \Reg{reg} to memory.

Note that \Reg{src1} is treated as an unsigned value
		($\MinUIntWord \cdots \MaxUIntWord$).

\begin{enumerate}

\item If \Reg{src1} + {\em uconst4} $>$ \MaxUIntWord,
	an "invalid address" error occurs and the processor halts.
	{\em uconst4} is treated
	as an unsigned 4-bit integer, while \Reg{src1} is treated as
	an unsigned 8-bit value.

\item Store the contents of \Reg{reg} into data memory location
	(\Reg{src1} + {\em uconst4}).

\end{enumerate}


\item	\renewcommand{\OPCODE}{{\tt 1 0 0 0}} % $Id

\ThreeRegisterOp{add}{Addition}{\OPCODE}{des}{src1}{src2}

\index{add@{\tt add}!for Ant-8}

8-bit integer addition (with overflow).

\begin{enumerate}

\item Let {\em temp} $~ \leftarrow$  \Reg{src1} + \Reg{src2}.

	\Reg{src1} and \Reg{src2} are treated as
	8-bit 2's complement integers.

\item Detect whether overflow/underflow has taken place:

\begin{tabular}{lrl}

\Reg{1} $~ \leftarrow$ & 1  & if \MaxIntWord $< ~$ {\em temp} \\

\Reg{1} $~ \leftarrow$ & 0  &
	if \MinIntWord $\leq$ {\em temp} $\leq ~$ \MaxIntWord \\

\Reg{1} $~ \leftarrow$ & -1 & if {\em temp} $< ~$ \MinIntWord \\

\end{tabular}

\item \Reg{des} $\leftarrow$ the lower eight bits of {\em temp}.

\end{enumerate}


\item	\renewcommand{\OPCODE}{{\tt 1 0 0 1}} % $Id: sub.tex,v 1.8 2001/06/21 15:18:39 ellard Exp $

\renewcommand{\INSTmnemonic}	{sub}
\renewcommand{\INSTdesc}	{Subtraction}
\renewcommand{\INSTopcode}	{0x81}
\renewcommand{\INSToptype}	{ThreeReg}
\renewcommand{\INSTfieldA}	{des}
\renewcommand{\INSTfieldB}	{src1}
\renewcommand{\INSTfieldC}	{src2}

\renewcommand{\INSTsemantic}	{
	\Reg{des} $\leftarrow$ \Reg{src1} - \Reg{src2}
}


\renewcommand{\INSTexceptions}	{}

\item	\renewcommand{\OPCODE}{{\tt 1 0 1 0}} % $Id: mul.tex,v 1.8 2001/06/21 15:18:21 ellard Exp $

\renewcommand{\INSTmnemonic}	{mul}
\renewcommand{\INSTdesc}	{Multiplication}
\renewcommand{\INSTopcode}	{0x82}
\renewcommand{\INSToptype}	{ThreeReg}
\renewcommand{\INSTfieldA}	{des}
\renewcommand{\INSTfieldB}	{src1}
\renewcommand{\INSTfieldC}	{src2}

\renewcommand{\INSTsemantic}	{
	\Reg{des} $\leftarrow$ \Reg{src1} $\times$ \Reg{src2}
}


\renewcommand{\INSTexceptions}	{}

\item	\renewcommand{\OPCODE}{{\tt 1 0 1 1}} % $Id: shf.tex,v 1.3 2002/04/16 01:02:41 ellard Exp $

\ThreeRegisterOp{shf}{Bit shift}{\OPCODE}{des}{src1}{src2}

\index{shf@{\tt shf}!for Ant-8}

Bit shift.

\begin{enumerate}

\item

\begin{tabular}{lll}

\Reg{des} $\leftarrow$ 
	& \Reg{src1} \verb$<<$ \Reg{src2} & if \Reg{src2} $>$ 0 \\

\Reg{des} $\leftarrow$ 
	& \Reg{src1} \verb$>>$ $-$ \Reg{src2} & otherwise \\

\end{tabular}

	The contents of registers {\em src1} and {\em src2}
	are treated as 8-bit two's complement integers.

\item \Reg{1} $\leftarrow$ 0.

\end{enumerate}

Right shifts are done without sign extension, mimicking the
behavior of the {\sc C} operator \verb$>>$ for unsigned integers.


\item	\renewcommand{\OPCODE}{{\tt 1 1 0 0}} % $Id: and.tex,v 1.7 2001/06/21 15:17:42 ellard Exp $

\renewcommand{\INSTmnemonic}	{and}
\renewcommand{\INSTdesc}	{Bitwise {\sc and}}
\renewcommand{\INSTopcode}	{0x88}
\renewcommand{\INSToptype}	{ThreeReg}
\renewcommand{\INSTfieldA}	{des}
\renewcommand{\INSTfieldB}	{src1}
\renewcommand{\INSTfieldC}	{src2}

\renewcommand{\INSTsemantic}	{
	\Reg{des} $\leftarrow$ \Reg{src1} {\sc and} \Reg{src2}
}


\renewcommand{\INSTexceptions}	{}

\item	\renewcommand{\OPCODE}{{\tt 1 1 0 1}} % $Id: nor.tex,v 1.4 2002/04/16 01:02:41 ellard Exp $

\ThreeRegisterOp{nor}{Bitwise {\sc nor}}{\OPCODE}{des}{src1}{src2}

\index{nor@{\tt nor}!for Ant-8}

Bitwise logical {\sc nor}.

\begin{enumerate}

\item {\em temp} $\leftarrow$ $\sim$ ( \Reg{src1} \verb$|$ \Reg{src2} )
		(the bitwise {\sc or} of \Reg{src1} and \Reg{src2}).

\item \Reg{des} $\leftarrow$ {\em temp};

\item \Reg{1}   $\leftarrow$ $\sim$ {\em temp}
		(the bitwise negation of {\em temp}).

\end{enumerate}


\item	\renewcommand{\OPCODE}{{\tt 1 1 1 0}} % $Id: in.tex,v 1.6 2002/10/09 23:53:08 ellard Exp $


\TwoRegisterOp{in}{Input}{\OPCODE}{reg}{\em unused}{uconst4}

\index{in@{\tt in} for Ant-8}

The {\tt in} instruction is used to perform input
of a single character from a peripheral device.

The 4 bits of the {\em uconst4} determine the ``peripheral'' from which
the byte is read.  The standard Ant-8 implementation has three
peripherals:

\begin{description}

\item[{\tt 0000}] Hexadecimal --
	the value is treated as a two-digit hexadecimal number.

\item[{\tt 0001}] Binary --
	the value is treated as an eight-digit binary number.

\item[{\tt 0010}] {\sc ASCII} --
	the value is interpreted as an ASCII character code.

\end{description}

\begin{itemize}

\item \Reg{reg} $~ \leftarrow$ one byte read from the input, in one of
	the formats.

\item \Reg{1} $~ \leftarrow$ 1 if end-of-input (EOI) has been reached,
	0 otherwise.

\end{itemize}


\item	\renewcommand{\OPCODE}{{\tt 1 1 1 1}} % $Id: out.tex,v 1.5 2002/10/09 23:53:08 ellard Exp $

\TwoRegisterOp{out}{Output}{\OPCODE}{\em unused}{reg}{uconst4}

\index{out@{\tt out} for Ant-8}

The {\tt out} instruction is used to perform output
of a single character to a peripheral device. {\tt r1} is
set to zero.

The 4 bits of the {\em uconst4} determine the ``peripheral'' to which
the byte is written.  The standard Ant-8 implementation has three
peripherals:

\begin{description}

\item[{\tt 0000}] Hexadecimal --
	the value is treated as a two-digit hexadecimal number.

\item[{\tt 0001}] Binary --
	the value is treated as a eight-digit binary number.

\item[{\tt 0010}] {\sc ASCII} --
	the value is interpreted as an {\sc ASCII} character code.

\end{description}





\end{description}

\section{{\sc Ant-8} Executable Files}
\label{executable-file-sec}

{\sc Ant-8} program files are stored as text, as a sequence of
hexadecimal numbers, one per line.  Anything that appears after the
number on each line is ignored by the program loader, although it may
contain information that is used by the debugger.  Empty blank lines
or lines that begin with a \verb$#$ are ignored during the execution
of the program, although they may contain information used by the {\sc
Ant-8} debugger.  Each line in a program file must be less than 512
characters in length, and each line must end with a newline.

The program begins with the instructions, which are written as pairs
of 8-bit hexadecimal numbers.  If any of the hexadecimal numbers that
are supposed to represent instructions are too large to fit into 8
bits, then the program is invalid.

The loader reads bytes from the file until either the end of file is
reached, or 256 bytes have been read.

If there are fewer than 256 bytes specified in the file, the bytes not
specified are implicitly {\tt 0x00}.


