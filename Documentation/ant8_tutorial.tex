% 07/09/99
% $Id: ant8_tutorial.tex,v 1.13 2002/04/17 20:05:59 ellard Exp $

\documentclass[11pt,makeidx,psfig]{book}
\usepackage{ifthen}
\usepackage{makeidx}

% % macros.tex
%
% Definitions of commands extending Latex for the CS51 Project Book.
%
% Bob Walton
% Dec 7, 1993

\setcounter{secnumdepth}{5}
\setcounter{tocdepth}{5}

% \AND			\wedge, the math AND operator
% \OR			\vee, the math OR operator
% \NOT			\sim, the math NOT operator
% \IMPLIES		\Rightarrow, the math IMPLIES operator
% \EQUIV		\equiv, the math EQUIVALENT-TO operator
%
% \contributors{<contributor>, <contributor>, ...}
%			Specifies comma separated list of contributors to
%			be acknowledged in the preface.
%
% \begin{history}{<filename>}
% <history>		Displays <history> as a quote if \makehistory
% \end{history}		has been included in the input.  Histories are
%			used to indicate author, date, and changes to
%			project documentation files.  Histories are not
%			printed in the copy of the project book received
%			by students.
%
%			The <filename> is the name of the file whose
%			history is being given.  It is printed at the
%			beginning of the history.  The extension is
%			assumed to be .tex, and is not included in
%			<filename>.
%
%			\end{history} must appear literally without
%			any spaces just like \end{verbatim}.
%			
% \EOL			Indicates a spot in a word where a line break
%			is OK.  E.g. {\tt car-\EOL stream}.
%
% \makehistory		Causes histories to be displayed.
%
% \inputfigure{<file>}{<height>}
%			Inputs a postscript figure from the indicated
%			file and makes it the indicated height.
%
%			Extra vertical space will be put before \inputfigure
%			only if a blank line proceeds it.  Extra vertical
%			space will be put after \inputfigure only if
%			a blank line follows it.
%
% \cbox{<code>} 	Displays <code> as code names or phrases
%			embedded in text.  Similar to \verb but
%			always terminates with }.
%
%			Restriction: \cbox cannot appear in any macro
%			argument.
%
% \$<code>$		Same as \cbox{<code>} except <code> is terminated
%			by a $.  Cannot appear in any macro argument.
%
% \dollar		The $ character.  Same as \$ in normal Latex,
%			(we redefine \$ above to allow \$<code>$).
%
% \cindex{<translated-code>}
%			Similar to index, and in fact equivalent to:
%
%			  \index{<translated-code>@\tt <translated-code>}
%
%			where <translated-code> is <code> with the special
%			characters translated for latex: namely
%				``\'' is denoted by ``$\backslash$''
%				``^'' is denoted by ``\^{~}''
%				``~'' is denoted by ``$\sim$''
%				``$'' is denoted by ``\dollar''
%				``#'' is denoted by ``\#''
%				``%'' is denoted by ``\%''
%				``&'' is denoted by ``\&''
%				``_'' is denoted by ``\_''
%				``{'' is denoted by ``\{''
%				``}'' is denoted by ``\}''
%
% \ckey{<code>}		Same as \cbox but puts <code> in index as per
%			\index.
%
%			Warning: because of limitations of latex, <code>
%			is communicated to the index as {\tt <code>},
%			and therefore <code> cannot contain special
%			characters.  If you need special characters,
%			use \ickey.
%
% \ickey{<code>}{<translated-code>}
%			Must be used in place of \ckey when <code>
%			contains characters that cannot be included
%			in the scope of \tt.
%
%			Same as \ckey but calls \cindex with
%			<translated-code> instead of <code>.
%
% \key{<text>}		Applies \em or \bf to <text> and also puts <text>
%			in the index.
%
% \ikey{<text>}{<itext>}
%			Applies \em or \bf to the first argument <text>,
%			and puts the second argument <itext> in the index.
%
% \begin{code}{<name>}	Displays <code>.  Similar to \begin{verbatim}
% <code>		but indents or centers <code>.
% \end{code}
%			\end{code} must appear literally without
%			any spaces just like \end{verbatim}.
%
%			<name> is ignored during text processing.  It is
%			used to extract the <code> from the file so it can
%			be tested.
%
%			Extra vertical space will be put before \begin{code}
%			only if a blank line proceeds it.  Extra vertical
%			space will be put after \end{code} only if
%			a blank line follows it.
%
% \begin{code*}{<name>}	Ditto but displays spaces as square cups, like
% <code>		verbatim*.
% \end{code*}		WARNING: this macro has NEVER WORKED.
%
% \begin{labpar}{<label>}
%			Begin labeled paragraph, which is an indented
%			paragraph with <label> in the left margin.
% \end{labpar}		End labeled paragraph.
%
% \begin{indentpar}	Begin indented paragraph ({list} with no label).
% \end{indentpar}	End indented paragraph.
%
% \spc			A temporary length command usable in conjunction
%			with \settowidth.
%
% \begin{problems}	Begin a list of problems.
% \problem		Start the next problem.
% \end{problems}	End a list of problems.
%
% \begin{moreproblems}	Continue a list of problems.
%			(Just like problems environment,
%			 but does not reset problem number counter.)
% \end{moreproblems}	End a continued list of problems.
%
% \begin{subproblems}	Begin a list of subproblems.
% \subproblem		Start the next subproblem.
% \end{subproblems}	End a list of subproblems.
%
% \begin{boxpicture}{<x-size>}{<y-size>}
% <picture commands>	Like \begin{picture} put sets the size of one
% \end{boxpicture}	unit so that 100 units is the side of a square
%			box suitable from making dotted pair diagrams.
%			<x-size> and <y-size> are the sizes of the
%			picture in these units.  (0,0) is the lower
%			left corner, (<x-size>-1,<y-size>-1) is the
%			upper right corner.
%
% \cons{<x>}{<y>}	Draws a CONS cell consisting of two side-by-side
%			boxes.  The Lower left coordinate of the left
%			box is  (<x>,<y>), while the lower left coordinate
%			of the right box is (<x>+100,<y>).
%
% \nil{<x>}{<y>}	Draws the lower left to upper right slant line
%			that indicates a box points at NIL.  <x>,<y> are
%			the lower left coordinates of the slant line.
%
% \sym{<x>}{<y>}{<name>}
%			Draws a symbol, which is just the symbol <name>,
%			a piece of text, centered in the box whose lower
%			left coordinates are <x>, <y>.
%
% \lab{<x>}{<y>}{<pos>>}{<text>}
%			Ditto, but positions the <text>:
%			    Against the left side of the box if <pos> = l.
%			    Against the right side of the box if <pos> = r.
%			    Against the top of the box if <pos> = t.
%			    Against the bottom of the box if <pos> = b.
%
%			<pos> can consist of two letters, one for
%			horizontal control and one for vertical control.
%			The default is to center.
%			
%
% \cpointer{<x1>}{<y1>}{<x2>}{<y2>}
%			Draws a pointer from a CONS cell box (either left
%			(car) or right (cdr)) to another box.  (<x1>,<y1>)
%			are the coordinates of the CONS cell box (its
%			lower leftmost point, actually), and (<y1>,<y2>)
%			are the coordinates of the target box (also
%			its lower leftmost point).  The pointer actually
%			goes from the middle of the CONS box toward the
%			middle of the target box, but stops at the
%			boundary of the target box.
%
%			Restriction: pointers must be horizontal pointing
%			right; vertical pointing down; or at 45 degree
%			angles pointing either down left of down right.
%
% \spointer{<x1>}{<y1>}{<x2>}{<y2>}
%			Ditto but the pointer, instead of starting from
%			the middle of the box, starts from the edge of the
%			box.  Used to point from a symbol to any box.
%
% \sline{<x1>}{<y1>}{<x2>}{<y2>}
%			Ditto but a line is drawn instead of a pointer
%			(i.e. there is no arrowhead).
%

\newcommand{\AND}{\wedge}
\newcommand{\OR}{\vee}
\newcommand{\NOT}{\sim}
\newcommand{\IMPLIES}{\Rightarrow}
\newcommand{\EQUIV}{\equiv}

\newcommand{\EOL}{\penalty \exhyphenpenalty}

\newcount\ATCATCODE
\ATCATCODE=\catcode`@

\catcode `@=11	% @ is now a letter

% The following are alterations of latex.tex macros.

% Variation on @ifnextchar:
\long\def\ifnexttoken#1#2#3{\let\@tempe #1\def\@tempa{#2}\def\@tempb{#3}%
	\futurelet\@tempc\@ifnch}

\def\inputfigure#1#2{\ifvmode \vspace{2ex}\else \par\fi
		    \centerline{\psfig{figure=#1,height=#2}}
		    \ifnexttoken\par{\vspace{2ex}}{}}

\begingroup \catcode `|=0 \catcode `[= 1
\catcode`]=2 \catcode `\{=12 \catcode `\}=12
\catcode`\\=12
|gdef|@xcode#1\end{code}%
	[#1|end[code]|ifnexttoken|par[|vspace[2ex]][]]
|gdef|@sxcode#1\end{code*}%
	[#1|end[code*]|ifnexttoken|par[|vspace[2ex]][]]
|long|gdef|@xhistory#1\end{history}[|end[history]]
|endgroup

\newlength{\codewidth}

\def\code #1{\ifvmode \vspace{2ex}\else \par\fi % must be before setlength
	     \setlength{\codewidth}{\linewidth}%
	     \addtolength{\codewidth}{-\parindent}%
	     \begin{minipage}[t]{\codewidth}%\small
	     \@verbatim \frenchspacing\@vobeyspaces \@xcode}
\def\endcode{\endtrivlist\if@endpe\@doendpe\fi\end{minipage}}

\@namedef{code*} #1{%
	     \ifvmode \vspace{2ex}\else \par\fi % must be before setlength
	     \setlength{\codewidth}{\linewidth}%
	     \addtolength{\codewidth}{-\parindent}%
	     \begin{minipage}[t]{\codewidth}%\small
	     \@verbatim \@xscode}
\@namedef{endcode*}{\endtrivlist\if@endpe\@doendpe\fi\end{minipage}}

\let\dollar\$
% \begingroup \catcode `|=0 \catcode `[= 1
% \catcode`]=2 \catcode `\{=12 \catcode `\}=12 \catcode `$=12
% \catcode`\\=12
% |gdef|cbox #1[|verb }#1}]
% |gdef|$[|verb $]
% |endgroup

\gdef\cindex #1{\index{#1@{\tt #1}}}
\gdef\ckey #1{\cbox{#1}\index{#1@{\tt #1}}}
\gdef\ickey #1#2{\cbox{#1}\index{#2@{\tt #2}}}

\gdef\key #1{{\em #1}\index{#1}}
\gdef\ikey #1#2{{\em #1}\index{#2}}

\def\history#1{\begingroup \@noligs \let\do\@makeother \dospecials \@xhistory}
\def\endhistory{\endgroup}
\def\makehistory{\def\history##1{\begin{quote} \small {\bf ##1}:}
		 \def\endhistory{\end{quote}}}

\def\contributors #1{\def\contributorlist{#1}}

% End of altered latex.tex macros.

\catcode `@=\ATCATCODE	% @ is now restored

\newenvironment{labpar}[1]%
	{\begin{list}{#1}{\settowidth{\leftmargin}{#1}%
			  \addtolength{\leftmargin}{\labelsep}%
			  \settowidth{\labelwidth}{#1}%
			  \setlength{\partopsep}{\parskip}%
			  \setlength{\parskip}{0in}%
			  \setlength{\topsep}{0in}%
			  \item}}%
	{\end{list}}

\newenvironment{indentpar}%
	{\begin{list}{}{}\item}%
	{\end{list}}

\newcounter{pnumber}
\newenvironment{problems}%
	{\begin{list}{\arabic{pnumber}.}{\usecounter{pnumber}}}%
	{\end{list}}
\newcommand{\problem}{\item}

\newcounter{mpnumber}
\newenvironment{moreproblems}%
	{\setcounter{mpnumber}{\value{pnumber}}%
	 \begin{list}{\arabic{pnumber}.}{\usecounter{pnumber}}%
	 \setcounter{pnumber}{\value{mpnumber}}}%
	{\end{list}}

\newcounter{spnumber}
\newenvironment{subproblems}%
	{\begin{list}{(\alph{spnumber})}{\usecounter{spnumber}}}%
	{\end{list}}
\newcommand{\subproblem}{\item}

\newlength{\spc}

\newcount\TCOUNTX
\newcount\TCOUNTY
\newcount\TCOUNTZ

\newenvironment{boxpicture}[2]%
	{\setlength{\unitlength}{0.002in}
	 \begin{picture}(#1,#2)
	}%
        {\end{picture}}

\newcommand{\cons}[2]
	{\put(#1,#2){\begin{picture}(200,100)
		     \put(0,0){\framebox(200,100){}}
		     \put(100,0){\line(0,1){100}}
		     \end{picture}
		    }
	}

\newcommand{\sym}[3]
	{\put(#1,#2){\makebox(100,100){\tt #3}}
        }

\newcommand{\lab}[4]
	{\put(#1,#2){\makebox(100,100)[#3]{\tt #4}}
        }

\newcommand{\nil}[2]
        {\put(#1,#2){\line(1,1){100}}
        }


\newcommand{\cpointer}[4]
       	{\TCOUNTX=#1
	 \advance\TCOUNTX by 50
	 \TCOUNTY=#2
	 \advance\TCOUNTY by 50

	 \ifnum#2=#4 {\TCOUNTZ=#3
		      \advance\TCOUNTZ by -\TCOUNTX
		      \put(\number\TCOUNTX,\number\TCOUNTY)
				{\vector(1,0){\TCOUNTZ}}
		     }
	\else {
		\ifnum#1<#3 {\TCOUNTZ=#3
		     	     \advance\TCOUNTZ by -\TCOUNTX
		     	     \put(\number\TCOUNTX,\number\TCOUNTY)
					{\vector(1,-1){\TCOUNTZ}}
		    	    } 
		\fi
		\ifnum#1=#3 {\TCOUNTZ=#4
			     \advance\TCOUNTZ by 100
			     \multiply\TCOUNTZ by -1
			     \advance\TCOUNTZ by \TCOUNTY
		     	     \put(\number\TCOUNTX,\number\TCOUNTY)
					{\vector(0,-1){\TCOUNTZ}}
		    	    } 
		\fi
		\ifnum#1>#3 {\TCOUNTZ=#3
			     \advance\TCOUNTZ by 100
			     \multiply\TCOUNTZ by -1
		     	     \advance\TCOUNTZ by \TCOUNTX
		     	     \put(\number\TCOUNTX,\number\TCOUNTY)
					{\vector(-1,-1){\TCOUNTZ}}
		    	    } 
		\fi
	      }

	\fi
        }

\newcommand{\spointer}[4]
       	{\TCOUNTX=#1
	 \advance\TCOUNTX by 50
	 \TCOUNTY=#2
	 \advance\TCOUNTY by 50

	 \ifnum#2=#4 {\TCOUNTZ=#3
		      \advance\TCOUNTX by 50
		      \advance\TCOUNTZ by -\TCOUNTX
		      \put(\number\TCOUNTX,\number\TCOUNTY)
				{\vector(1,0){\TCOUNTZ}}
		     }
	\else {
		\ifnum#1<#3 {\TCOUNTZ=#3
			     \advance\TCOUNTX by 50
			     \advance\TCOUNTY by -50
		     	     \advance\TCOUNTZ by -\TCOUNTX
		     	     \put(\number\TCOUNTX,\number\TCOUNTY)
					{\vector(1,-1){\TCOUNTZ}}
		    	    } 
		\fi
		\ifnum#1=#3 {\TCOUNTZ=#4
			     \advance\TCOUNTY by -50
			     \advance\TCOUNTZ by 100
			     \multiply\TCOUNTZ by -1
			     \advance\TCOUNTZ by \TCOUNTY
		     	     \put(\number\TCOUNTX,\number\TCOUNTY)
					{\vector(0,-1){\TCOUNTZ}}
		    	    } 
		\fi
		\ifnum#1>#3 {\TCOUNTZ=#3
			     \advance\TCOUNTX by -50
			     \advance\TCOUNTY by -50
			     \advance\TCOUNTZ by 100
			     \multiply\TCOUNTZ by -1
		     	     \advance\TCOUNTZ by \TCOUNTX
		     	     \put(\number\TCOUNTX,\number\TCOUNTY)
					{\vector(-1,-1){\TCOUNTZ}}
		    	    } 
		\fi
	      }

	\fi
        }

\newcommand{\sline}[4]
       	{\TCOUNTX=#1
	 \advance\TCOUNTX by 50
	 \TCOUNTY=#2
	 \advance\TCOUNTY by 50

	 \ifnum#2=#4 {\TCOUNTZ=#3
		      \advance\TCOUNTX by 50
		      \advance\TCOUNTZ by -\TCOUNTX
		      \put(\number\TCOUNTX,\number\TCOUNTY)
				{\line(1,0){\TCOUNTZ}}
		     }
	\else {
		\ifnum#1<#3 {\TCOUNTZ=#3
			     \advance\TCOUNTX by 50
			     \advance\TCOUNTY by -50
		     	     \advance\TCOUNTZ by -\TCOUNTX
		     	     \put(\number\TCOUNTX,\number\TCOUNTY)
					{\line(1,-1){\TCOUNTZ}}
		    	    } 
		\fi
		\ifnum#1=#3 {\TCOUNTZ=#4
			     \advance\TCOUNTY by -50
			     \advance\TCOUNTZ by 100
			     \multiply\TCOUNTZ by -1
			     \advance\TCOUNTZ by \TCOUNTY
		     	     \put(\number\TCOUNTX,\number\TCOUNTY)
					{\line(0,-1){\TCOUNTZ}}
		    	    } 
		\fi
		\ifnum#1>#3 {\TCOUNTZ=#3
			     \advance\TCOUNTX by -50
			     \advance\TCOUNTY by -50
			     \advance\TCOUNTZ by 100
			     \multiply\TCOUNTZ by -1
		     	     \advance\TCOUNTZ by \TCOUNTX
		     	     \put(\number\TCOUNTX,\number\TCOUNTY)
					{\line(-1,-1){\TCOUNTZ}}
		    	    } 
		\fi
	      }

	\fi
        }

%%% 02/26/94
%%% An attempt at writing my own macros...

\newcommand{\codesize}{\small}

%%% algorithm environment:
%%%     Arg 1 is some description of the algorithm.
%%%     Arg 2 is the reference name of the algorithm.
\newenvironment{algorithm}[2]{
        \begin{figure}[hbtp]
        \hrule
	\vspace{2mm}
        \begin{alghead}
        \label{#2}
        {\em #1}
        \end{alghead}
        % \hrule
}{
        \vspace{3mm}
        \hrule
        \end{figure}
}

%%% dan-code environment:
%%%     Arg 1 is the name of the program or code fragment.
%%%     Arg 2 is the reference name of the code.
%%%     Arg 3 is any additional text to add to the preface of the code.
\newenvironment{dan-code}[3]{
        \newpage
        \section{{\tt #1}}
        \label{#2}
        \index{#1 (complete listing)}
        #3
        \vspace{3mm}
        \hrule
        \vspace{3mm}
        \footnotesize
}{
        \normalsize
        \vspace{3mm}
        \hrule
}

\newcommand{\marginnote}[1]{\marginpar{{\bf {\footnotesize #1}}}}

%% The variables must be set before any of the following macros
%% can be used.  They should be assigned in the root of the document,
%% not here, however.
%%
%% UseDefaultAntOpcodes - True if the Ant uses the default opcodes
%%	(whatever they happen to be at any given instant...) instead
%%	of some variation.  Currently, the only variation is for QRR,
%%	which omits sys and adds halt and in/out.
%%
%% AllowUnwriteableDes - The current ANT faults if the program tries
%%	to write r0 or r1.  If this bool is true, then the processor
%%	does not fault.  I just silently ignores the write, leaving
%%	r0 with zero (always) and r1 with whatever it would have been
%%	had the destination register been something else...
%%
%%	As if 1/19, the default is to fault.  However, this is very
%%	likely to change in the future.

\newboolean{UseDefaultAntOpcodes}
\newboolean{AllowUnwriteableDes}

%%% end of dan.tx

% $Id: ant-macros.tex,v 1.7 2002/01/02 02:13:47 ellard Exp $

\newcommand{\ThreeRegisterOp}[6]{
	\vspace{0.1in}
	\begin{tabular}{p{0.6in}p{2.4in}|p{0.6in}|p{0.6in}|p{0.6in}|p{0.6in}|}
	\cline{3-6}
	\fbox{\tt\Large #1} & {\bf #2} &
			#3 & {\em #4} & {\em #5} & {\em #6} \\
	\cline{3-6}
	\end{tabular}
	\vspace{0.1in}
}

% TwoRegisterOp is actually exactly the same as ThreeRegisterOp right now.
% It exists only to clarify the spec (and because perhaps in the future
% we will distinguish them).

\newcommand{\TwoRegisterOp}[6]{\ThreeRegisterOp{#1}{#2}{#3}{#4}{#5}{#6}}

\newcommand{\OneRegisterOp}[5]{
	\vspace{0.1in}
	\begin{tabular}{p{0.6in}p{2.4in}|p{0.6in}|p{0.6in}|p{1.4in}|}
	\cline{3-5}
	\fbox{\tt\Large #1}	& {\bf #2} &
			#3 & {\em #4} & {\em #5} \\
	\cline{3-5}
	\end{tabular}
	\vspace{0.1in}
}

\newcommand{\MaxIntWord}{{\bf\tt MAX\_INT8}}
\newcommand{\MinIntWord}{{\bf\tt MIN\_INT8}}
\newcommand{\MaxUIntWord}{{\bf\tt MAX\_UINT8}}
\newcommand{\MinUIntWord}{{\bf\tt MIN\_UINT8}}
\newcommand{\MaxUIntHWord}{{\bf\tt MAX\_UINT4}}
\newcommand{\MinUIntHWord}{{\bf\tt MIN\_UINT4}}

\newcommand{\Reg}[1]{{\bf\tt R({\em #1})}}
\newcommand{\Mem}[1]{{\bf\tt MEMORY(#1)}}

\newcommand{\tw}[1]{\tt {#1}}



\newtheorem{alghead}{Algorithm}[chapter]
\newtheorem{codehead}{Program}[chapter]

\makeindex

\setlength{\textheight}{8.0in}
\setlength{\textwidth}{6.5in}
\setlength{\oddsidemargin}{0.0in}
\setlength{\evensidemargin}{0.0in}
\raggedbottom

% Define some variables that control which version of Ant-8 is being
% documented...

\title{\Huge\bf {\sc Ant-8} \input{../CurrVersion} \\ Assembly Language Tutorial }

\begin{document}

\frontmatter
\maketitle

{\bf Copyright 1996-2002 by the President and Fellows of Harvard College. }

\tableofcontents

\chapter{Preface}

This document is a brief tutorial for {\sc Ant-8} assembly language
programming, and a description of the {\sc Ant-8} architecture.

{\sc Ant-8} is a small and simple {\sc RISC} architecture that was
created as a teaching tool for students in introductory computer
programming and computer architecture courses.  It is simple enough so
that the entire architecture can be learned and understood in a few
hours by a novice programmer and small enough so that the entire state
of the machine can be displayed on a single 24-by-80 text screen.  The
{\sc Ant-8} architecture is also realistic enough, however, to give
students a real understanding of computer architecture and how their
programs are executed.

The first chapter gives a tutorial for {\sc Ant-8} assembly language
programming.  After reading this chapter, students should be able to
write simple {\sc Ant-8} assembly language programs.

The second chapter gives an overview specification of the {\sc Ant-8}
architecture.  After reading this chapter, students should be
able to create their own implementation of the {\sc Ant-8} (as a
software simulation, or as hardware).  More advanced students should
also be able to write {\sc Ant-8} debugging tools or even their own
assembler for {\sc Ant-8} assembly language after reading this chapter.

\mainmatter

% 02/20/94
% $Id: tutorial.tex,v 1.18 2002/04/22 16:32:22 ellard Exp $
%

\chapter{An {\sc Ant-8} Tutorial}

This section is a tutorial for {\sc Ant-8} assembly language
programming and the {\sc Ant-8} environment.
This chapter covers the basics of {\sc Ant-8} assembly language,
including arithmetic operations, simple I/O, conditionals, loops,
and accessing memory.

\section{What is Assembly Language?}
\index{assembly}

Computer instructions are represented, in a computer, as sequences of
bits.  Generally, this is the lowest possible level of representation
for a program-- each instruction is equivalent to a single,
indivisible action of the CPU.  This representation is called {\em
machine language}, and it is the only form that can be ``understood''
directly by the computer.

A slightly higher-level representation (and one that
is much easier for humans to use) is called {\em assembly language}.
Assembly language is very closely related to machine language,   
and there is usually a straightforward way to translate
programs written in assembly language into machine language.
(This translation is usually implemented by a program called
an {\em assembler}.)
Assembly language is usually a direct translation of the
machine language; one instruction in assembly language
corresponds to one instruction in the machine language.

Because of the close relationship between machine and assembly
languages, each different machine architecture usually has its own assembly
language (in fact, a particular architecture may have several),
and each is unique.

\section{Getting Started with {\sc Ant-8} Assembly: {\tt add.asm}}
\index{add.asm}

To get our feet wet, we'll write an assembly language program named
{\tt add.asm} that computes the sum of 1 and 2.  Although this task is
simple, in order to accomplish it we will need to explore several key
concepts in {\sc Ant-8} assembly language programming.

\subsection{Registers}

Like many modern CPU architectures, the {\sc Ant-8} CPU can only operate
directly on data that is stored in special locations called {\em
registers}.  The {\sc Ant-8} architecture has 16 registers, named {\tt r0}
through {\tt r15}.  Each of these registers can hold a single value.
Two of these registers have special purposes: 
register zero ({\tt r0}) always contains the value zero, and register
one ({\tt r1}) is used to hold useful values computed as part of the
most recently executed instruction.

While most modern computers have many megabytes of memory, it is
unusual for a computer to have more than a few dozen registers.  Since
most computer programs use much more data than can fit into these
registers, it is usually necessary to juggle the data back and forth
between memory and the registers, where it can be operated upon by the
CPU.  (The first few programs that we write will only use registers,
but in section \ref{load-store-sec} the use of memory is introduced.)

\subsection{Commenting}
\index{commenting}

Before we start to write the executable statements of
our program, however, we'll need to write a comment that describes
what the program is supposed to do.  In the {\sc Ant-8} assembly language,
any text between a pound sign ({\tt \#}) and the subsequent newline
is considered to be a comment, and is ignored by the assembler.
Good comments are absolutely essential!
Assembly language programs are notoriously difficult to read
unless they are well organized and properly documented.

Therefore, we start by writing the following:

{\codesize
\begin{verbatim}
# Dan Ellard -- 11/2/96
# add.asm-- A program that computes the sum of 1 and 2,
#       leaving the result in register r2.
# Registers used:
# r2 - used to hold the result.

# end of add.asm
\end{verbatim}}

Even though this program doesn't actually do anything yet, at least
anyone reading our program will know what this program is supposed to do,
and who to blame if it doesn't work\footnote{
You should put your own name on your own programs, of course;
Dan Ellard shouldn't take all the blame.}.

Unlike programs written in higher level languages, it is usually
appropriate to comment every line of an assembly language program,
often with seemingly redundant comments.  Uncommented code that seems
obvious when you write it will be baffling a few hours later.  While a
well-written but uncommented program in a high level language might be
relatively easy to read and understand, even the most well-written
assembly code is unreadable without appropriate comments.  Some
programmers prefer to add comments that paraphrase the steps performed
by the assembly instructions in a higher-level language.

We are not finished commenting this program,
but we've done all that we can do until we know a little more about
how the program will actually work.

\subsection{Finding the Right Instructions}

Next, we need to figure out what instructions the computer will need
to execute in order to add two numbers.  Since the {\sc Ant-8} architecture
has very few instructions, it won't be long before you have memorized
all of the instructions that you'll need, but as you are getting
started you'll need to spend some time browsing through the list of
instructions, looking for ones that you can use to do what you want. 
A table of all of the instructions is given in figure
\ref{mnemonic-table}, on page \pageref{mnemonic-table}.

Luckily, as we look through the list of instructions, the very first
instruction we come across is the {\tt add} instruction, which adds
two numbers together.

The {\tt add} instruction takes three operands, which must appear
in the following order:

\begin{enumerate}

\item A register that will be used to hold the result of the addition. 
	For our program, this will be {\tt r2}.

\item A register that contains the first number to be added. 
	Therefore, we're going to have to place the value 1 into a
	register before we can use it as an operand of {\tt add}. 
	Checking the list of registers used by this program (which is
	an essential part of the commenting) we select {\tt r3}, and
	make note of this in the comments.

\item A register that holds the second number to be added.  We're also
	going to have to place the value 2 into a register before we
	can use it as an operand of {\tt add}.  Checking the list of
	registers used by this program we select {\tt r4}, and make
	note of this in the comments.

\end{enumerate}

We now know how we can add the numbers, but we have to figure out how
to place 1 and 2 into the appropriate registers.  To do this, we can
use the {\tt lc} (load constant value) instruction, which places an
8-bit constant into a register.  Therefore, we arrive at the following
sequence of instructions:

{\codesize
\begin{verbatim}
# Dan Ellard -- 11/2/96
# add.asm-- A program that computes the sum of 1 and 2,
#       leaving the result in register r2.
# Registers used:
# r2 - used to hold the result.
# r3 - used to hold the constant 1.
# r4 - used to hold the constant 2.

        lc      r3, 1           # r3 = 1
        lc      r4, 2           # r4 = 2
        add     r2, r3, r4      # r2 = r3 + r4.

# end of add.asm
\end{verbatim}}

\subsection{Completing the Program}

These three instructions perform the calculation that we want, but
they do not really form a complete program.  We have told the
processor what we want it to do, but we have not told it to stop
after it has done it!

{\sc Ant-8} programs always begin executing at the first instruction in
the program.  There is no rule for where the program ends, however,
and if not told otherwise the {\sc Ant-8} processor will read past the
end of the program, interpreting whatever it finds as instructions and
trying to execute them.  It might seem sensible (or obvious) that the
processor should stop executing when it reaches the ``end'' of the
program (in this case, the {\tt add} instruction on the last line), but
there are some situations where we might want the program to continue
past the ``end'' of the program, or stop before it reaches the end.
Therefore, the {\sc Ant-8} architecture contains an instruction named
{\tt hlt} that {\em halts} the processor.

The {\tt hlt} instruction does not take any operands.
(For more information about {\tt hlt}, consult Figure
\ref{mnemonic-table} on page \pageref{mnemonic-table}.)

\index{add.asm (complete listing)}
% $Id

\ThreeRegisterOp{add}{Addition}{\OPCODE}{des}{src1}{src2}

\index{add@{\tt add}!for Ant-8}

8-bit integer addition (with overflow).

\begin{enumerate}

\item Let {\em temp} $~ \leftarrow$  \Reg{src1} + \Reg{src2}.

	\Reg{src1} and \Reg{src2} are treated as
	8-bit 2's complement integers.

\item Detect whether overflow/underflow has taken place:

\begin{tabular}{lrl}

\Reg{1} $~ \leftarrow$ & 1  & if \MaxIntWord $< ~$ {\em temp} \\

\Reg{1} $~ \leftarrow$ & 0  &
	if \MinIntWord $\leq$ {\em temp} $\leq ~$ \MaxIntWord \\

\Reg{1} $~ \leftarrow$ & -1 & if {\em temp} $< ~$ \MinIntWord \\

\end{tabular}

\item \Reg{des} $\leftarrow$ the lower eight bits of {\em temp}.

\end{enumerate}



\subsection{The Format of {\sc Ant-8} Assembly Programs}

As you read {\tt add.asm}, you may notice several formatting
conventions-- every instruction is indented,
and each line contains at most one instruction.  These conventions are
{\em not} simply a matter of style, but are actually part of the
definition of the {\sc Ant-8} assembly language.

The first rule of {\sc Ant-8} assembly formatting is that instructions {\em
must} be indented.  Comments do not need to be indented, but all of
the code itself must be.  The second rule of {\sc Ant-8} assembly formatting
is that only one instruction can appear on each line.  (There are a
few additional rules, but these will not be important until section
\ref{Labels-subsec}.)

Unlike many programming languages, where the use of whitespace and
formatting is largely a matter of style, in {\sc Ant-8} assembly language some
use of whitespace is required.

\subsection{Running {\sc Ant-8} Assembly Language Programs}

At this point, we should have a working program.  Now, it's time to
run it and see what happens.

There are two principal ways of running an {\sc Ant-8} program-- using
the commandline tools ({\tt aa8}, {\tt ad8} and {\tt ant8}), or using
{\sc AIDE8} (the {\sc Ant-8} Integrated Development Environment).

\subsubsection{Using the Commandline Tools}

Before the commandline tools can run on a program, the program
must be written in a file.  This file must be plain text, and by
convention {\sc Ant-8} assembly language files have a suffix of {\tt .asm}.
In this example, we will assume that the file {\tt add.asm} contains
a copy of the {\tt add} program listed earlier.

Before we can run the program, we must {\em assemble} it.  The
assembler translates the program from the assembly language
representation to the machine language representation.  The assembler
for {\sc Ant-8} is called {\tt aa8}, so the appropriate command would be:

\begin{verbatim}
        aa8 add.asm
\end{verbatim}

This will create a file named {\tt add.ant} that contains
the {\sc Ant-8} machine-language representation of the program in
{\tt add.asm}.

Now that we have the assembled version of the program,
we can test it by loading it into the {\sc Ant-8} debugger in
order to execute it.
The name of the {\sc Ant-8} debugger is {\tt ad8}, so to run the debugger,
use the {\tt ad8} command followed
by the name of the machine language file to load.
For example, to run the program that we just wrote
and assembled:

\begin{verbatim}
        ad8 add.ant
\end{verbatim}

After starting, the debugger will display the following
prompt: {\tt >>}.
Whenever you see the {\tt >>} prompt, you know that the debugger
is waiting for you to specify a command for it to execute.

Once the program is loaded, you can use the {\tt r} (for {\em run})
command to run it:
\begin{verbatim}
        >> r
\end{verbatim}

The program runs, and then the debugger indicates that it is ready to
execute another command.  Since our program is supposed to
leave its result in register {\tt r2}, we can verify that the
program is working by asking the debugger to print out the contents of
the registers using the {\tt p} (for {\em print}) command,
to see if it contains the result we expect:

{\codesize
\begin{verbatim}
>> p 
 r01  r02  r03  r04  r05  r06  r07  r08  r09  r10  r11  r12  r13  r14  r15
  00   03   01   02   00   00   00   00   00   00   00   00   00   00   00
   0    3    1    2    0    0    0    0    0    0    0    0    0    0    0
\end{verbatim}}

The {\tt p} command displays the contents of each register.
The first line lists the register names.  The following line
lists the value of each register in hexadecimal, and the
last line lists the same number in decimal.

The {\tt q} command exits the debugger.

{\tt ad8} includes a number of features that will make debugging your
{\sc Ant-8} assembly language programs much easier.  Type {\tt h} (for {\em help})
at the
{\tt >>} prompt for a full list of the {\tt ad8} commands, or consult the
{\tt ad8} documentation for more information.

\subsubsection{Using {\sc AIDE8}}

{\sc AIDE8} provides a more tightly integrated way of creating,
debugging, and running {\sc Ant-8} programs.  Although it is less
flexible than running each of the tools individually, for most users
it is more than sufficient.

When {\sc AIDE8} starts, only the editor window is shown.  This window
can be used to create or edit an {\sc Ant-8} program.  After the program
is written, it can be assembled by pressing the {\bf Assemble} button. 
If the assembly process is successful, no error messages will be
displayed; otherwise, the cause of the error is printed at the bottom
of the screen and the offending line of the program is highlighted.

Once the program has been written and assembled, pressing the {\bf
Debug} button brings up the debugger window.  The debugger window
displays the complete state of the {\sc Ant-8} machine.  (The debug
window can be displayed at any time, even if there isn't a program
loaded into the {\sc Ant-8} machine, but without a program to display
there isn't much to see.)

To simply run the program, click on the {\bf Run} button in the
upper-left corner of the debugger window.  The execution of the
program, instruction by instruction, will be displayed in the debugger
window as the state of the processor is updated.

The {\sc AIDE8} contains many other features.  Consult the {\bf Help}
menu of {\sc AIDE8} for more information.


\section{Reading and Printing: {\tt add2.asm}}
\index{add2.asm}
\label{add2-sec}

Our program to compute $1 + 2$ is not particularly useful, although
it does demonstrate a number of important details about programming
in {\sc Ant-8} assembly language and the {\sc Ant-8} environment.  For our next
example, we'll write a program named {\tt add2.asm} that computes
the sum of two numbers specified by the user at runtime,
and displays the result on the screen.

The algorithm this program will use is:
\begin{enumerate}

\item   Read the two numbers from the user.

        We'll need two registers to hold these two numbers.  We will use
        {\tt r3} and {\tt r4} for this.

\item   Compute their sum.

        We'll need a register to hold the result of this addition.  We
        can use {\tt r2} for this.

\item   Print the sum, followed by a newline.

\item   Halt.

	We already know how to do this, using {\tt hlt}.

\end{enumerate}

The only parts of the algorithm that we don't know how to do yet are
to read the numbers from the user, and print out the sum. 

{\sc Ant-8} does its I/O (or ``input/output'') using the {\tt in}
instruction to read values from the user into the computer, and
the {\tt out} instruction to display values to the user.

The {\tt in} instruction allows the user to specify values in
one of three different formats:  hexadecimal, binary, or
{\sc ASCII}.  Similarly, the {\tt out} instruction can display a
value in hexadecimal, binary, or {\sc ASCII}.  Note that there is no
way to directly input or output a decimal value!

This gives the following program:

\index{add2.asm (complete listing)}
\input{Tutorial/add2}

\section{Branches, Jumps, and Conditional Execution: {\tt larger.asm}}
\index{branching}
\index{larger.asm}

The next program that we will write will read two numbers from the
user, and print out the larger of the two.  The basic structure of
this program is similar to the one used by {\tt add2.asm}, except
that we're computing the maximum rather than the sum of two numbers. 
The difference is that the behavior of this program depends upon the
input, which is unknown when the program is written.
The program must be able to
decide whether to execute instructions to print out the first number
or execute the instructions to print out the second number at runtime. 
This is known as {\em conditional execution}-- whether or not certain
parts of program are executed depends on a condition that is not known
when the program is written.

Browsing through the instruction set again, we find a description of
the {\sc Ant-8} branching instructions.  These allow the programmer to specify
that execution should {\em branch} (or {\em jump}) to a location other
than the next instruction.  These instructions allow conditional
execution to be implemented in assembly language (although in not
nearly as clean a manner as higher-level languages provide).

In {\sc Ant-8} assembler, there are three branching instructions:  {\tt
bgt}, {\tt beq} and {\tt jmp}.  {\tt bgt} and {\tt beq} are called
{\em conditional} branches, because they cause the program to branch
when a specific condition holds.  In contrast, the {\tt jmp}
instruction is an {\em unconditional} branch, which is always taken.

The {\tt bgt} instruction takes three registers as arguments.  If the
number in the second register is larger than the number in the third,
then execution will jump to the location specified by the first;
otherwise it continues at the next instruction.

The {\tt beq} instruction is similar in form to the {\tt bgt}
instruction, except that the branch occurs if the second and third
registers contain the same value.

The {\tt jmp} instruction takes a single argument, which is an
unsigned 8-bit constant.  Execution jumps to the location specified by
the constant.

\subsection{Labels}
\label{Labels-subsec}
\index{labels}

Keeping track of the numeric addresses in memory of the instructions
to which we want to branch is troublesome and tedious at best-- a small
error can make our program misbehave in strange ways, and if we change
the program at all by inserting or removing instructions, we will have
have to carefully recompute all of these addresses and then change all
of the instructions that use these addresses.  This is much more than
most humans can possibly keep track of.  Luckily, the computer is very
good at keeping track of details like this, so the {\sc Ant-8} assembler
provides {\em labels}, a human-readable shorthand for addresses.

A {\em label} is a symbolic name for an address in memory.  In {\sc Ant-8}
assembler, a {\em label definition} is an identifier followed by a colon.
{\sc Ant-8} identifiers use the same conventions as Python, Java, C, C++, and
many other contemporary languages:

\begin{itemize}

\item {\sc Ant-8} identifiers must begin with an underscore, an uppercase
	character (A-Z) or a lowercase character (a-z).

\item Following the first character there may be zero or more
	underscores, or uppercase, lowercase, or numeric (0-9)
	characters.  No other characters can appear in an identifier.

\item Although there is no intrinsic limit on the length of {\sc Ant-8}
	identifiers, some {\sc Ant-8} tools may reject identifiers longer than
	100 characters.

\end{itemize}


Labels must be the first item on a line, and must begin in the ``zero
column'' (immediately after the left margin).  Label definitions {\em
cannot} be indented, but all other non-comment lines {\em must} be.

Since labels must begin in column zero, only one label definition is
permitted on each line of assembly language, but a location in memory
may have more than one label.  Giving the same location in memory more
than one label can be very useful.  For example, the same location in
your program may represent the end of several nested ``if''
statements, so you may find it useful to give this instruction several
labels corresponding to each of the nested ``if'' statements.

When a label appears alone on a line, it refers to the following
memory location.  This is often good style, since it allows the use of
long, descriptive labels without disrupting the indentation of the
program.  It also leaves plenty of space on the line for the
programmer to write a comment describing what the label is used for,
which is very important since even relatively short assembly language
programs may have a large number of labels.

\subsection{Branching Using Labels}
\index{branch with labels }

Using the branching instructions and labels we can do what we want in
the {\tt larger.asm} program.  Since the branching instructions take a
register containing an address as their first argument, we need to
somehow load the address represented by the label into a register.  We
do this by using the {\tt lc} ({\em load constant}) command.  The {\tt
larger.asm} program illustrates how this is done.

\input{Tutorial/larger}

Note that since {\sc Ant-8} does not have an instruction to {\em copy} or {\em
move} the contents of one register to another, in order to copy the
value of one register to another register we've added 0 to one
register and put the sum in the destination register in order to
achieve the desired result.  (Recall that register {\tt r0} always
contains the constant zero.)

\section{Looping: {\tt loop.asm}}
\index{looping}
\index{loop.asm}
\label{loop-sec}

In the previous example program, we used the jump and branch
instructions to implement conditional execution, which meant that we
could skip over some instructions depending on the values that the
user typed.  We can also use these instructions to implement {\em
loops}, which allow the program to repeatedly execute a sequence of
instructions an arbitrary number of times.

The next program that we will write will read a character $A$ (as
{\sc ASCII}) and then a number $B$ (as hexadecimal) from the user, and then
print $B$ copies of the character $A$.  This algorithm translates
easily into {\sc Ant-8} assembler; the only thing that is new is that the
execution might jump ``backwards'' in the program to repeat some
instructions more than once.  The {\tt loop.asm} programs shows how
this is done.

\input{Tutorial/loop}


\section{Strings: {\tt hello.asm}}
\index{hello.asm}
\label{hello-sec}
\label{load-store-sec}

The next program that we will write is the ``Hello World'' program,
a program that simply prints the message ``Hello World'' to the screen
and then halts.

There is no way in {\sc Ant-8} to print out more than one character at a
time, so we must use a loop to print out each character of the string,
starting at the beginning and continuing until we reach the end of the
string.


The string ``{\tt Hello World}'' is not part of the instructions of
the program, but it is part of the memory used by the program.  The
assembler places all data values (not instructions) after all of the
instructions in memory.

The value of the data memory is loaded into memory at assembly time. 
You will have to be careful to not accidently overwrite your data
during run-time!

The way that the initial contents of data memory are defined is via
the {\tt .byte} directive.  {\tt .byte} looks like an instruction
which takes as many as eight 8-bit constants, but it is not an
instruction at all.  Instead, it is a directive to the assembler to
fill in the next available locations in memory with the given
values.

All of the {\tt .byte} items in an {\sc Ant-8} program must appear at
the end of the program, after the special label {\tt \_data\_}.  The
{\tt \_data\_} label indicates to the assembler that all subsequent
items are data.  No instructions are permitted after the {\tt \_data\_}
label.

In our programs, we will use the following convention for {\sc ASCII}
strings:  a {\em string} is a sequence of characters terminated by a 0
byte.  For example, the string ``hi'' would be represented by the
three characters `h', `i', and 0.  Using a 0 byte to mark the end
of the string is a convenient method, used by several contemporary
languages.

The program {\tt hello.asm} is an example of how to use labels and
treat characters in memory as strings:

\index{hello.asm (complete listing)}
\input{Tutorial/hello}

The label {\tt str\_data} is the symbolic representation of the
memory location where the string begins in data memory.

\section{Character I/O: {\tt echo.asm}}
\index{echo.asm}
\label{echo-sec}
\index{character I/O}

Now that we have mastered loops and reading and printing integers,
we'll turn our attention to reading and printing single characters. 
(We've already seen how to read and write numbers, and the process is
similar, except that we use the {\sc ASCII} device instead of {\tt
Hex}.)

The program we'll write in this section simply echos whatever you type
to it, until EOI ({\em end of input}) is reached.

The way that EOI is detected in {\sc Ant-8} is that when the EOI is
reached, any attempt to use {\tt in} to read more input will
immediately fail, and a non-zero value will be placed in register {\tt
r1} to indicate that there was an error.  If the {\tt in} succeeds,
then {\tt r1} is set to zero. 

Therefore, our program will loop, continually using {\tt in} to read
characters, and checking {\tt r1} after each {\tt in} to see whether
or not the EOI has been reached.

\input{Tutorial/echo}

{\bf Note:} because of the difference between the user interface of
the debugger in {\sc AIDE8} and the ordinary runtime {\sc Ant-8}
environment, {\tt echo.asm} behaves differently when run under {\sc
AIDE8} than when run via {\tt ant8} or {\tt ad8}.  This is because in
{\sc AIDE8}, every input, including {\sc ASCII} input, must be followed
by a newline, while in ordinary operation {\sc ASCII} input does not. 
This should be considered a shortcoming of {\sc AIDE8}, not a problem
with {\tt echo.asm}.

\section{Bit Operations: {\tt shout.asm}}

The next program we shall write is very similar to {\tt echo.asm},
except that instead of simply echoing its input, it converts all of
the lowercase characters in the input to uppercase in the output.

There are several ways that this could be accomplished.  The easiest
way, and one that does not require learning about any new {\sc Ant-8}
instructions, would be to note that all of the lowercase characters in
{\sc ASCII} are arranged consecutively, starting with {\tt 'a'} (with
a value of {\tt 0x61}) and continuing through {\tt 'z'} (with a value
of {\tt 0x7A}).  The uppercase characters are arranged in the same
manner, ranging from {\tt 'A'} = {\tt 0x41} to {\tt 'Z'} = {\tt 0x5A}. 
Therefore, in order to convert from uppercase to lowercase all we need
to do is take any characters that have {\sc ASCII} codes in the range
{\tt 0x61} to {\tt 0x7A} and subtract {\tt 0x20} from them before
printing them.

However, our goal in this section is to learn about {\sc Ant-8}'s
bitwise instructions, and so we use a different observation-- the {\tt
ASCII} code for the lowercase characters and the corresponding
uppercase characters differ only in a single bit.  All the lowercase
characters have the bit corresponding to {\tt 0x20} set to 1, while
all the uppercase characters do not.  Therefore, if we have a
lowercase character, in order to convert it to uppercase all we need to
do is set this bit to 0.  The bit we are interested in is the fifth
bit (counting from the right and starting at zero).

To change the fifth bit from 1 to 0, we can use the {\tt and}
instruction.  If the 8-bit value is in register {\tt r2}, then
computing the bitwise {\sc And} of the value of {\tt r2} and the 8-bit
value value consisting of all 1 bits {\em except} the fifth bit, the
result will be identical to the original value of {\tt r2} except that
the fifth bit will be 0.

We could explicitly compute the value of the 8-bit value that has all
1-bits except for the fifth bit, or we can let the computer do this
work for us.  We choose the latter approach, because this also means
that we can introduce two more instructions, {\tt shf} and {\tt nor}.

Our first task is to initialize {\tt r8} with the 8-bit value
consisting of all zero bits, except for the fifth bit.  (We know this
value is {\tt 0x20}, and so we could simply use this value, but we'll
use {\tt shf} instead:

\begin{verbatim}
        lc      r8, 1           # r8 has all 0 bits, except
                                # bit zero which is 1.
        lc      r9, 5
        shf     r8, r8, r9      # shift the 1 bit over 5 spaces.
                                # r8 is now 0x20.

        nor     r8, r8, r8      # NOR r8 with itself, which changes  
                                # all the 1 bits to 0 and vice versa.
                                # r8 is now 0xDF.
\end{verbatim}

The complete program is implemented in {\tt shout.asm}.

\input{Tutorial/shout}


%%% end of tutorial.tex



% \chapter{Functions}
% % $Id: func.tex,v 1.8 2002/04/17 20:05:59 ellard Exp $

\documentclass[makeidx,psfig]{article}
\usepackage{ifthen}
\usepackage{makeidx}
\usepackage{psfig}

% % macros.tex
%
% Definitions of commands extending Latex for the CS51 Project Book.
%
% Bob Walton
% Dec 7, 1993

\setcounter{secnumdepth}{5}
\setcounter{tocdepth}{5}

% \AND			\wedge, the math AND operator
% \OR			\vee, the math OR operator
% \NOT			\sim, the math NOT operator
% \IMPLIES		\Rightarrow, the math IMPLIES operator
% \EQUIV		\equiv, the math EQUIVALENT-TO operator
%
% \contributors{<contributor>, <contributor>, ...}
%			Specifies comma separated list of contributors to
%			be acknowledged in the preface.
%
% \begin{history}{<filename>}
% <history>		Displays <history> as a quote if \makehistory
% \end{history}		has been included in the input.  Histories are
%			used to indicate author, date, and changes to
%			project documentation files.  Histories are not
%			printed in the copy of the project book received
%			by students.
%
%			The <filename> is the name of the file whose
%			history is being given.  It is printed at the
%			beginning of the history.  The extension is
%			assumed to be .tex, and is not included in
%			<filename>.
%
%			\end{history} must appear literally without
%			any spaces just like \end{verbatim}.
%			
% \EOL			Indicates a spot in a word where a line break
%			is OK.  E.g. {\tt car-\EOL stream}.
%
% \makehistory		Causes histories to be displayed.
%
% \inputfigure{<file>}{<height>}
%			Inputs a postscript figure from the indicated
%			file and makes it the indicated height.
%
%			Extra vertical space will be put before \inputfigure
%			only if a blank line proceeds it.  Extra vertical
%			space will be put after \inputfigure only if
%			a blank line follows it.
%
% \cbox{<code>} 	Displays <code> as code names or phrases
%			embedded in text.  Similar to \verb but
%			always terminates with }.
%
%			Restriction: \cbox cannot appear in any macro
%			argument.
%
% \$<code>$		Same as \cbox{<code>} except <code> is terminated
%			by a $.  Cannot appear in any macro argument.
%
% \dollar		The $ character.  Same as \$ in normal Latex,
%			(we redefine \$ above to allow \$<code>$).
%
% \cindex{<translated-code>}
%			Similar to index, and in fact equivalent to:
%
%			  \index{<translated-code>@\tt <translated-code>}
%
%			where <translated-code> is <code> with the special
%			characters translated for latex: namely
%				``\'' is denoted by ``$\backslash$''
%				``^'' is denoted by ``\^{~}''
%				``~'' is denoted by ``$\sim$''
%				``$'' is denoted by ``\dollar''
%				``#'' is denoted by ``\#''
%				``%'' is denoted by ``\%''
%				``&'' is denoted by ``\&''
%				``_'' is denoted by ``\_''
%				``{'' is denoted by ``\{''
%				``}'' is denoted by ``\}''
%
% \ckey{<code>}		Same as \cbox but puts <code> in index as per
%			\index.
%
%			Warning: because of limitations of latex, <code>
%			is communicated to the index as {\tt <code>},
%			and therefore <code> cannot contain special
%			characters.  If you need special characters,
%			use \ickey.
%
% \ickey{<code>}{<translated-code>}
%			Must be used in place of \ckey when <code>
%			contains characters that cannot be included
%			in the scope of \tt.
%
%			Same as \ckey but calls \cindex with
%			<translated-code> instead of <code>.
%
% \key{<text>}		Applies \em or \bf to <text> and also puts <text>
%			in the index.
%
% \ikey{<text>}{<itext>}
%			Applies \em or \bf to the first argument <text>,
%			and puts the second argument <itext> in the index.
%
% \begin{code}{<name>}	Displays <code>.  Similar to \begin{verbatim}
% <code>		but indents or centers <code>.
% \end{code}
%			\end{code} must appear literally without
%			any spaces just like \end{verbatim}.
%
%			<name> is ignored during text processing.  It is
%			used to extract the <code> from the file so it can
%			be tested.
%
%			Extra vertical space will be put before \begin{code}
%			only if a blank line proceeds it.  Extra vertical
%			space will be put after \end{code} only if
%			a blank line follows it.
%
% \begin{code*}{<name>}	Ditto but displays spaces as square cups, like
% <code>		verbatim*.
% \end{code*}		WARNING: this macro has NEVER WORKED.
%
% \begin{labpar}{<label>}
%			Begin labeled paragraph, which is an indented
%			paragraph with <label> in the left margin.
% \end{labpar}		End labeled paragraph.
%
% \begin{indentpar}	Begin indented paragraph ({list} with no label).
% \end{indentpar}	End indented paragraph.
%
% \spc			A temporary length command usable in conjunction
%			with \settowidth.
%
% \begin{problems}	Begin a list of problems.
% \problem		Start the next problem.
% \end{problems}	End a list of problems.
%
% \begin{moreproblems}	Continue a list of problems.
%			(Just like problems environment,
%			 but does not reset problem number counter.)
% \end{moreproblems}	End a continued list of problems.
%
% \begin{subproblems}	Begin a list of subproblems.
% \subproblem		Start the next subproblem.
% \end{subproblems}	End a list of subproblems.
%
% \begin{boxpicture}{<x-size>}{<y-size>}
% <picture commands>	Like \begin{picture} put sets the size of one
% \end{boxpicture}	unit so that 100 units is the side of a square
%			box suitable from making dotted pair diagrams.
%			<x-size> and <y-size> are the sizes of the
%			picture in these units.  (0,0) is the lower
%			left corner, (<x-size>-1,<y-size>-1) is the
%			upper right corner.
%
% \cons{<x>}{<y>}	Draws a CONS cell consisting of two side-by-side
%			boxes.  The Lower left coordinate of the left
%			box is  (<x>,<y>), while the lower left coordinate
%			of the right box is (<x>+100,<y>).
%
% \nil{<x>}{<y>}	Draws the lower left to upper right slant line
%			that indicates a box points at NIL.  <x>,<y> are
%			the lower left coordinates of the slant line.
%
% \sym{<x>}{<y>}{<name>}
%			Draws a symbol, which is just the symbol <name>,
%			a piece of text, centered in the box whose lower
%			left coordinates are <x>, <y>.
%
% \lab{<x>}{<y>}{<pos>>}{<text>}
%			Ditto, but positions the <text>:
%			    Against the left side of the box if <pos> = l.
%			    Against the right side of the box if <pos> = r.
%			    Against the top of the box if <pos> = t.
%			    Against the bottom of the box if <pos> = b.
%
%			<pos> can consist of two letters, one for
%			horizontal control and one for vertical control.
%			The default is to center.
%			
%
% \cpointer{<x1>}{<y1>}{<x2>}{<y2>}
%			Draws a pointer from a CONS cell box (either left
%			(car) or right (cdr)) to another box.  (<x1>,<y1>)
%			are the coordinates of the CONS cell box (its
%			lower leftmost point, actually), and (<y1>,<y2>)
%			are the coordinates of the target box (also
%			its lower leftmost point).  The pointer actually
%			goes from the middle of the CONS box toward the
%			middle of the target box, but stops at the
%			boundary of the target box.
%
%			Restriction: pointers must be horizontal pointing
%			right; vertical pointing down; or at 45 degree
%			angles pointing either down left of down right.
%
% \spointer{<x1>}{<y1>}{<x2>}{<y2>}
%			Ditto but the pointer, instead of starting from
%			the middle of the box, starts from the edge of the
%			box.  Used to point from a symbol to any box.
%
% \sline{<x1>}{<y1>}{<x2>}{<y2>}
%			Ditto but a line is drawn instead of a pointer
%			(i.e. there is no arrowhead).
%

\newcommand{\AND}{\wedge}
\newcommand{\OR}{\vee}
\newcommand{\NOT}{\sim}
\newcommand{\IMPLIES}{\Rightarrow}
\newcommand{\EQUIV}{\equiv}

\newcommand{\EOL}{\penalty \exhyphenpenalty}

\newcount\ATCATCODE
\ATCATCODE=\catcode`@

\catcode `@=11	% @ is now a letter

% The following are alterations of latex.tex macros.

% Variation on @ifnextchar:
\long\def\ifnexttoken#1#2#3{\let\@tempe #1\def\@tempa{#2}\def\@tempb{#3}%
	\futurelet\@tempc\@ifnch}

\def\inputfigure#1#2{\ifvmode \vspace{2ex}\else \par\fi
		    \centerline{\psfig{figure=#1,height=#2}}
		    \ifnexttoken\par{\vspace{2ex}}{}}

\begingroup \catcode `|=0 \catcode `[= 1
\catcode`]=2 \catcode `\{=12 \catcode `\}=12
\catcode`\\=12
|gdef|@xcode#1\end{code}%
	[#1|end[code]|ifnexttoken|par[|vspace[2ex]][]]
|gdef|@sxcode#1\end{code*}%
	[#1|end[code*]|ifnexttoken|par[|vspace[2ex]][]]
|long|gdef|@xhistory#1\end{history}[|end[history]]
|endgroup

\newlength{\codewidth}

\def\code #1{\ifvmode \vspace{2ex}\else \par\fi % must be before setlength
	     \setlength{\codewidth}{\linewidth}%
	     \addtolength{\codewidth}{-\parindent}%
	     \begin{minipage}[t]{\codewidth}%\small
	     \@verbatim \frenchspacing\@vobeyspaces \@xcode}
\def\endcode{\endtrivlist\if@endpe\@doendpe\fi\end{minipage}}

\@namedef{code*} #1{%
	     \ifvmode \vspace{2ex}\else \par\fi % must be before setlength
	     \setlength{\codewidth}{\linewidth}%
	     \addtolength{\codewidth}{-\parindent}%
	     \begin{minipage}[t]{\codewidth}%\small
	     \@verbatim \@xscode}
\@namedef{endcode*}{\endtrivlist\if@endpe\@doendpe\fi\end{minipage}}

\let\dollar\$
% \begingroup \catcode `|=0 \catcode `[= 1
% \catcode`]=2 \catcode `\{=12 \catcode `\}=12 \catcode `$=12
% \catcode`\\=12
% |gdef|cbox #1[|verb }#1}]
% |gdef|$[|verb $]
% |endgroup

\gdef\cindex #1{\index{#1@{\tt #1}}}
\gdef\ckey #1{\cbox{#1}\index{#1@{\tt #1}}}
\gdef\ickey #1#2{\cbox{#1}\index{#2@{\tt #2}}}

\gdef\key #1{{\em #1}\index{#1}}
\gdef\ikey #1#2{{\em #1}\index{#2}}

\def\history#1{\begingroup \@noligs \let\do\@makeother \dospecials \@xhistory}
\def\endhistory{\endgroup}
\def\makehistory{\def\history##1{\begin{quote} \small {\bf ##1}:}
		 \def\endhistory{\end{quote}}}

\def\contributors #1{\def\contributorlist{#1}}

% End of altered latex.tex macros.

\catcode `@=\ATCATCODE	% @ is now restored

\newenvironment{labpar}[1]%
	{\begin{list}{#1}{\settowidth{\leftmargin}{#1}%
			  \addtolength{\leftmargin}{\labelsep}%
			  \settowidth{\labelwidth}{#1}%
			  \setlength{\partopsep}{\parskip}%
			  \setlength{\parskip}{0in}%
			  \setlength{\topsep}{0in}%
			  \item}}%
	{\end{list}}

\newenvironment{indentpar}%
	{\begin{list}{}{}\item}%
	{\end{list}}

\newcounter{pnumber}
\newenvironment{problems}%
	{\begin{list}{\arabic{pnumber}.}{\usecounter{pnumber}}}%
	{\end{list}}
\newcommand{\problem}{\item}

\newcounter{mpnumber}
\newenvironment{moreproblems}%
	{\setcounter{mpnumber}{\value{pnumber}}%
	 \begin{list}{\arabic{pnumber}.}{\usecounter{pnumber}}%
	 \setcounter{pnumber}{\value{mpnumber}}}%
	{\end{list}}

\newcounter{spnumber}
\newenvironment{subproblems}%
	{\begin{list}{(\alph{spnumber})}{\usecounter{spnumber}}}%
	{\end{list}}
\newcommand{\subproblem}{\item}

\newlength{\spc}

\newcount\TCOUNTX
\newcount\TCOUNTY
\newcount\TCOUNTZ

\newenvironment{boxpicture}[2]%
	{\setlength{\unitlength}{0.002in}
	 \begin{picture}(#1,#2)
	}%
        {\end{picture}}

\newcommand{\cons}[2]
	{\put(#1,#2){\begin{picture}(200,100)
		     \put(0,0){\framebox(200,100){}}
		     \put(100,0){\line(0,1){100}}
		     \end{picture}
		    }
	}

\newcommand{\sym}[3]
	{\put(#1,#2){\makebox(100,100){\tt #3}}
        }

\newcommand{\lab}[4]
	{\put(#1,#2){\makebox(100,100)[#3]{\tt #4}}
        }

\newcommand{\nil}[2]
        {\put(#1,#2){\line(1,1){100}}
        }


\newcommand{\cpointer}[4]
       	{\TCOUNTX=#1
	 \advance\TCOUNTX by 50
	 \TCOUNTY=#2
	 \advance\TCOUNTY by 50

	 \ifnum#2=#4 {\TCOUNTZ=#3
		      \advance\TCOUNTZ by -\TCOUNTX
		      \put(\number\TCOUNTX,\number\TCOUNTY)
				{\vector(1,0){\TCOUNTZ}}
		     }
	\else {
		\ifnum#1<#3 {\TCOUNTZ=#3
		     	     \advance\TCOUNTZ by -\TCOUNTX
		     	     \put(\number\TCOUNTX,\number\TCOUNTY)
					{\vector(1,-1){\TCOUNTZ}}
		    	    } 
		\fi
		\ifnum#1=#3 {\TCOUNTZ=#4
			     \advance\TCOUNTZ by 100
			     \multiply\TCOUNTZ by -1
			     \advance\TCOUNTZ by \TCOUNTY
		     	     \put(\number\TCOUNTX,\number\TCOUNTY)
					{\vector(0,-1){\TCOUNTZ}}
		    	    } 
		\fi
		\ifnum#1>#3 {\TCOUNTZ=#3
			     \advance\TCOUNTZ by 100
			     \multiply\TCOUNTZ by -1
		     	     \advance\TCOUNTZ by \TCOUNTX
		     	     \put(\number\TCOUNTX,\number\TCOUNTY)
					{\vector(-1,-1){\TCOUNTZ}}
		    	    } 
		\fi
	      }

	\fi
        }

\newcommand{\spointer}[4]
       	{\TCOUNTX=#1
	 \advance\TCOUNTX by 50
	 \TCOUNTY=#2
	 \advance\TCOUNTY by 50

	 \ifnum#2=#4 {\TCOUNTZ=#3
		      \advance\TCOUNTX by 50
		      \advance\TCOUNTZ by -\TCOUNTX
		      \put(\number\TCOUNTX,\number\TCOUNTY)
				{\vector(1,0){\TCOUNTZ}}
		     }
	\else {
		\ifnum#1<#3 {\TCOUNTZ=#3
			     \advance\TCOUNTX by 50
			     \advance\TCOUNTY by -50
		     	     \advance\TCOUNTZ by -\TCOUNTX
		     	     \put(\number\TCOUNTX,\number\TCOUNTY)
					{\vector(1,-1){\TCOUNTZ}}
		    	    } 
		\fi
		\ifnum#1=#3 {\TCOUNTZ=#4
			     \advance\TCOUNTY by -50
			     \advance\TCOUNTZ by 100
			     \multiply\TCOUNTZ by -1
			     \advance\TCOUNTZ by \TCOUNTY
		     	     \put(\number\TCOUNTX,\number\TCOUNTY)
					{\vector(0,-1){\TCOUNTZ}}
		    	    } 
		\fi
		\ifnum#1>#3 {\TCOUNTZ=#3
			     \advance\TCOUNTX by -50
			     \advance\TCOUNTY by -50
			     \advance\TCOUNTZ by 100
			     \multiply\TCOUNTZ by -1
		     	     \advance\TCOUNTZ by \TCOUNTX
		     	     \put(\number\TCOUNTX,\number\TCOUNTY)
					{\vector(-1,-1){\TCOUNTZ}}
		    	    } 
		\fi
	      }

	\fi
        }

\newcommand{\sline}[4]
       	{\TCOUNTX=#1
	 \advance\TCOUNTX by 50
	 \TCOUNTY=#2
	 \advance\TCOUNTY by 50

	 \ifnum#2=#4 {\TCOUNTZ=#3
		      \advance\TCOUNTX by 50
		      \advance\TCOUNTZ by -\TCOUNTX
		      \put(\number\TCOUNTX,\number\TCOUNTY)
				{\line(1,0){\TCOUNTZ}}
		     }
	\else {
		\ifnum#1<#3 {\TCOUNTZ=#3
			     \advance\TCOUNTX by 50
			     \advance\TCOUNTY by -50
		     	     \advance\TCOUNTZ by -\TCOUNTX
		     	     \put(\number\TCOUNTX,\number\TCOUNTY)
					{\line(1,-1){\TCOUNTZ}}
		    	    } 
		\fi
		\ifnum#1=#3 {\TCOUNTZ=#4
			     \advance\TCOUNTY by -50
			     \advance\TCOUNTZ by 100
			     \multiply\TCOUNTZ by -1
			     \advance\TCOUNTZ by \TCOUNTY
		     	     \put(\number\TCOUNTX,\number\TCOUNTY)
					{\line(0,-1){\TCOUNTZ}}
		    	    } 
		\fi
		\ifnum#1>#3 {\TCOUNTZ=#3
			     \advance\TCOUNTX by -50
			     \advance\TCOUNTY by -50
			     \advance\TCOUNTZ by 100
			     \multiply\TCOUNTZ by -1
		     	     \advance\TCOUNTZ by \TCOUNTX
		     	     \put(\number\TCOUNTX,\number\TCOUNTY)
					{\line(-1,-1){\TCOUNTZ}}
		    	    } 
		\fi
	      }

	\fi
        }

%%% 02/26/94
%%% An attempt at writing my own macros...

\newcommand{\codesize}{\small}

%%% algorithm environment:
%%%     Arg 1 is some description of the algorithm.
%%%     Arg 2 is the reference name of the algorithm.
\newenvironment{algorithm}[2]{
        \begin{figure}[hbtp]
        \hrule
	\vspace{2mm}
        \begin{alghead}
        \label{#2}
        {\em #1}
        \end{alghead}
        % \hrule
}{
        \vspace{3mm}
        \hrule
        \end{figure}
}

%%% dan-code environment:
%%%     Arg 1 is the name of the program or code fragment.
%%%     Arg 2 is the reference name of the code.
%%%     Arg 3 is any additional text to add to the preface of the code.
\newenvironment{dan-code}[3]{
        \newpage
        \section{{\tt #1}}
        \label{#2}
        \index{#1 (complete listing)}
        #3
        \vspace{3mm}
        \hrule
        \vspace{3mm}
        \footnotesize
}{
        \normalsize
        \vspace{3mm}
        \hrule
}

\newcommand{\marginnote}[1]{\marginpar{{\bf {\footnotesize #1}}}}

%% The variables must be set before any of the following macros
%% can be used.  They should be assigned in the root of the document,
%% not here, however.
%%
%% UseDefaultAntOpcodes - True if the Ant uses the default opcodes
%%	(whatever they happen to be at any given instant...) instead
%%	of some variation.  Currently, the only variation is for QRR,
%%	which omits sys and adds halt and in/out.
%%
%% AllowUnwriteableDes - The current ANT faults if the program tries
%%	to write r0 or r1.  If this bool is true, then the processor
%%	does not fault.  I just silently ignores the write, leaving
%%	r0 with zero (always) and r1 with whatever it would have been
%%	had the destination register been something else...
%%
%%	As if 1/19, the default is to fault.  However, this is very
%%	likely to change in the future.

\newboolean{UseDefaultAntOpcodes}
\newboolean{AllowUnwriteableDes}

%%% end of dan.tx

% $Id: ant-macros.tex,v 1.7 2002/01/02 02:13:47 ellard Exp $

\newcommand{\ThreeRegisterOp}[6]{
	\vspace{0.1in}
	\begin{tabular}{p{0.6in}p{2.4in}|p{0.6in}|p{0.6in}|p{0.6in}|p{0.6in}|}
	\cline{3-6}
	\fbox{\tt\Large #1} & {\bf #2} &
			#3 & {\em #4} & {\em #5} & {\em #6} \\
	\cline{3-6}
	\end{tabular}
	\vspace{0.1in}
}

% TwoRegisterOp is actually exactly the same as ThreeRegisterOp right now.
% It exists only to clarify the spec (and because perhaps in the future
% we will distinguish them).

\newcommand{\TwoRegisterOp}[6]{\ThreeRegisterOp{#1}{#2}{#3}{#4}{#5}{#6}}

\newcommand{\OneRegisterOp}[5]{
	\vspace{0.1in}
	\begin{tabular}{p{0.6in}p{2.4in}|p{0.6in}|p{0.6in}|p{1.4in}|}
	\cline{3-5}
	\fbox{\tt\Large #1}	& {\bf #2} &
			#3 & {\em #4} & {\em #5} \\
	\cline{3-5}
	\end{tabular}
	\vspace{0.1in}
}

\newcommand{\MaxIntWord}{{\bf\tt MAX\_INT8}}
\newcommand{\MinIntWord}{{\bf\tt MIN\_INT8}}
\newcommand{\MaxUIntWord}{{\bf\tt MAX\_UINT8}}
\newcommand{\MinUIntWord}{{\bf\tt MIN\_UINT8}}
\newcommand{\MaxUIntHWord}{{\bf\tt MAX\_UINT4}}
\newcommand{\MinUIntHWord}{{\bf\tt MIN\_UINT4}}

\newcommand{\Reg}[1]{{\bf\tt R({\em #1})}}
\newcommand{\Mem}[1]{{\bf\tt MEMORY(#1)}}

\newcommand{\tw}[1]{\tt {#1}}



\newtheorem{alghead}{Algorithm}[section]
\newtheorem{codehead}{Program}[section]

\makeindex

\setlength{\textheight}{8.0in}
\setlength{\textwidth}{6.5in}
\setlength{\oddsidemargin}{0.0in}
\setlength{\evensidemargin}{0.0in}
\raggedbottom

\title{Functions in {\sc Ant-8}}

\begin{document}

In this document we will see how to build modular code and how
functions and methods in higher-level languages can be written in {\sc
Ant-8}.

\section{Avoiding Repeated Code}

Recall the program {\tt hello.asm} (described in Section \ref{hello}). 
Now imagine that you had a program that printed two strings.  We could
write such a program by making a copy of most of the {\tt hello.asm}
program so that there were two loops, one for printing the first
string, and the other for printing the second, but this seems a little
wasteful.  Now imagine that we wanted to print three or four different
strings-- we could make three or four copies of this code, with slight
changed to print each string, but this would soon become ridiculous.

\begin{figure}
\caption{Source code of {\tt print-2.asm}}
\hrule
\input{Tutorial/print-2}
\hrule
\end{figure}

What we would like to do instead is discover how we can write code so
that similar functionality is implemented once, instead of many times.

In this example, one way we could do this would be to make an extra
loop, outside the printing loop, that printed each string in turn. 
This approach would work for this example, but it does not work in
general because it assumes that all we are doing is printing strings. 
In the more general case, there might be a lot of other things that
are done by the program, and they might be different from loop to
loop.

What we would like is for the printing loop to be something we can use
any time we need to print a string, no matter what else the program
does.  In order to accomplish this, we need to learn some new
techniques.

\begin{figure}
\caption{Source code of {\tt print-3.asm}}
\hrule
\input{Tutorial/print-3}
\hrule
\end{figure}

\subsection{The Return Address}

The most important thing to note about this goal is how execution gets
back to where it was before it jumped to the printing code.  This is
accomplished by noting that after a {\tt jmp} instruction is executed,
register {\tt r1} contains the address of the instruction that would
have been executed had the been an ordinary instruction (not a branch
or a jump).  This means that after the program jumps to the code for
printing out the string, register {\tt r1} contains the address that
we want to branch back to after the printing is finished.  This is
called the {\em return address}.  It is important to grab the return
address out of register {\tt r1} immediately, since many instructions
change {\tt r1}.

In this code, we grab the return address and store it in register {\tt
r5}.  When we the code is finished printing the string, it branches
back to this address by using the {\tt beq} instruction.

\subsection{Saving and Restoring State}

The code in {\tt print-3} almost accomplishes our goal, but it has an
important flaw-- it changes the contents of registers {\tt r2} through
{\tt r5}.  Therefore, we can't just use this code wherever we like,
because when the function returns the values of these registers may be
changed.  Note that register {\tt r1} will also be changed by this
piece of code, but since many operations change {\tt r1} anyway, we
don't care so much about this.

One solution to this problem would be to rest of our program so that
it didn't use these registers for anything (as is true in {\tt
print-3}), but this approach quickly becomes unworkable for programs
that have more than a few very simple functions-- there simply aren't
enough registers.

A much more general solution is to {\em preserve} the contents of the
registers used by the function (in this case, registers {\tt r2}
through {\tt r4}) by storing them to memory whenever the function is
called, and then {\em restore} them by loading their values back into
these registers just before the function jumps back to the return
address.  To do this, all we need to do is set aside a small amount of
memory to store the contents of these registers in.  In this case,
the body of the {\tt print\_str} function uses registers {\tt r2} - {\tt r4},
so we need three bytes of memory to store these values.

This solution still isn't perfect, however, because we need to use one
register to load the address of {\tt print\_str\_mem} into!  Since we
use this register to compute where the registers we're saving are
stored in memory, it is overwritten before it can be saved. 
Therefore, not {\em all} registers can be preserved and restored using
this scheme.

We have the same problem with storing the return address.  This value
must be left in a register, so the program can do the {\tt beq} to
return from the function.  We could use another register for this, but
it turns out that this is unnecessary.  By being careful and storing
the return address in memory, along with the values of registers {\tt
r2} through {\tt r4}, we can get away with using just one ``scratch''
register, register {\tt r15}.

In program {\tt print-4}, the four bytes of memory after the {\tt
print\_str\_mem} are used to store the preserved values (the return
address, and the contents of registers {\tt r2} - {\tt r4}).

\begin{figure}
\caption{The source code of {\tt print-4.asm}}
\hrule
\input{Tutorial/print-4}
\hrule
\end{figure}

\section{Recursive Functions}

The method of preserving the values of the registers used by a
function in explicit memory locations, as done in {\tt print-4} has
serious drawbacks.  First, it requires that memory be set aside to
store the registers for each function, and this memory is always set
aside for this purpose, even when none of the functions are being
called.

More importantly, however, it cannot be used to implement {\em
recursive} functions.  A recursive function is a function that calls
itself (either indirectly or directly).

For this section, the recursive function we will use is the function
for computing the $n$'th Fibonacci number.  The sequence of Fibonacci 
numbers is defined as:

\begin{itemize}

	\item	Fib(0) = 1
	\item	Fib(1) = 1
	\item	Fib(n) = Fib(n-1) + Fib(n-2) if $n > 1$.

\end{itemize}

\subsection{Using Memory as a Stack}

Instead of setting aside a specific area memory for each function, we
will set aside a pool of memory and use it as temporary storage for
all of the functions.  At any given moment, we will keep track of what
part of the memory is being used, and what parts are unused.

Keeping track of what parts are used and unused would seem like a
tedious and difficult exercise, but it is not, thanks to a key
observation about the way that functions execute-- if function A calls
function B, then function B must return before function A.  Therefore,
we can simply organize our pool of memory as an array.  When we call
function A, we can set aside as much of the array as A needs, starting
at the beginning of the array.  When function B is called, we can put
its temporary storage immediately after the storage from A.  All we
really need to keep track of is how much of the array is in use at any
time.

This data structure, where temporary function records are stacked on
top of each other, is called a {\em stack}.  The common operations on
a stack are to {\em push} a value, which means to add it to the end of
the stack, and to {\em pop} a value, which means to remove it from the
end.

\subsection{General Function Linkage}

\subsubsection{Calling a Function}

Before jumping or branching to a function, ...

\subsubsection{Function Preamble}

\begin{enumerate}

\item {\bf Save the return address in register {\tt r4}.}

	The {\tt jmp} or branch instruction that invokes the function
	saves the return address in register {\tt r1}.  Many
	instructions overwrite register {\tt r1}, so we must extract
	the return address from {\tt r1} before executing any of them,
	or else the return address will be lost.  To simplify things,
	we might as well do this immediately.

	By convention, we temporarily save the return address in
	register {\tt r4}.


\item {\bf Preserve the return address.}

	Store the return address onto the stack.  Register {\tt r2}
	is used as the stack pointer.

\item {\bf Preserve the registers.}

	Store each of the registers that we want to restore later onto
	the stack.

\item {\bf Increment the stack pointer.}

	Move the stack pointer up, so that if any other functions
	are called they start with the stack in the right place.

\end{enumerate}

\subsubsection{Returning From a Function}

\begin{enumerate}

\item	{\bf Put the return value (if any) into register {\tt r3}.}

	If the function returns a value, by convention the caller will
	expect to find it in register {\tt r3}.

\item	{\bf Decrement the stack pointer.}

	Move the stack pointer back to its previous position,
	deallocating the current stack frame.

\item	{\bf Restore the return address.}

	By convention, we load the return address into register {\tt
	r4}.

\item	{\bf Restore the registers.}

	Load the contents of each of the preserved registers back into
	the registers, from the stack.  For each {\tt st1} instruction
	in the function preamble, there must be a corresponding {\tt
	ld1}.

\item	{\bf Branch to the return address.}

	Using {\tt beq}, branch back to the return address, which
	by convention was stored in {\tt r4}.

\end{enumerate}

\subsection{Optimized Function Linkage}

The general method of building stack frames described in the previous
section is always correct, but frequently it is far from optimal.  For
example, in the {\tt fib} function, we always save all of the
registers that {\em might} potentially be used by the function, even
though many of these registers are not used.  The base case of the
recursion occurs in more than half of the calls to {\tt fib}, so more
than half of the time this is wasted effort.  It would be more
efficient to treat the base case separately from the recursive case,
and only do all the work of preserving the registers when actually
necessary.

An example is shown in {\tt fib-2.asm}.  The {\tt fib} function is
very simple, and doesn't preserve many registers, but the basic idea
of handling the base case separately is illustrated.

% \begin{figure}
% \caption{The source code of {\tt fib-2.asm}}
% \hrule
% \input{Tutorial/fib-2}
% \hrule
% \end{figure}

\end{document}


% $Id: tutex.tex,v 1.7 2002/04/16 01:02:38 ellard Exp $

\section{Exercises}

\begin{enumerate}

\item The {\tt add2.asm} program can produce confusing output-- if the
	sum of the two numbers is greater than 127 or less than -128,
	the incorrect sum will be printed.
	
	\begin{enumerate}
	
	\item Starting with a copy of {\tt add2.asm}, write a program
		named {\tt add3.asm} that prints a warning message
		if this occurs.

	\item Extend {\tt add3.asm} so that it always prints the
		correct sum, even if the sum is larger than 127 or
		less than -128.
		
		Hint-- the sum will be no larger than 254 and no
		smaller than -256, and there are only three main
		situations to worry about-- the correct sum is between
		-128 and 127 (in which case nothing special is
		necessary), between 128 and 254, or between -129 and
		-256. 

	\end{enumerate}

\item Write an {\sc Ant-8} program named {\tt box.asm} that asks the
	user for a height and a width, makes sure that both are larger
	than zero but less than twenty, and then draws a solid box of
	asterisks with the given height and width.

\item Write an {\sc Ant-8} program named {\tt box2.asm} that asks the
	user for a height and a width, makes sure that both are larger
	than zero but less than twenty, and then draws a hollow box of
	asterisks with the given height and width.

\item Write an {\sc Ant-8} program named {\tt decimal1.asm} that takes
	as input a single hexadecimal number in the range {\tt 00} - {\tt
	7F} and prints it in decimal notation.
	
	Note that the range of {\tt 00} through {\tt 7F} in hex is
	equal to the range from 0 to 127 in decimal-- you only need to
	deal with positive numbers.

\item Write an {\sc Ant-8} program named {\tt decimal2.asm} that takes
	as input a single hexadecimal number in the range {\tt 80} -
	{\tt 7F} and prints it in decimal notation.  Recall that {\tt
	80} in hex is -128 in decimal, and {\tt FF} in hex is -1 in
	decimal.

\item Write an {\sc Ant-8} program named {\tt sort.asm} that reads 20
	numbers from the user, sorts them, and then prints them in
	ascending order.

\end{enumerate}


% 02/13/94
% $Id: ant-chap.tex,v 1.7 2002/04/16 01:02:37 ellard Exp $
%

\chapter{The {\sc Ant-8} Instruction Set and Assembly Language}
\label{ant-chapter}

This chapter gives a more detailed description of the {\sc Ant-8}
\input{../CurrVersion} instruction set and some additional details
about the {\sc Ant-8} assembler that are not covered by the tutorial
chapter.  The exact definition of the {\sc Ant-8} instruction set and a
precise specification for how {\sc Ant-8} programs are executed are
given in the {\em {\sc Ant-8} \input{../CurrVersion} Architecture
Reference}.

% Dan Ellard -- 02/13/94
% $Id: qrr.tex,v 1.10 2003/02/14 16:52:04 ellard Exp $
%

\section{{\sc Ant-8} Architecture Overview}

The {\sc Ant-8} architecture is a load/store architecture; the only
instructions that can access memory are the {\em load} and {\em store}
instructions.  All other operations access only registers (or, in the
case of {\tt in} and {\tt out}, access peripherals).

The {\sc Ant-8} CPU has 16 registers, named {\tt r0} through {\tt r15}.
Register {\tt r0} always contains the constant 0, and register
{\tt r1} is used to hold results related to previous operations
(described later).  {\tt r0} and {\tt r1} are {\em read-only}.
They can be used as destination registers, but their values are
unchanged.
The other 14 registers ({\tt r2} through {\tt r15}) are
general-purpose registers.

\section{Instructions}
\label{mnemonic-sec}

In the description of the instructions, the notation described in
the following table is used:

\vspace{3mm}
\begin{center}
\begin{tabular}{|lp{5.5in}|}
\hline
{\em des}       & Must always be a register. The {\em des} register
			may be modified by the instruction. \\
{\em reg}       & Must always be a register. \\
{\em src1}      & Must always be a register. \\
{\em src2}      & Must always be a register. \\
{\em const8}     & Must be an 8-bit constant (-128 .. 127):
			an integer (signed), char, or label. \\
{\em uconst8}	& Must be an 8-bit constant (0 .. 255):
			an integer (unsigned) or label. \\
{\em uconst4}	& Must be a 4-bit constant integer (0 .. 15). \\
\hline
\end{tabular}
\end{center}
\vspace{3mm}

The {\sc Ant-8} assembly language instructions are listed in Figure
\ref{mnemonic-table}.

Note that for all instructions, register {\tt r1} is always
updated {\em after} the rest of the instruction is done,
so that it is always safe to use {\tt r1} as a source register.

\begin{figure}[htp]
\caption{ \label{mnemonic-table} {\sc Ant-8} Instruction Mnemonics }
\vspace{3mm}
\noindent
\begin{tabular}{|ll|p{4.5in}|}
\hline
        {\bf Op}        & {\bf Operands}        & {\bf Description}     \\
\hline
\hline
        {\tt add}       & {\em des, src1, src2} &
                {\em des} gets {\em src1} + {\em src2}.
                {\tt r1} gets 1 if the result is $>$ 127,
		-1 if $<$ -128, or 0 otherwise. \\
\hline
        {\tt sub}       & {\em des, src1, src2} &
                {\em des} gets {\em src1} - {\em src2}.
                {\tt r1} gets 1 if the result is $>$ 127,
		-1 if $<$ -128, or 0 otherwise. \\
\hline
        {\tt mul}       & {\em des, src1, src2} &
                Multiply {\em src1} and {\em src2},
                leaving the low-order byte in register {\em des}
                and the high-order byte in register {\tt r1}. \\

\hline
        {\tt and}       & {\em des, src1, src2} &
                {\em des} gets the bitwise logical {\sc and} of
                {\em src1} and {\em src2}.  {\tt r1} gets the
                bitwise negation of the {\sc and} of {\em src1} and {\em src2}. \\
\hline
        {\tt nor}        & {\em des, src1, src2} &
                {\em des} gets the bitwise logical {\sc nor} of
                {\em src1} and {\em src2}.  {\tt r1} gets the
                bitwise negation of the {\sc nor}
		of {\em src1} and {\em src2}. \\
\hline
        {\tt shf}        & {\em des, src1, src2} &
		{\em des} gets the bitwise shift of {\em src1} by
		{\em src2} positions.  If {\em src2} is positive,
		{\em src1} is shifted to the left, if {\em src2}
		is negative {\em src1} is shifted to the right. \\
\hline
{\tt beq}       & {\em reg, src1, src2} &
        Branch to {\em reg} if {\em src1} is equal to {\em src2}.
        {\tt r1} is unchanged. \\
\hline
{\tt bgt}       & {\em reg, src1, src2} &
        Branch to {\em reg} if {\em src1} $>$ {\em src2}.
        {\tt r1} is unchanged. \\
\hline
{\tt ld1}        & {\em des, src1, uconst4} & 
        Load the byte at {\em src1 + uconst4} into {\em des}.
        {\tt r1} is unchanged.
        \\
\hline
{\tt st1}        & {\em reg, src1, uconst4} &
        Store the contents of register {\em reg} to {\em src1 + uconst4}.
        {\tt r1} is unchanged.
        \\
\hline
{\tt lc}        & {\em des, const8}      & 
        Load the constant {\em const8} into {\em des}.
        {\tt r1} is unchanged. \\
\hline
{\tt jmp}	& {\em uconst8}	&
	Branch unconditionally to the specified constant.
	{\tt r1} is set to the address
	of the instruction following the {\tt jmp}. \\

\hline

{\tt inc}	& {\em reg, const8}	&
	Add {\em const8} to the specified register.
                {\tt r1} gets 1 if the result is $>$ 127,
		-1 if $<$ -128, or 0 otherwise. \\
\hline

{\tt in}	& {\em des, uconst4}	&
	Read in a byte from the peripheral specified by {\em uconst4},
	and place it in the destination register. {\tt r1} is set to 0
	if successful, 1 if EOI is reached.
	\\
\hline
{\tt out}	& {\em src1, uconst4}	&
	Write the byte in the {\em src1} register to the peripheral
	specified by {\em uconst4}. {\tt r1} is set to 0. \\

\hline
        {\tt hlt}   & 			&
		Dump core to {\tt ant8.core}, and halt. \\

\hline 
\end{tabular}
\end{figure}
\vspace{3mm}

\subsection{{\sc Ant-8} Peripherals}
\label{peripheral-sec}

The {\tt in} and {\tt out} operations set {\tt r1} to 0 if successful,
and set {\tt r1} to non-zero values to indicate failure. 

\vspace{3mm}
% $Id: ant-periph.tex,v 1.3 2002/01/02 02:13:47 ellard Exp $

\vspace{3mm}
\noindent
\begin{tabular}{|l||c|l|p{3.75in}|}
\hline
{\bf Operation}   & {\bf Format}    & {\bf Mnemonic} & {\bf Description} \\
\hline
\hline
{\tt in}	& 0     & {\tt Hex}	& Read two characters and interpret
				them as an 8-bit hexadecimal number.
				Each character must be a valid hexadecimal
				number.  \\
		& 1     & {\tt Binary}	& Read eight characters and interpret
				them as an 8-bit binary number.
				Each character must be '0' or '1'. \\
		& 2	& {\tt ASCII}	& Read a single character, as ASCII. \\
\hline
{\tt out}	& 0     & {\tt Hex}	& Write an 8-bit number as two
				characters, using hexadecimal notation.  \\
		& 1     & {\tt Binary}	& Write an 8-bit number as 8
				characters, using binary notation. \\
		& 2	& {\tt ASCII}	& Write the 8-bit number, as ASCII. \\
\hline
\end{tabular}

\vspace{3mm}

Note that the behavior of the {\tt in} operation is undefined if the
input is not properly formatted, or contains illegal characters.  For
hexadecimal input, only the characters {\tt 0-9} and {\tt A-F} (or
{\tt a-f}) may be used.  For binary input, only the {\tt 1} and {\tt
0} characters are permitted.  There is no restriction on {\sc ASCII}
characters.

\vspace{3mm}



%%% end of ant.tx


% $Id: ant-asm.tex,v 1.5 2002/04/16 01:02:37 ellard Exp $

\section{The Ant-8 Assembler Reference}

\subsection{Comments}

A comment begins with a \verb$#$ and continues until the following
end-of-line.  The only exception to this is when the \verb$#$ character
appears as part of an ASCII character constant (as described in
section \ref{data-const-sec}).

\subsection{The {\tt \_data\_} Label}

A special label, {\tt \_data\_}, is used to mark the boundary between
the instructions of the program (which must appear before the {\tt
\_data\_} label) and the data of the program (which appear afterward).

\subsection{Constants}
\label{data-const-sec}

Several Ant-8 assembly instructions contain 8-bit or 4-bit constants.

The 8-bit constants can be specified in a variety of ways:
as decimal, octal, hexadecimal, or binary numbers, {\sc ASCII} codes (using
the same conventions as C), or labels.  Examples are shown in the
following table:

\begin{center}
\begin{tabular}{|l|l|l|}
\hline
Representation	& Value	& Decimal Value \\
\hline
{\em Decimal (base 10)}		&	{\tt 65}	&	65 \\
{\em Hexadecimal (base 16)}	&	{\tt 0x41}	&	65 \\
{\em Octal (base 8)}		&	{\tt 0101}	&	65 \\
{\em Binary (base 2)}		&	{\tt 0b01000001}&	65 \\
{\em {\sc ASCII}}		&	{\tt 'A'}	&	65 \\
\hline
{\em Decimal (base 10)}		&	{\tt 10}	&	10 \\
{\em Hexadecimal (base 16)}	&	{\tt 0xa}	&	10 \\
{\em Octal (base 8)}		&	{\tt 012}	&	10 \\
{\em Binary (base 2)}		&	{\tt 0b1010}	&	10 \\
{\em {\sc ASCII}}		&	{\tt '\verb$\$n'}	&	10 \\
\hline
\end{tabular}
\end{center}
\vspace{3mm}

The value of a label is the index of the subsequent instruction in
instruction memory for labels that appear in the code, or the index of
the subsequent {\tt .byte} item for labels that appear in the data.

The 4-bit constants must be specified as unsigned numbers (using
decimal, octal, hexadecimal, or binary notation).  ASCII constants or
labels cannot be used as 4-bit constants, even if their value can be
represented in 4 bits.

\subsection{Symbolic Constants}
\label{data-symconst-sec}

Constants can be given symbolic names via the {\tt .define} directive. 
This can result in substantially more readable code.  The first
operand of the {\tt .define} directive is the symbolic name for the
constant, and the second value is an integer constant.  The integer
constant must not be a label or another symbolic constant, however.

\vspace{3mm}
{\codesize
\begin{verbatim}
        .define ROWS, 10        # Defining ROWS to be 10
        .define COLS, 10        # Defining COLS to be 10

        lc      r2, ROWS        # Using ROWS as a constant
        inc     r3, COLS        # Using COLS as a constant
\end{verbatim}}
\vspace{3mm}

\subsection{The {\tt .byte} Directive}
\label{data-directive-sec}
\label{byte-figure}

The {\tt .byte} directive is used to specify data values to be
assembled into the next available locations in memory.

\vspace{3mm}
\noindent
\begin{tabular}{|ll|p{4.0in}|}
\hline
{\bf Name}      & {\bf Parameters}      & {\bf Description}     \\
\hline
{\tt .byte}     & {\em byte1, $\cdots$, byteN }   &
		Assemble the given bytes (8-bit integers) into the
		next available locations in the data segment.  As many
		as 8 bytes can be specified on the same line.  Bytes
		may be specified as hexadecimal, octal, binary, decimal
		or character constants (as described in
		\ref{data-const-sec}).

		No more than 8 bytes can be defined with the same
		{\tt .byte} statement.
                \\
\hline
\end{tabular}
\vspace{3mm}





\backmatter

\printindex

\end{document}
