% $Id: tutex.tex,v 1.7 2002/04/16 01:02:38 ellard Exp $

\section{Exercises}

\begin{enumerate}

\item The {\tt add2.asm} program can produce confusing output-- if the
	sum of the two numbers is greater than 127 or less than -128,
	the incorrect sum will be printed.
	
	\begin{enumerate}
	
	\item Starting with a copy of {\tt add2.asm}, write a program
		named {\tt add3.asm} that prints a warning message
		if this occurs.

	\item Extend {\tt add3.asm} so that it always prints the
		correct sum, even if the sum is larger than 127 or
		less than -128.
		
		Hint-- the sum will be no larger than 254 and no
		smaller than -256, and there are only three main
		situations to worry about-- the correct sum is between
		-128 and 127 (in which case nothing special is
		necessary), between 128 and 254, or between -129 and
		-256. 

	\end{enumerate}

\item Write an {\sc Ant-8} program named {\tt box.asm} that asks the
	user for a height and a width, makes sure that both are larger
	than zero but less than twenty, and then draws a solid box of
	asterisks with the given height and width.

\item Write an {\sc Ant-8} program named {\tt box2.asm} that asks the
	user for a height and a width, makes sure that both are larger
	than zero but less than twenty, and then draws a hollow box of
	asterisks with the given height and width.

\item Write an {\sc Ant-8} program named {\tt decimal1.asm} that takes
	as input a single hexadecimal number in the range {\tt 00} - {\tt
	7F} and prints it in decimal notation.
	
	Note that the range of {\tt 00} through {\tt 7F} in hex is
	equal to the range from 0 to 127 in decimal-- you only need to
	deal with positive numbers.

\item Write an {\sc Ant-8} program named {\tt decimal2.asm} that takes
	as input a single hexadecimal number in the range {\tt 80} -
	{\tt 7F} and prints it in decimal notation.  Recall that {\tt
	80} in hex is -128 in decimal, and {\tt FF} in hex is -1 in
	decimal.

\item Write an {\sc Ant-8} program named {\tt sort.asm} that reads 20
	numbers from the user, sorts them, and then prints them in
	ascending order.

\end{enumerate}
