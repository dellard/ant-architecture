% $Id: reg32.tex,v 1.18 2002/04/22 16:32:21 ellard Exp $
% $Date: 2002/04/22 16:32:21 $

\section{Introduction}

Any of the general registers in the Ant-32 architecture can, in
general, be used in whatever way the programmer wishes.  The
architecture imposes no restrictions or limitations (apart from the
restriction that the zero register always contains the constant 0, and
that for the operations that take a register pair as an operand, the
register pair must begin with an even-numbered register).

Most software architectures, however, include some conventions about
the use of specific registers.  These conventions are principally
focussed on supporting features of high-level languages, such as
functions, recursion, and separate compilation.

In order to facilitate the implementation of higher-level software
architectures using Ant-32, the Ant-32 tools support two register
names and conventions.

The first is a very simple model, useful for introductory programming
courses and demonstrating how function calls and recursion can be
implemented.  This convention is the focus of the rest of this document.

The second is a more advanced model, which refines the simple model in
a manner that allows for more efficient code.  It is described only
briefly in this document.

\section{Simple Register Use Conventions}

The simple register use conventions implement a straight-forward stack
architecture.  The conventions are outlined in Figure
\ref{simple-conventions-fig}, and described in more detail below.

\begin{figure}
\caption{\label{simple-conventions-fig} Simplified Register Use Conventions}

\begin{center}
\begin{tabular}{|l|l|p{2.0in}|}

\hline
	{\bf Mnemonic}	& {\bf Registers} & {\bf Description} \\
\hline
\index{ze - the zero register}
	{\tt ze}	& {\tt r0}		& Always zero \\
\hline
\index{ra - the return address}
	{\tt ra}	& {\tt r1}		& Return address \\
\hline
\index{sp - the stack pointer}
	{\tt sp}	& {\tt r2}		& Stack pointer \\
\hline
\index{fp - the frame pointer}
	{\tt fp}	& {\tt r3}		& Frame pointer \\
\hline
	{\tt g0-g55}	& {\tt r4 - r59}	& General-purpose registers \\
\hline
\index{u0-u3 - scratch registers}
	{\tt u0-u3}	& {\tt r60 - r63}	& Reserved registers \\
\hline

\end{tabular}
\end{center}

\end{figure}

\subsection{{\tt ze} - The zero register}

	The {\tt ze} register is simply register zero, which always
	contains the number zero.

\subsection{{\tt ra} - The Return Address}

	The {\tt ra} register is used to store the return address of
	the most recent function call.

\subsection{{\tt sp} - The Stack Pointer}

	{\tt sp} is used as the {\em stack pointer}.  The stack grows
	``downward''; a push moves the stack pointer to a numerically
	lower address, and a pop moves the stack pointer toward
	numerically greater address.

	The Ant-32 architecture does not contain native push or pop
	instructions, and these operations require more than one
	instruction to execute.  The push and pop operations, for
	example, can be coded as shown in Figure
	\ref{simple-push-pop}.

	In general, {\tt sp} points to the ``top'' of the stack
	(although this may seem somewhat confusing, since the stack
	grows downward -- so the top of the stack is located at the lowest
	address).  This convention can be relaxed in order to
	implement groups of push or pop operations (see Figure
	\ref{combined-push-pop}), as long as the stack pointer is
	never moved past any values that are still on the stack.


	\begin{figure}
	\hrule
	\caption{\label{simple-push-pop}
			Implementing {\tt push} and {\tt pop}}

        \begin{verbatim}
                # push register g0:
                subi    sp, sp, 4
                st4     g0, sp, 0

                # pop into register g1:
                ld4     g1, sp, 0
                addi    sp, sp, 4
        \end{verbatim}
	\hrule
	\end{figure}

	The Ant-32 assembler provides macro implemenations of {\tt
	push} and {\tt pop}, using this method.

	\begin{figure}
	\hrule
	\caption{\label{combined-push-pop}
			Combining Multiple Push or Pop Operations}
			\vspace{3mm}

	For consecutive pushes and pops, it can increase code
	efficiency to reduce the number of {\tt addi} and {\tt subi}
	instructions by aggregating the movement of the stack pointer,
	as shown in the following code fragment.

	\begin{verbatim}
                # push registers g0, g1, g2:
                subi    sp, 12
                st4     g0, sp, 8
                st4     g1, sp, 4
                st4     g2, sp, 0

                # pop into registers g3, g4, g5:
                ld4     g3, sp, 0
                ld4     g4, sp, 4
                ld4     g5, sp, 8
                addi    sp, 12
        \end{verbatim}
	\hrule
	\end{figure}

\subsection{{\tt fp} - The Frame Pointer}

The {\tt fp} register is used as a {\em frame pointer}.  The frame is
often used to implement activation records, or simplify the
implementation of function calls.

\subsection{{\tt g0}-{\tt g55} - General-Purpose Registers}

These registers are free to be used for any purpose.

\subsection{{\tt u0}-{\tt u3} - Reserved Registers}

These registers are reserved for use by the assembler.  They are used
as scratch space for the expansion of macros.  They should not be used
for any other purpose, and programs should never make any assumptions
about their contents.

\section{Function Calls}
\index{function calls}

This section describes how the stack pointer, frame pointer, and
return address registers can be used to implement the abstraction of
function calls.  The description is divided into four steps:

\begin{enumerate}

\item	Preparing to call the function and performing the call.

\item	Function preamble.

\item	Preparing to return from the function.

\item	Cleaning up after the function call.

\end{enumerate}

\subsection{Preparing to Call: Using {\tt call}}
\label{reg32-call-sec}
\index{call}

\begin{enumerate}

\item All of the {\tt g-}registers whose values need to be preserved
	are pushed onto the stack.  The order that they are pushed
	onto the stack is up to the caller.

	Before the function call takes place, the caller must save any
	registers that contain necessary values, because otherwise the
	function might overwrite these values.

\item The arguments to the function are pushed onto the stack, in the
	reverse order that they appear (from right to left).

	The stack only contains whole words (32-bit values).  If the
	arguments to the function are 8 or 16-bit values, then they
	are still pushed as the lower 8 or 16 bits of a 32-bit value,
	requiring four bytes of storage.  It is the responsibility of
	the called function to ignore the extra bits.

\item Jump or branch to the function (using {\tt jez}, {\tt jnz}, {\tt
	bez}, or {\tt bnz}), specifying the return address register
	{\tt ra} as the destination register.

\end{enumerate}

Note that the last step can be accomplished with the {\tt call} macro.

\subsection{Handling the Call: Using {\tt entry}}
\index{entry}
\label{reg32-entry-sec}

\begin{enumerate}

\item The current value of the {\tt fp} and {\tt ra} registers
	are pushed onto the stack.

\item The frame pointer gets a copy of the stack pointer.

\item The stack pointer is decremented by the size of the local frame. 
	The area of memory thus allocated between the stack pointer
	and the frame pointer is used for local storage -- for example,
	the local variables of the current function.

	Note that the local frame size must always be a multiple of 4,
	so that the stack pointer is always aligned properly on a
	4-byte boundary.

\end{enumerate}

These steps can be accomplished by using the {\tt entry} macro.  This
macro takes a single constant argument, which is the size of the stack
frame to create.

After this preamble is finished, the stack contains the information
about the function call in the order shown in Figure
\ref{simple-stack-call}.

\begin{figure}
\caption{\label{simple-stack-call} Stack at start of call.}

\begin{center}

\begin{tabular}{|l|l|l|}
\hline
{\bf Address}	& {\bf Contents} & {\bf Description} \\
\hline
$\vdots$	& {\tt g0} $\cdots$ {\tt g55}	&
		Saved copies of {\tt g-}registers. \\
$\vdots$	& $\vdots$			& \\
\hline
${\tt fp} + 8 + (N ~ \times 4 )$ 	& $arg_{N}$	& \\
$\vdots$	& $\vdots$			& 
		Arguments to the function. \\
${\tt fp} + 8$	& $arg_{0}$		& \\

\hline
${\tt fp} + 4$	& {\tt fp}			&
		The saved value of the {\tt fp}. \\
\hline
${\tt fp} + 0$	& {\tt ra}			&
		The saved value of the {\tt ra}. \\
\hline
${\tt fp} - 4$	& 				& \\
$\vdots$	& $\vdots$			&
		{\em local variables}		\\
${\tt fp} - (4 + (M ~ \times 4))$	&	& \\
\hline
\end{tabular}

\end{center}

\end{figure}

Note that the function can always access its arguments and local
variables via fixed offsets relative to the frame pointer, and the
stack pointer is free to move.  For example, the first argument
($arg_{0}$) is accessible at the address ${\tt fp} + 8$, while the
second argument is at address ${\tt fp} + 12$, and so forth.

During a function call, the stack pointer can be used to manage the
allocation of dynamic but function-private storage.  If the storage
requirements of the function can be computed in advance, however, it
can be just as convenient to allocate this space from the frame.

\subsection{Returning from a Call: Using {\tt return}}
\index{return}
\label{reg32-return-sec}

\begin{enumerate}

\item The return value (if any) is put into register {\tt g0}.

	Functions that return multiple values, or a single value that
	is too large to fit into a single register, use a more
	complicated method for returning their values.  This method is
	not documented here.

\item The stack pointer is reset to contain a copy of the frame
	pointer.

\item The return address is popped into {\tt ra}, and then
	the {\tt ra} register is incremented by 4.

	This increment is necessary because when the function is
	called via a jump or branch instruction, {\tt ra} gets the
	address of the instruction that performed the call.  The
	address we want to return to is the address of the instruction
	after the call.

\item The frame pointer is popped into {\tt fp}.

	At this point, the stack pointer is in the same position as it
	was before the function was called.

\item Use the {\tt jez} instruction to jump to the {\tt ra}.

\end{enumerate}

For a function that returns a single value, the {\tt return} macro is
provided to perform all of these steps.  The single operand to the
{\tt return} macro can be the name of the register that contains the value
to return, or the constant to return.

\subsection{Handling the Return}

When the execution resumes in the caller, the stack is exactly the
same as it was before the jump to the caller.  All that remains is to
save the results, and restore the rest of the environment to the way
it was before the call took place.  This can be done by popping the
parameters and then by popping the saved {\tt g}-registers.  Once the
stack is restored, execution can resume as normal.

\section{Examples of Functions}

\newpage
Program {\tt add-func.asm} gives a very simple example of a
function that takes two arguments and returns their sum.

\vspace{3mm}
\hrule
\index{add-func.asm}
\input{Tut32/add-func}

\newpage
Program {\tt fibonacci.asm} gives an example of a recursive
function.

\vspace{3mm}
\hrule
\index{fibonacci.asm}
\input{Tut32/fibonacci}

\newpage
\section{Advanced Register Use Conventions}
\index{function calls, optimized}

\subsection{Optimizing Saving and Restoring of Registers}

The function calling conventions described in the first part of this
chapter can result in very inefficient code.  For example, imagine
that we have a function $\alpha$ that calls function $\beta$.  Before
$\alpha$ calls $\beta$, it has to save all the registers it is using. 
If $\alpha$ uses many registers, and $\beta$ only uses a few, then it
may be that many of $\alpha$'s registers didn't need to be saved,
because their values weren't modified by $\beta$ at all.

One solution to this particular problem is to change the
responsibility for saving the registers to the called function -- in
this case, $\beta$ would be responsible for saving and restoring the
few registers that it uses.  Unfortunately, in the opposite case,
where $\alpha$ only uses a few registers and $\beta$ uses many, then
this approach results in the same kind of inefficiency as we saw
initially.

Ideally, each function would have its own set of registers available
for its exclusive use.  Unfortunately, this is impossible:  typical
programs have thousands of functions but processors only have dozens
of registers -- and even if a huge number of registers were available,
recursive functions would still be a problem.

However, there is a relatively straightforward way to solve most of
this problem, by dividing the register set into two groups -- one group
which is caller-saved (like all the registers in the earlier
convention) and a second which is callee-saved.  Ideally, functions
that call other functions will use the callee-saved registers, and
{\em leaf functions} (functions that do not call other functions) or
the base case code of recursive functions will use the caller-saved
registers.  If, in our previous example, $\beta$ is a leaf function,
then if $\alpha$ uses only callee-saved registers, and $\beta$ uses
only caller-saved registers, then no registers will need to be saved
at all.

\subsection{Optimizing Parameter Passing}

Another cause of inefficiency in the normal function call conventions
is the pushing of the parameters onto the stack, and then accessing them
via the frame pointer.   In terms of the number of instructions executed,
this convention is not terribly inefficient -- but in terms of the {\em kind}
of instructions executed, it can be very slow.  Passing the parameters on
the stack means storing to memory and then loading from memory, and on
most modern processors accessing memory is at least an order of magnitude 
slower than accessing values in registers.

Therefore, to optimize the passing of parameters, we reserve a small
number of registers to use for passing parameters.  If there are more
parameters than will fit in these registers, the remainder are passed
on the stack as before.  Studies of existing bodies of software have
shown, however, that six (or even four) argument registers are
sufficient for an overwhelming majority of common functions.

\subsection{The Advanced Conventions}

\begin{figure}
\caption{\label{advanced-conventions-fig} Advanced Register Use Conventions}

\begin{center}
\begin{tabular}{|l|l|p{2.0in}|}

\hline
	{\bf Mnemonic}	& {\bf Registers} & {\bf Description} \\
\hline
	{\tt ze}	& {\tt r0}		& Always zero \\
\hline
	{\tt ra}	& {\tt r1}		& Return address \\
\hline
	{\tt sp}	& {\tt r2}		& Stack pointer \\
\hline
	{\tt fp}	& {\tt r3}		& Frame pointer \\
\hline
	{\tt v0-1}	& {\tt r4 - r5}		& Returned values \\
\hline
	{\tt a0-5}	& {\tt r6 - r11}	& Argument registers \\
\hline
	{\tt s0-23}	& {\tt r12 - r35}	& Callee-saved \\
\hline
	{\tt t0-23}	& {\tt r36 - r59}	& Caller-saved \\
\hline
	{\tt u0-3}	& {\tt r60 - r63}	& Reserved for the assembler \\
\hline

\end{tabular}
\end{center}

\end{figure}

The conventions are similar to the previous, except that the {\tt
g}-registers have been partitioned into four different kinds of
registers:  return value registers, argument value registers, saved
registers, and temporary registers.  These registers are described in
more detail below.

\subsubsection{Return Value Registers: {\tt v0} - {\tt v1}}

Values returned from a function.  If the return value of the function
requires more than two registers to express, the remainder of the
return value is returned via the stack.

\subsubsection{Argument Value Registers: {\tt a0} - {\tt a5}}

Parameters to a function.  If the function has more than six
parameters, then the additional parameters are pushed onto the stack,
in the opposite order that they appear (right to left).

\subsubsection{Callee Saved Registers:  {\tt s0} - {\tt s23}}

If any of these registers are used by a function, then the function is
responsible for saving their original values and then restoring them
when the function returns.

How the values are preserved and restored is up to the implementation. 
For implementations of languages that permit recursive or reentrant
functions, using the stack is an appropriate method.

\subsubsection{Temporary (Caller Saved) Registers:  {\tt t0} - {\tt t23}}

If any of these registers contains live values when a function is
called, they are preserved by the caller and then restored after the
function has returned.

How the values are preserved and restored is up to the implementation. 
For implementations of languages that permit recursive or reentrant
functions, using the stack is an appropriate method.

The distinction between the saved registers and the temporary
registers allows some useful optimizations, especially with leaf
functions (functions that do not call any other functions) or the base
case of recursive functions.  If these functions use can manage to
exclusively use {\tt t}-registers, and their callers use only {\tt
s}-registers, then these calls do not require saving and restoring any
registers:  it is the responsibility of the caller to save any {\tt
t}-registers it needs, and the callee to save any {\tt s}-registers it
needs, so if the caller only uses {\tt s}-registers and the callee
only uses {\tt t}-registers, a significant reduction in the overhead
of function calls is obtained.

