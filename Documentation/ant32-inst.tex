% $Id: ant32-inst.tex,v 1.4 2002/04/16 15:19:09 ellard Exp $

\chapter{The Ant-32 Instruction Set}
\label{ant32-inst}

\section{Terms}

Figure \ref{Terms} gives the definitions of some of the terms
used throughout the rest of this section.

\begin{figure}[ht]
\caption{\label{Terms} Terms}

\begin{center}
\begin{tabular}{|l|l|}
\hline

{\em src}	& General integer register \\
\hline

{\em des}	& General integer register modified by an instruction \\
\hline

{\em const16}	& Signed 16-bit constant \\
\hline

{\em uconst8}	& Unsigned 8-bit constant \\
\hline

{\em const8}	& Signed 8-bit constant \\
\hline

{\em uconst5}	& Unsigned 5-bit constant \\
\hline

\end{tabular}
\end{center}
\end{figure}

\section{Arithmetic Operations}

The arithmetic operations fall into four classes:

\begin{itemize}

\item 32-bit register-to-register operations, which take two 32-bit
	register operands and generate a 32-bit result.

\item 32-bit immediate operations, which take one 32-bit register
	operand and one constant and generate a 32-bit result.

\item 64-bit register-to-register operations, which take two 32-bit
	register operands and generate a 64-bit result.

\item 64-bit immediate operations, which take one 32-bit register
	operand and one constant and generate a 64-bit result.

\end{itemize}

The mnemonics for the 64-bit operations all end in the ``o''
character.  For the 64-bit operations, the {\tt des} register must be
even, and the upper 32 bits of the result are placed in register {\tt
des+1}.  For example, if the destination is {\tt r2}, then the
overflow will be placed in {\tt r3}.

\subsection{32-Bit Three-register Operations}

The 32-bit three-register operations all use three registers:  the
first is the destination register, and the second two are source
registers.  Overflow or underflow is ignored.

\begin{description}

\item	\INSTspec{add}
\item	\INSTspec{sub}
\item	\INSTspec{mul}
\item	\INSTspec{div}
\item	\INSTspec{mod}
\item	\INSTspec{or}
\item	\INSTspec{nor}
\item	\INSTspec{xor}
\item	\INSTspec{and}
\item	\INSTspec{shr}
\item	\INSTspec{shru}
\item	\INSTspec{shl}

\end{description}

\subsection{32-Bit Immediate Operations}

The 32-bit immediate arithmetic operations take two integer registers: 
(one source and one destination) and a signed 8-bit constant.  Overflow or underflow is ignored.

\begin{description}

\item \INSTspec{addi}
\item \INSTspec{subi}
\item \INSTspec{muli}
\item \INSTspec{divi}
\item \INSTspec{modi}
\item	\INSTspec{shri}
\item	\INSTspec{shrui}
\item	\INSTspec{shli}

\end{description}

\subsection{64-Bit Three-Register Operations}

The 64-bit three-register operations all use three registers operands: 
the first is the base of the register pair used to hold the result,
and the second two are {\em src} registers.

\begin{description}

\item	\INSTspec{addo}
\item	\INSTspec{subo}
\item	\INSTspec{mulo}

\end{description}

\subsection{64-Bit Immediate Operations}

The 64-bit immediate arithmetic operations take two integer registers: 
(one source and one destination) and a signed 8-bit constant.

\begin{description}

\item \INSTspec{addio}
\item \INSTspec{subio}
\item \INSTspec{mulio}

\end{description}

\section{Comparison Operations}

The comparison operations take three registers.  The {\em des}
register is set to 0 or 1 depending on whether the given condition
holds for the two {\em src} registers.

\begin{description}

\item	\INSTspec{eq}
\item	\INSTspec{gts}
\item	\INSTspec{ges}
\item	\INSTspec{gtu}
\item	\INSTspec{geu}

\end{description}

\section{Branch Operations}

There are two sets of change of control flow operators.  The first are
branches relative to the address of the currently executing
instruction, and the second are absolute jumps.  Both are constructed
in such a way to allow for use as subroutine call instructions.
If used for simple branching, {\tt r0} may be specified as the 
destination register.
Each set is further divided into register-based and immediate
categories.  Note that unconditional branches and jumps result if
\Reg{src1} is set to {\tt r0}.

There are both register-based branches and immediate branches.
The register-based branches take three general registers.
If the given condition holds for \Reg{src1}
register, then execution branches to location indicated by
\Reg{src2} and the address of the currently executing instruction
is placed in \Reg{des}.
In the immediate form, execution branches to a location computed by
multiplying the signed 16-bit constant by 4 and then adding it to the
address of the currently executing instruction.

\begin{description}

\item	\INSTspec{bez}
\item	\INSTspec{jez}
\item	\INSTspec{bnz}
\item	\INSTspec{jnz}
\item	\INSTspec{bezi}
\item	\INSTspec{bnzi}
\end{description}

\section{Load/Store Operations}

The load instructions take two registers and a signed 8-bit constant. 
The first register is assigned the value loaded from the address which
is the sum of \Reg{src1} and the constant.

\begin{description}

\item	\INSTspec{ld1}
\item	\INSTspec{ld4}
%%% \item	\INSTspec{ldf}

\end{description}

The store instructions take two registers and a signed 8-bit constant. 
The \Reg{src1} is the value to store, and it is stored at the address
which is the sum of \Reg{src2} and the constant.

\begin{description}

\item	\INSTspec{st1}
\item	\INSTspec{st4}
%%% \item	\INSTspec{stf}
\end{description}

\begin{description}
\item	\INSTspec{ex4}
\end{description}

\section{Constants}

The lc operations take a single register and a signed 16-bit constant. 
{\tt lcl} performs sign extension.

\begin{description}

\item	\INSTspec{lcl} 
\item	\INSTspec{lch} 

\end{description}

\section{Special Instructions}

\begin{description}
\item	\INSTspec{trap}
\item	\INSTspec{info}
\end{description}

\section{Optional Instructions}

The following instructions are provided optionally by the architecture.
The {\tt info} instruction can be used to determine which instructions
are implemented on a specific processor.

{\tt srand} can only be executed in supervisor mode.

\begin{description}

\item	\INSTspec{srand}
\item	\INSTspec{rand}
\item	\INSTspec{cin}
\item	\INSTspec{cout}

\end{description}

\section{MMU operations (supervisor mode only)}

\begin{description}

\item	\INSTspec{tlbpi}

\item	\INSTspec{tlble}

\item	\INSTspec{tlbse}

\end{description}

\section{Supervisor Mode Instructions}

\begin{description}
\item	\INSTspec{leh}
\item	\INSTspec{rfe}
\item	\INSTspec{sti}
\item	\INSTspec{cli}
\item	\INSTspec{ste}
\item	\INSTspec{cle}
\item	\INSTspec{timer}
\item	\INSTspec{idle}
\item	\INSTspec{halt}
\end{description}


