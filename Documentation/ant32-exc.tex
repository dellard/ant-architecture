% $Id: ant32-exc.tex,v 1.4 2002/05/29 18:43:00 ellard Exp $

\chapter{The Ant-32 Exception Architecture}
\label{ExceptionArchitecture}

When the Ant-32 processor takes an exception, control transfers to the
address that has been specified as the {\em exception handler
address}, via the {\tt leh} instruction.  If an exception occurs
before the exception handler address has been loaded, the behavior is
undefined.

There are eight exception registers, {\tt k0-k3} and {\tt e0-e3}, and two
processor flags relating to exception handling: the interrupt disable
flag and the exception disable flag. All of these items are accessible
only in supervisor mode.  The exception registers occupy the last eight
positions in the register name space, with the following mappings 
and definitions given in Figure \ref{exc-registers}.

\begin{figure}[ht]
\begin{center}
\caption{\label{exc-registers} Exception Registers}

\begin{tabular}{|l |l |p{4.in} |l|}
\hline

r248	& k0	&		&\\
\cline{1-2}

r249	& k1 	& Supervisor-only general registers	& read/write\\
\cline{1-2}

r250	& k2 	&		&\\
\cline{1-2}

r251	& k3 	&		&\\
\hline

r252	& e0 	& Program counter shadow register.  
When the exception disable flag is equal to 0, this register is updated 
every cycle with a copy of the program counter.		&\\
\cline{1-3}

r253	& e1 	& Interrupt mask shadow register.
When the exception disable flag is equal to 0, this register is updated 
every cycle with a copy of the interrupt disable flag in bit 0.
The remaining bits are reserved. 		&read only\\
\cline{1-3}

r254	& e2 	& TLB latch register: When the exception disable
flag is equal to 0, this register is updated with every address sent
to the memory system.  Therefore, if any memory exception occurs,
this register will contain the address that caused the problem.  &\\
\cline{1-3}

r255	& e3	& Exception code register.  See Figures \ref{e3-table}
			and \ref{exceptions-table} for more information. &\\
\hline
\end{tabular} \linebreak
\end{center}
\end{figure}

When an exception occurs:

\begin{itemize}

\item The exception registers {\tt e0-e3} contain the exception
	context, as described above.  In particular, {\tt e3} contains
	information about the type of exception that occurred.  {\tt
	e3} can be interpreted as shown in Figures \ref{e3-table} and
	\ref{exceptions-table}.

	Note that at most one of bits 1-3 of {\tt e3} will be set. 
	None of the three bits will be set if the exception was due to
	anything other than a memory access.

\item The interrupt disable flag and the exception disable flag are
	set.

\item Execution continues at the exception handler address.

\end{itemize}

Note that the supervisor general registers {\tt k0-3} are not modified
by the hardware at exception time; these registers are available to be
used by the exception handling code.

The intended exception handling model is that at exception entry,
registers {\tt k0-3} are used as scratch space; at exception exit,
these registers are used to hold the operands of the {\tt rfe}
instruction.  Register {\tt k0} is not destroyed and can be used to
hold information such as the current supervisor-mode stack pointer,
thread structure address, or the like.  In SMP systems, this avoids
the need to look up such things based on CPU number.

The {\tt rfe} (return from exception) instruction clears the exception
disable flag and takes three register arguments:

\begin{center}
\begin{tabular}{l l}
1st argument	&this register is used to load the program counter.\\
2nd argument	&bit 0 of this register is loaded into the interrupt disable flag.\\
3rd argument	&bit 0 of this register is loaded into the user-mode flag.\\
\end{tabular}
\end{center}

For compatibility with possible future revisions with expanded
interrupt masks, the entire contents of the {\tt e1} register should
be saved at exception time and made available to {\tt rfe} for
restoration.

The interrupt disable flag inhibits interrupts.  The exception disable
flag inhibits further exceptions; if an exception occurs while the
exception disable flag is set, the processor resets.

It is not meaningful to have exceptions disabled and interrupts
enabled at the same time.  Such a configuration may produce undefined
behavior.

The exception disable flag is intended to prevent nested exceptions
causing an unrecoverable loss of data:  if an exception were to occur
in an exception handler before {\tt e0-3} could be saved, their
contents, and thus important information about the state of some
process, would be overwritten.  The exception disable flag should be
cleared as soon as the exception handler has saved all the necessary
state to be able to resume if interrupted.  The exception disable
flag should then be re-asserted before reloading the general-purpose
registers before returning from exception, in order to ensure that these
registers are properly restored.

The result of this arrangement is that the only way to fatally crash
the processor is to have a mistake in the exception entry/exit code. 
This is, while not ideal, definitely to be preferred to architectures
where an error such as stack overflow in supervisor mode at any time
can cause a processor reset.  There is one drawback, however, which is
that the exception handler must appear in one of the unmapped
supervisor segments.


\begin{figure}[hpt]
\caption{\label{e3-table} Exception Register {\tt e3}}

\begin{center}
\begin{tabular}{|l|lp{3.5in}|}
\hline

bits [31..4]	& \multicolumn{2}{|l|}{Exception Type
			(see Figure \ref{exceptions-table})}\\
\hline

bit 3		&DF	&= 1, if the exception occurred on a data fetch.\\
\hline
bit 2		&DS	&= 1, if the exception occurred on a data store.\\
\hline
bit 1		&IF	&= 1, if the exception occurred on an instruction fetch.\\
\hline
bit 0		&SU	&= 1, if the processor was in supervisor mode when the exception occurred.\\
\hline
\end{tabular}
\end{center}

% \end{figure}

% \begin{figure}[hbp]
\caption{\label{exceptions-table}
	Valid Values for Bits [31..4] of Exception Register {\tt e3}}

\begin{center}
\begin{tabular} {|l|l|p{4.0in}|}
\hline
{\em Exception type}	& {\em Value}	& {\em Description} \\
\hline
IRQ      		& 1    & Generated when the IRQ line on
				the processor is raised and the interrupt
				disable flag is not set. \\
\hline
Bus error       	& 3	 & Generated when the
				physical memory bus (external to the processor)
				rejects a memory access. \\
\hline
Illegal Instruction     & 4	& Generated when instruction decoding fails. \\
\hline
Privileged Instruction  & 5	& Generated when an instruction that 
				is reserved for supervisor mode is
				encountered in user mode. \\
\hline
Trap 			& 6     & Generated when a trap instruction is executed.
				\\
\hline
Divide by Zero		& 7	& Generated by a division or mod by zero. \\
\hline
Alignment Error		& 8	& Generated if a memory
				access is generated that does not meet alignment
				requirements, that is, if an $n$-byte access is 
				not aligned on an $n$-byte boundary. \\
\hline
Segment Privilege	& 9	& Generated if a reference to segments 1,
				2, or 3 is attempted while in
				user mode. \\
\hline
Register Violation	& 10	& Generated by
				trying to write a supervisor-only
				register while in user mode,
				or trying to write a read-only
				register other than {\tt r0}. \\
\hline
TLB Miss        	& 11	& Generated when address
				translation via the TLB cannot find a matching
				valid TLB entry. \\
\hline
TLB Protection		& 12	& Generated when address translation via the TLB
				finds a valid matching TLB entry, but the
				entry prohibits the requested operation.
				\\
\hline
TLB Multiple Match	& 13	& Generated when a TLB probe detects two (or more)
				TLB entries for the probe address.\\
\hline
TLB Invalid Index	& 14	& Generated when an invalid TLB index is
				used with {\tt tlbse} or {\tt tlble}. \\
\hline
Timer Exception		& 20	& Generated when the timer expires. \\
\hline
\end{tabular}
\end{center}
\end{figure}

Notes about particular exceptions:

\begin{description}

\item[IRQ] If the IRQ line remains asserted when the interrupt disable
	flag is cleared again, another IRQ exception will occur. 
	However, at least one IRQ exception will occur if the IRQ line
	is raised, even if it is subsequently lowered before the
	exception occurs.

\item[Trap] The low 24 bits of the trap instruction are not
	interpreted by the processor and may thus be used by the
	operating system.

\item[Illegal Instruction]

	An illegal instruction exception occurs when an instruction
	with an invalid opcode is encountered, or when an invalid
	operand to a valid opcode is encountered.  Illegal operands
	include non-existent registers, or odd-numbered registers
	to instructions that use register pairs.

\end{description}

